%%%%%%%%%%%%%%%%%%%%%%%%
\section{Introduction}
\label{sec:intro}
%%%%%%%%%%%%%%%%%%%%%%%%
%
Inclined and horizontally- or vertically-spanned cylinder are used in various
engineering applications: cable-stayed, suspension, and tied-arch bridges,
power transmission lines, offshore risers, sub-sea pipelines etc. These
cylinders are prone to large-amplitude flow-induced vibration, which can lead
to catastrophic failure of the cylinders and the structures supported by them.
The vibration mechanisms involve complex aeroelastic (motion-induced)
interactions that depend on the spatial orientation, geometry,
surface-characteristics, and dynamic properties of
cylinders. \cite{davenport1995dynamics} shows that large amplitude vibrations
can lead to either catastrophic- or fatigue failure of the cylinders and/or the
adjoining structures, which poses a significant threat to the safety and
serviceability of these systems.

Wind induced vibration can be classified into the following categories:
K\'arm\'an vortex-induced vibration (VIV), rain-wind induced vibration (RWIV),
wake-induced vibration, and dry- or iced-cable galloping. While VIV and RWIV
occur at relatively low wind speeds ($<20$ m/s), wake-induced vibration and
dry-cable galloping occur at higher speeds. Simulation of any of these
aeroelastic phenomena requires accurate computation of the aerodynamic loads
acting on the cable.  This paper presents a comprehensive verification of a
high-fidelity flow simulation technique (detached eddy simulations or DES) for
flow over rigidly and elastically-mounted rigid circular cylinders in normal
and yawed flow. Yaw angle ($\beta$) is defined as the angle between the
cylinder axis and a vector orthogonal to the flow velocity vector in the plane
of the cylinder axis. Yawed flow is considered because RWIV and dry-cable
galloping occur only when $\beta$ is non-zero. The diameter-based Reynolds
number ($Re_D$) in the simulations is $2\times 10^4$. At this $Re_D$, the
boundary layer is laminar when it separates and transitions to turbulence in
the free shear layer. The results of this paper are presented in two parts -
the first part deals with rigidly-mounted (static) cylinders and the second
part addresses elastically-mounted cylinders.

In the first part of the paper, simulations of flow over a static cylinder are
verified with measurements from the literature as well as new experiments that
are performed for $\beta=0^\circ$ and $30^\circ$ as a part of this study.
Comparisons between the simulations and the measurements include mean and rms
of surface pressure, mean wake velocity profile, and power spectra of lift and
drag. The agreement with the measured data ranges from very good to acceptable.
Simulations are performed for four values of $\beta$
($=0^\circ,~15^\circ,~30^\circ,$ and $45^\circ$) and the validity of the {\em
independence principle} (IP). According to IP, the aerodynamic loads scaled
using the velocity component normal to the cylinder axis are independent of
$\beta$ (\citet{zdravkovich2003flow}). Prior research (see e.g.,
\citet{franzini2009experimental} and \citet{zhao2009direct}) has suggested the
validity of IP for $\beta$ up to $45^\circ$. The results of yawed flow
simulations confirm the validity of IP for the static configurations analyzed.

The second part of the paper deals with vortex-induced vibrations (VIV).  VIV
are self-excited oscillations that occur due to the unsteady lift generated on
a cylinder as a result of the vortex shedding in its wake. VIV is a very common
occurrence in many engineering structures, e.g., transmission lines, stacks,
bridges, and offshore platform supports. Of particular interest is the
phenomenon of ``lock-in'' or ``synchronization'' which occurs when the
vortex-shedding frequency gets locked in with the natural frequency of the
system. High but finite-amplitude (limit-cycle) oscillations are observed in
this resonance condition due to non-linearity in the system. These
high-amplitude oscillations can cause significant damage to the structure. VIV
has been investigated extensively in the past few decades due to its practical
significance. \cite{bearman1984vortex},~\cite{sarpkaya2004critical} and
\cite{williamson2004vortex} present detailed reviews of the research on this
topic.

Investigations of VIV over the past several decades, have resulted in an
improved understanding of the phenomenon. Until recently, most of these
investigations considered relatively high mass-damping systems. In the last
two decades, VIV of low mass-damping systems has garnered considerable
interest due to the applicability of the phenomenon in marine engineering and
offshore wind turbines (mooring lines). Such low mass-damping systems have been
largely investigated using experiments (e.g., \citet{jain2013vortex} and
\citet{franzini2013one}) and semi-analytical methods that model vortex shedding
using the classical Van der Pol oscillator (e.g., ~\cite{xu2008high}). 

Experiments of VIV in low mass-damping systems have been typically conducted in
water channels with partially or fully-submerged cylinders. Such setup can
suffer from finite-span effects due to different end conditions on the two ends
-- wall on one side and free surface on the other. These effects are distinctly
observable in yawed flow measurements (see e.g.,~\citet{franzini2013one}) where
the results are found to be dependent on the direction with respect to the flow
the cylinder is yawed (i.e., $\beta=45^\circ$ versus $-45^\circ$). Simulations
do not suffer from end effects as periodic boundary conditions enable
simulating an infinitely long cylinder. 

Numerical computations of VIV of low mass-damping systems have been performed
using the Reynolds Averaged Navier-Stokes (RANS) equations (e.g.,
\citet{guilmineau2004numerical}), large eddy simulations or LES (e.g.,
\citet{al2004vortex}), and direct numerical simulations or DNS (e.g.,
\citet{lucor2005vortex}, \citet{evangelinos2000dns}, and \citet{dong2005dns}).
While the DNS approach is attractive from the perspective of resolving the
entire turbulence spectrum, it is only feasible for very low $Re_D$ flows due
to the associated computation cost. Large Eddy Simulation (LES) is more
feasible at moderate $Re_D$ while Detached Eddy Simulations (DES) can be used
over a very large range of $Re_D$ without making the computations prohibitively
expensive. The DES approach is used here.

The phenomenon of VIV is investigated with elastically-mounted cylinder
simulations and the results are verified against measured data available in the
literature. The VIV simulations are also performed at multiple values of
$\beta$ ($=0^\circ$ and $45^\circ$) to investigate the validity of the
independence principle for VIV of low mass-damping systems. With the detailed
comparisons against measurements for rigidly-mounted and elastically-mounted
oscillating cylinder presented here, this paper can serve as a benchmark for
the accuracy of the DES technique in predicting aerodynamic loads on cables in
static condition and in dynamic motion.
%%%%%%%%%%%%%%%%%%%%%%%%%%%%%%%%%%%%%%%%%%
