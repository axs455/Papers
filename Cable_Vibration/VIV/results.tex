%%%%%%%%%%%%%%%%%%%%%%%%%%%%%%%%%%%%%%%%%%%%%%%%%%%%%%%%%%%%%
\section{Numerical Results and Verification with Measured Data}
\label{sec:comp_results}
%%%%%%%%%%%%%%%%%%%%%%%%%%%%%%%%%%%%%%%%%%%%%%%%%%%%%%%%%%%%%
%
The objective of this paper is to demonstrate the capability of detached eddy
simulations to predict aerodynamic loads on a static cylinder and an
elastically-mounted cylinder.  Verification with existing experimental data in
the literature, and data from new experiments conducted as a part of this
study, are presented for both the static cylinder and the elastically-mounted
cylinder. Smooth inflow is used -- zero turbulence in the numerical simulations
and the minimum possible inflow turbulence intensity ($\sim$0.2\%) in the
tunnel. Cylinder surface is very smooth and hence surface roughness is not
modeled in the simulations.

Static testing is performed for (1) flow normal to the cylinder axis, and (2)
flow at an angle to the cylinder axis (inclined cylinder); three inclination
angles are analyzed in this study. These cases are simulated at Reynolds number
$Re_D=20,000$ which is corresponding to laminar separation (LS), where flows
are laminar before separation and transition to turbulence occurs in the shear
layer.

Dynamic testing is performed for an elastically-mounted cylinder in eight
different inflow reduce velocities with (1) flow normal to the cylinder axis,
and (2) flow at an angle to the cylinder axis (inclined cylinder). However, due
to high computational cost of simulations, only one inclination angle is
analyzed in the dynamic study($45^\circ$). The cylinder is limited to vibrate
along the direction which is perpendicular to both the flow direction and the
cylinder axis. All dynamic cases are simulated at the same Reynolds number,
$Re_{D,n}=20,000$.
%%%%%%%%%%%%%%%%%%%%%%%%%%%%%%%%%%%%%%%%%%
% input section
%%%%%%%%%%%%%%%%%%%%%%%%%%%%%%%%%%%%%%%%%%%%%%%%%%%%%%%%%%%%%
\section{Static-Cylinder Results}
\label{sec:comp_results}
%%%%%%%%%%%%%%%%%%%%%%%%%%%%%%%%%%%%%%%%%%%%%%%%%%%%%%%%%%%%%
%
The objective of this paper is to demonstrate the capability of DES to predict
aerodynamic loads on a static cylinder and an elastically-mounted cylinder.
This section discusses the results of the simulations of flow over a static
cylinder.

Simulations are performed for (1) flow normal to the cylinder axis, and (2)
flow at an angle to the cylinder axis (inclined/yawed cylinder); three yaw
angles ($\beta$) are analyzed in this study. Smooth inflow is used -- zero
turbulence in the numerical simulations and the minimum possible inflow
turbulence intensity ($\sim$0.2\%) in the tunnel. Cylinder surface is very
smooth and hence surface roughness is not modeled in the simulations.
Verification is performed with existing experimental data in the literature, as
well as data from new experiments conducted as a part of this study.

%%%%%%%%%%%%%%%%%%%%%%%%%%%%%%%%%%%%%%%%%%
\subsection{Normally-Incident Flow}
\label{sec:normally_incident}
%%%%%%%%%%%%%%%%%%%%%%%%%%%%%%%%%%%%%%%%%%
%
Table~\ref{tab:comparisonRe20k} summarizes the simulation results for the
static cases and compares them with two sets of experimental data. The peak
shedding Strouhal number ($St_p$), the mean drag coefficient, $\overline{C}_d$
and the mean back pressure coefficient, $\overline{C}_{pb}$ are compared in the
table. Strouhal number is a non-dimensional frequency, defined as $St =
f\,D/V_\infty$, and $St_p$ corresponds to the peak vortex shedding frequency of
the cylinder.

Figure~\ref{fig:Cp_compared_Re20k} compares the predicted mean aerodynamic
pressure coefficient, $\overline{C}_p$ and the root mean square of perturbation
pressure coefficient, $C_{p'rms} = \left(\,\overline{C^2_p -
\overline{C}^2_p}\,\right)^{1/2}$ with the data from the two experiments. The
predicted $\overline{C}_p$ agrees very well with the data from Experiment I;
Exp-ISU data shows slightly lower $\overline{C}_p$ than observed in
Experiment I and the simulation, especially after $100^\circ$, and the mean
back pressure, $\overline{C}_{pb}$ is lower as well. The predicted
$C_{p'rms}$ distribution lies in between the two measurements. Both
measurements as well as the simulation show the peak to be around $80^\circ$,
which indicates the location of flow separation. The predicted distribution
over the cylinder surface agrees well with the measurements.
%
\begin{table}[htb!]
  \caption{Summary of results for normally-incident flow simulations} 
  \label{tab:comparisonRe20k} 
  \begin{center}
  \begin{tabular}{c|c|c|c|c}
      $\boldsymbol{Re_D}$ & \textbf{Method} & $\boldsymbol{St_p}$ & $\boldsymbol{\overline{C}_d}$ & $\boldsymbol{\overline{C}_{pb}}$ \\ \hline
      \hline
      20,000 & $k-\omega$ DDES & 0.21 & 1.13 & -1.16 \\ \hline
      20,000 & Experiment I    & 0.19 & 1.22 & -1.1  \\ \hline
      51,500 & Exp-ISU         & 0.21 & 1.14 & -1.3  \\ \hline
      \hline
  \end{tabular}
  \end{center}
\end {table}

\begin{figure}[htb!]
  \centering
  \subcaptionbox{Mean aerodynamic pressure coeff., $\overline{C}_p$}
    [.48\linewidth]{\incfig[width=.48\textwidth]{Figures/Cp_Compared.png}}
  \hspace*{\fill}
  \subcaptionbox{r.m.s. of pressure coeff., $C_{p'rms}$ }
    [.48\linewidth]{\incfig[width=.48\textwidth]{Figures/CpRMS_Compared.png}}
  \caption{Comparisons of mean and rms of aerodynamic pressure coefficient
  between the simulation and experimental measurements.}
  \label{fig:Cp_compared_Re20k}
\end{figure}

Figure~\ref{fig:velocity_Normal_Re20k} plots the predicted and measured wake
velocity profiles at the axial station, $x/D=2$; the cylinder axis is located
at $x/D=0$. The peak wake deficit and the wake profile are predicted
accurately. The measured data shows a slight asymmetry in the data, which is
perhaps due to an asymmetry in the experimental setup (the distance from the
tunnel wall between the top and bottom surfaces of the cylinder is slightly
different). The simulation data is averaged over 120 wake shedding cycles and
the experimental data is averaged over 540 cycles.
%
\begin{figure}[htb!]
  \incfig[width=.5\textwidth]{Figures/velocity_Normal_Re20k.png}
  \caption{Comparison of predicted and measured velocity profiles in the
    cylinder wake $2D$ downstream of the cylinder axis.}
  \label{fig:velocity_Normal_Re20k}
\end{figure}

Figure~\ref{fig:force_20k} presents predicted temporal variation of sectional
lift and drag coefficients ($C_l$ and $C_d$). As expected for a circular
cylinder, the mean lift coefficient ($\overline{C}_l$) is zero but the mean
drag coefficient ($\overline{C}_d$) is finite. The high-frequency oscillations,
more apparent in $C_l$ time history are due to K\'arm\'an vortex shedding, which
occurs at $St=fD/U \sim 0.2$ for bluff bodies in the range of $Re_D$ considered
here. In addition to the oscillations at the K\'arm\'an vortex shedding frequency,
the entire signal appears to modulate at a frequency which is an order of
magnitude lower than that corresponding to $St=0.2$. This modulation has a
certain randomness to it and is not perfectly periodic. This modulation
phenomenon has been reported elsewhere, see e.g.,~\cite{travin2000detached}. 
%
\begin{figure}[htb!]
  \incfig[width=0.9\textwidth]{Figures/force_20k.png}
  \caption{Predicted temporal variations of lift and drag coefficients}
  \label{fig:force_20k}
\end{figure}

Figure~\ref{fig:St_Compared_Re20k} (a) compares the power spectral densities of
$C_l$ between data from Exp-ISU and the simulation.  The lift in the
measurements is obtained by integrating the surface pressure measured using the
pressure taps. Figure~\ref{fig:St_Compared_Re20k} (b) presents the DES computed
spectra of $C_d$. Because vortex shedding alternates between the top and bottom
sides of the cylinder, one vortex shedding period contains two cycles of drag
but only one cycle of lift. This can be seen in
Figure~\ref{fig:St_Compared_Re20k}, where the spectral peak for $C_l$ occurs at
$f_p$ while the spectral peak for drag is at $2\times f_p$, where $f_p$ is
determined by the peak shedding Strouhal number, $St_p = f_p\,D/V_\infty$.
Based on existing literature (see~\cite{travin2000detached}
and~\cite{norberg2013pressure}), $St_p \sim 0.2$. Both measurement and
prediction agree very well with each other and show the peak for lift to be
around $f_p$ corresponding to $St_p$.
%
\begin{figure}[htb!]
  \centering
  \subcaptionbox{PSD of $C_l$ }
    [.48\linewidth]{\incfig[width=.48\textwidth]{Figures/St_Compared_Cl_Re20k.png}}
  \hspace*{\fill}
  \subcaptionbox{PSD of $C_d$}
    [.48\linewidth]{\incfig[width=.48\textwidth]{Figures/St_Compared_Cd_Re20k.png}}
    \caption{Comparison of predicted and measured power spectral densities
      (PSDs) of $C_l$ and $C_d$. The
      measured data here is from our experiments (Exp-ISU).}
\label{fig:St_Compared_Re20k}
\end{figure}

The peak frequency and its first three harmonics that occur at $St = 0.4, 0.6,
\& \,0.8$, are identified in the figure using vertical grid lines and labeled
as $2f_p,\,3f_p,\,\&\,4f_p$. The prediction and experiment both show a second,
smaller peak in the lift spectrum at the third harmonic ($St=0.6$).  Since the
lift vector alternates with the side the vortex sheds from, only odd harmonics
of $f_p$ (i.e., $3f_p,~5f_p,\ldots$) are expected in the spectra. Therefore, no
peak is observed in the lift spectra at the second harmonic ($St=0.4$) or
higher {\em even} harmonics in either the measured or the simulated data. The
spectral shape of the PSD of $C_l$ is correctly predicted, even though the
magnitude is slightly higher than the measured data.

%%%%%%%%%%%%%%%%%%%%%%%%%%%%%%%%%%%%%
\subsection{Yawed Flow (Inclined Cylinder)}
\label{sec:inclined_cylinder}
%%%%%%%%%%%%%%%%%%%%%%%%%%%%%%%%%%%%%
%
The schematic in Fig.~\ref{fig:yawedModel} illustrates the setup for the
inclined-cylinder simulations. The relative inclination of the cylinder axis
with respect to the flow is obtained by yawing the flow rather than inclining
the cylinder; these simulations are therefore also referred to as yawed-flow
simulations. Other than yawing the inflow, the setup is exactly the same as for
normally-incident flow.

Yaw angle, $\beta$ is defined as the angle between the inflow velocity vector
$\boldsymbol{V_\infty}$ and the $x$ axis; the cylinder is aligned with the $z$
axis (see Fig.~\ref{fig:yawedModel}. The normal component of the flow velocity
is $V_n=V_\infty \cos\,\beta$ and the spanwise component is $V_z = V_\infty
\,\sin\,\beta$, where $V_\infty=\abs{\boldsymbol{V_\infty}}$.  The
computational domain is $10\times D$ in the spanwise direction to resolve and
investigate spanwise variation of aerodynamic forces.
%
\begin{figure}[htb!]
  \incfig[width=0.6\textwidth]{Figures/yawedModel.png}
  \caption{A schematic of the computational setup for static inclined cylinder
    simulations. The right figure is a cross-sectional view.  The inflow is set
    to an angle with respect to the cylinder axis, which stays aligned with the $z$
    axis of the coordinate system.}
  \label{fig:yawedModel}
\end{figure}

Table~\ref{tab:comparisonYawedRe20k} summarizes the peak Strouhal number
($St_p$) and the back pressure coefficient ($C_{pb}$) for four different flow
yaw angles, $\beta=0,15,30,\,\&\,45$ degrees. The velocity component normal to
the cylinder axis ($V_n$) is used as the reference velocity scale to define a
new set of non-dimensional quantities, such as Reynolds number, $Re_{D,n} =
\rho V_n D / \mu$, Strouhal number, $St_{p,n}=f\,D /V_n$, and aerodynamic
pressure coefficient, $\overline{C}_{p,n}=2 (\overline{p}-p_\infty)/(\rho
V_n^2)$. The peak shedding Strouhal number and the mean back pressure
coefficient normalized in this manner are labeled respectively as $St_{p,n}$
and $\overline{C}_{pb,n}$. The measured value of $\overline{C}_{pb,n}$ is lower
than that predicted by the simulations (see
Table~\ref{tab:comparisonYawedRe20k}).
%
%%%%%%%%%%%%%%%%
\begin{table}[htb!]
  \caption{Summary of simulation results for four different flow yaw angles
  ($\beta=0, 15, 30,~\&~45$ deg). Experimental data is only shown for
  $\beta=30^\circ$.} 
  \label{tab:comparisonYawedRe20k} 
  \begin{center}
  \begin{tabular}{c|c|c|c|c|c}
      \textbf{Method} & \textbf{flow angle,} $\boldsymbol{\beta}$ & $\boldsymbol{Re_D}$ & $\boldsymbol{Re_{D,n}}$  & $\boldsymbol{St_{p,n}}$ & $\boldsymbol{\overline{C}_{pb,n}}$  \\ \hline
      \hline
      Simulation  & $0^\circ$  & 20,000 & 20,000 & 0.21 & -1.15 \\ \hline
      Simulation  & $15^\circ$ & 20,000 & 19,318 & 0.21 & -1.11 \\ \hline
      Simulation  & $30^\circ$ & 20,000 & 17,320 & 0.20 & -1.11  \\ \hline
      \rowcolor[gray]{.9}
      Exp-ISU     & $30^\circ$ & 51,500 & 44,600 & 0.19 & -1.27  \\ \hline
      Simulation  & $45^\circ$ & 20,000 & 14,142 & 0.21 & -1.16  \\ \hline
      \hline
  \end{tabular}
  \end{center}
\end{table}

Figure~\ref{fig:Cp_Compared_Yawed_Exp-ISU} compares with measured data the
predicted mean aerodynamic pressure coefficient ($\overline{C}_{p,n}$) and root
mean square of perturbation pressure coefficient, $C_{p'rms,n}$ for
$\beta=30^\circ$. The predicted back pressure ($\overline{C}_{pb,n}$) is found
to be slightly higher than Exp-ISU data, which is consisted with the
observation for the normally-incident flow cases. The predicted $C_{p'rms,n}$
distribution agrees very well with measurement, especially for
$\theta>120^\circ$, where $\theta$ is the angular position on the cylinder
surface measured from upstream. The peak of $C_{p'rms,n}$ is observed around
$80^\circ$ in both experiment and simulation, which is indicative of the
location of separation of the shear layer. For $\theta > 120^\circ$, the
measured data shows higher $C_{p'rms,n}$ than predicted by the simulations. A
similar underprediction is observed in the normally-incident flow case.
%
\begin{figure}[htb!]
  \centering
  \subcaptionbox{Mean pressure distribution,$\overline{C}_{p,n}$}
    [.48\linewidth]{\incfig[width=.48\textwidth]{Figures/Cp_Compared_Yawed_Exp-ISU.png}}
  \hspace*{\fill}
  \subcaptionbox{rms of perturbation pressure, $C_{p'rms,n}$ }
    [.48\linewidth]{\incfig[width=.48\textwidth]{Figures/CpRMS_Compared_Yawed_Exp-ISU.png}}
  \caption{Comparisons between simulation and experimental measurements for
    $\beta=30^\circ$ yawed-flow case.}
\label{fig:Cp_Compared_Yawed_Exp-ISU}
\end{figure}

%Figure~\ref{fig:velocity_Yawed30_Re20k.png} presents predicted and measured
%wake velocity profiles for $\beta=30^\circ$ case at $x/D=2$. The predicted peak
%wake deficit matches remarkably well with Exp-ISU data. However, the
%experimental data shows higher overshoots in streamwise velocity in the shear
%layer region than predicted by the simulations. The experimental data is also
%very slightly asymmetric, which is likely due to the fact that the cylinder is
%located closer to one side of the tunnel wall. It should be noted that the
%asymmetry is very small and the wall effects are minimal.
%%
%\begin{figure}[htb!]
%  \incfig[width=.5\textwidth]{Figures/velocity_Yawed30_Re20k.png}
%  \caption{Comparison of predicted and measured velocity profiles for
%    $\beta=30^\circ$ yawed flow in the cylinder wake $2D$ downstream of the cylinder axis}
%  \label{fig:velocity_Yawed30_Re20k.png}
%\end{figure}
%
%Figure~\ref{fig:St_Compared_Yawed30_Exp-ISU} compares the power spectral
%densities of the transverse force coefficient (along the $y$ axis), $C_{y,n}$
%for $\beta=30^\circ$ case between Exp-ISU data and predictions, where $C_{y,n}
%= 2\, F_y/ \left( \rho V^2_n (D \times L) \right)$, $F_y$ is the net force over
%the entire cylinder; longitudinal force coefficient, $C_{x,n}$ is similarly
%defined. In the simulation, the first peak is observed around $St_{p,n}\sim
%0.2$, which is the same as for the normally-incidence flow case (see
%Figure~\ref{fig:St_Compared_Re20k} (a)). The spectral shape is correctly
%predicted by the simulation although the measured curve appears to be shifted
%along the $x$ axis; this is likely due to a scaling factor in frequency (log
%scale is used for frequency in Fig.~\ref{fig:St_Compared_Yawed30_Exp-ISU}),
%arising perhaps from a slight mismatch in the measurement of the inflow
%velocity in the experiment.
%%
%\begin{figure}[htb!]
%  \incfig[width=.6\textwidth]{Figures/St_Compared_Yawed30_Exp-ISU.png}
%  \caption{Comparison of predicted and experimental power spectral densities
%    (PSDs) of force coefficient $C_{y,n}$ for $\beta=30^\circ$ yawed-flow
%    cases.}
%  \label{fig:St_Compared_Yawed30_Exp-ISU}
%\end{figure}

Figure~\ref{fig:Compared_Yawed_Re20k} compares the predicted mean aerodynamic
pressure coefficient ($\overline{C}_{p,n}$), for four different values of
inflow yaw angle, $\beta$. The distribution of $\overline{C}_{p,n}$ is found to
be very similar irrespective of $\beta$;~\cite{zdravkovich2003flow} refers to
this as `independence principle' (IP). IP is also observed in the power
spectral densities of the transverse force coefficient, $C_{y,n}$ for the same
set of values of $\beta$ analyzed. $C_{y,n}=2\, f_y/ (\rho V^2_n)$, where $f_y$
is force per unit area in the $y$ direction, and $C_{x,n}$ is correspondingly
defined for the $x-$component of force.  Figure~\ref{fig:Compared_Yawed_Re20k}
(b) shows that the spectra collapse when $V_n$ is used to normalize the force
coefficients and the frequency; the abscissa in
Fig.~\ref{fig:Compared_Yawed_Re20k} (b) is $St_n$.  
%
\begin{figure}[htb!]
  \subcaptionbox{Mean pressure coeff., $\overline{C}_{p,n}$}[0.48\linewidth]
    {\incfig[width=.48\textwidth]{Figures/Cp_Compared_Yawed_Re20k.png}}
  \hspace*{\fill}
  \subcaptionbox{PSDs of $C_{y,n}$}[0.48\linewidth]
    {\incfig[width=.48\textwidth]{Figures/St_Compared_Yawed_Cl_Re20k.png}}
  \caption{Independence principle: comparisons of (a) $\overline{C}_{p,n}$, and
    (b) power spectral densities (PSDs) of $C_{y,n}$ between predictions for various $\beta$ values.}
\label{fig:Compared_Yawed_Re20k}
\end{figure}

Figure~\ref{fig:Spatial_temporal_CxCyRe20k} shows spatio-temporal plots of the
the force coefficients $C_{x,n}$ and $C_{y,n}$.  The contours clearly show that
$C_{x,n}$ and $C_{y,n}$ vary along the span, indicating that vortex shedding
does not occur simultaneously along the entire span. In fact, a spatial drift
from left to right with increasing time can be seen in the contours (more
visible in the $C_{x,n}$ spatio-temporal plot) which is indicative of spanwise
flow over the cylinder.
%
\begin{figure}[htb!]
  \subcaptionbox {$C_{x,n}$}
    [.48\linewidth]{\incfig[width=.48\textwidth]{Figures/Spatial_temporal_Cx_Re20k_Yawed30.png}}
  \hspace*{\fill}
  \subcaptionbox{$C_{y,n}$ }
    [.48\linewidth]{\incfig[width=.48\textwidth]{Figures/Spatial_temporal_Cy_Re20k_Yawed30.png}}
  \caption{Spatio-temporal distribution of force coefficients at
  $\beta=30^\circ$}
  \label{fig:Spatial_temporal_CxCyRe20k}
\end{figure}

Figure~\ref{fig:Coherence_Yawed30_Re20k} presents coherence of force
coefficients for $\beta=30^\circ$ case. Magnitude squared coherence,
$\gamma^2(\Delta z)$ is defined as
%
\begin{align}
  \gamma^2(\Delta z) &= \frac{\langle \abs{S_{xy}}^2 \rangle}{\langle
    S_{xx}\rangle \langle S_{yy} \rangle},
  \label{eq:coherence}
\end{align}
%
where $S_{xy}$ denotes cross-spectral density of the quantity ($C_{x,n}$ or
$C_{y,n}$) at two points separated by a distance $\Delta z$, and
$S_{xx},~S_{yy}$ are auto-spectral densities; angular brackets denote ensemble
averaging, however ergodicity assumption is used to relate that to time
averaging. The coherence plot of $C_{y,n}$ indicates that spanwise correlation
is very high (over nearly the entire cylinder span) at the vortex shedding
frequency, but is small at other frequencies, which is expected based on
literature. $C_{x,n}$ however is not that highly correlated along the span even
at the peak vortex shedding frequency.
%
\begin{figure}[htb!]
  \subcaptionbox {$C_{x,n}$}
    [.48\linewidth]{\incfig[width=.48\textwidth]{Figures/Cd_coherence_Yawed30_Re20k.png}}
  \hspace*{\fill}
  \subcaptionbox{$C_{y,n}$ }
    [.48\linewidth]{\incfig[width=.48\textwidth]{Figures/Cl_coherence_Yawed30_Re20k.png}}
    \caption{Magnitude squared coherence, $\gamma^2(\Delta z)$ of transverse
      and longitudinal force coefficients, $C_{x,n}$ and $C_{y,n}$ for
      $\beta=30^\circ$ case.}
  \label{fig:Coherence_Yawed30_Re20k}
\end{figure}
%%%%%%%%%%%%%%%%%%%%%%%%%%%%%%%%%%%%%%%%%%%%%%%%%%


% input section
%%%%%%%%%%%%%%%%%%%%%%%%%%%%%%%%%%%%%%%%%%%%%%%%%%%%%%%%%%%%%
\section{Vortex-Induced Vibration Results}
\label{sec:VIV}
%%%%%%%%%%%%%%%%%%%%%%%%%%%%%%%%%%%%%%%%%%%%%%%%%%%%%%%%%%%%%
%
A schematic of the computational setup for vortex-induced vibration (VIV)
simulations is presented in Figure~\ref{fig:VIVmodel}. The setup is the same as
for the static simulations except for an additional mass-spring-damper system
that determines the motion of the cylinder. The cylinder is constrained to move
only in the cross-stream ($y$) direction. The $y$-component of the integrated
aerodynamic surface force on the cylinder drives the mass-spring-damper system
given by Eq.~\ref{eq:solidBodyDynamics}. Simulations are performed for eight
inflow reduced velocities and two values of yaw angle, $\beta$ ($=0^\circ$ and
$45^\circ$).
 
In the simulations the mass ratio $m^*=m/(\rho {\cal V})=2.6$, where $m$ is the
mass of the cylinder, ${\cal V}=\pi (D^2/4) . S$ is the volume of the cylinder,
$S$ is the cylinder span, and $\rho$ is the density of the fluid flowing over
the cylinder. The mechanical damping ratio of the system $\zeta = c/c_c$ is
$0.001$ where, $c_c=2\sqrt{k.m}$ is the critical damping, and the spring
stiffness $k$ is related to the natural frequency, $f_N$ by $k=m(2\pi\,f_N)^2$.
The non-dimensional parameters, $m^*$, $\zeta$, and reduced velocity, are
selected to match the measurements of~\citet{franzini2013one}. This measurement
dataset is referred as Exp II in this paper. The reduced velocity is varied in
the experiments by changing the freestream flow speed, which changes $Re_D$.
In the simulations, the reduced velocity is varied by changing the spring
stiffness, and $Re_D$ is held constant. In addition to Exp II, the simulation
results are also compared to another dataset reported
in~\cite{khalak1997fluid}, which is referred as Exp III here. The values of
$m^*$ ($=2.4$) and $\zeta$ ($=0.0045$) in Exp III are slightly different from
Exp II and the simulations. 
%
\begin{figure}[htb!]
  \incfig[width=.6\textwidth]{Figures/VIV_setup.jpg}
  \caption{A schematic of the computational setup for oscillating cylinder
    simulations. The right figure shows a cross-sectional view. The inflow is set
    to an angle with respect to the cylinder axis, which is aligned with the
    $z$-axis of the coordinate system.}
  \label{fig:VIVmodel}
\end{figure}

Figure~\ref{fig:VIV_amp} compares the predicted scaled mean displacement
amplitude ($\bar{A}/D$) with the measurements of Exp II and III over a wide
range of reduced velocity, $V_{R,n} = V_n/(f_N\,D)$, where $f_N$ is the natural
frequency of the system. The subscript $n$ refers to the component of the
velocity vector normal to the cylinder axis to accommodate yawed flow. Two
different yaw angle flows are evaluated -- $\beta=0^\circ$ and $45^\circ$ --
both at $Re_{D,n}=20,000$.

\citet{khalak1997fluid} identified the following four distinct branches in the
$\bar{A}/D$ versus $V_{R,n}$ plot of their VIV measurements for the zero-yaw
case: the ``initial excitation'' branch, the ``upper'' branch, the ``lower''
branch, and the ``desynchronization'' branch. These are labeled and identified
with solid black lines as best curve fits of the measured data in
Fig.~\ref{fig:VIV_amp} (a). Note that the variation of $\bar{A}/D$ with
$V_{R,n}$ is topologically different for systems with high $m^*$, e.g.,
measurements of \citet{feng1968measurement} show two branches (multi-valued
solution) in the lock-in regime. For the selected low $m^*$ system, $\bar{A}/D$
grows rapidly with $V_{R,n}$ in the {\em initial excitation} branch, reaches a
peak in the {\em upper} branch, then reduces to 60\% of the peak value in the
{\em lower} branch, and finally drops to a negligible value at higher $V_{R,n}$
in the {\em desynchronization} branch. The current DES simulations agree very
well with the data (particularly with Exp III) in the {\em initial excitation}
and {\em upper} branches. The peak amplitude is well captured and occurs around
$V_{R,n}=4.76$, which corresponds to the peak vortex shedding Strouhal number,
$St_p=0.21$ for a stationary cylinder. The predicted amplitude is slightly
lower than the measurements in the {\em lower} and {\em desynchronization}
branches.  Considering the relatively large differences in the two sets of
measurements (Exp II and Exp III), which provides an estimate of
uncertainity/repeatibility, the prediction accuracy of the simulations is very
good.
%
\begin{figure}[htb!]
  \centering
  \subcaptionbox{$\beta=0^\circ$} {\incfig[width=.47\textwidth]{fig/viv_amp_noyaw.pdf}}
  \qquad
  \subcaptionbox{$\beta=45^\circ$}{\incfig[width=.45\textwidth]{fig/viv_amp_yaw45_wInset.pdf}} \\
    \caption{Comparison of predicted and measured non-dimensional mean
      amplitude, $\bar{A}/D$ over a range of reduced velocities $V_{R,n}$ for
      a) $\beta=0^\circ$, and b) $\beta=45^\circ$. The inset in the plot on the
      right shows the two setups (UP and DN) used in Exp II for yawed-flow
      measurements.} 
  \label{fig:VIV_amp}
\end{figure}

Figure~\ref{fig:VIV_amp} (b) compares the predicted VIV amplitude with the data
from Exp II. The measurements were taken for two different configurations of
the cylinder, which are shown in the inset in the figure. Since the top and
bottom surfaces are not the same, the two configurations are not identical and
the measured data for the two configurations shows a large difference. Since
the setup in the experiment is asymmetric (wall at the bottom and free surface
on the top) and the fact that end plates were not used, end effects
(finite-span effect) might be the reason for the observed differences between
the two configurations. The predictions agree better with the ``Up''
configuration in the {\em initial excitation} and {\em upper} branches, and
with the ``Dn'' configuration in the {\em lower} branch. Exp II did not collect
any data at higher $V_{R,n}$ to test the prediction accuracy in the {\em
desynchronization} branch.

%%%%%%%%%%%%%%%%%%%%%%%%%%%%%%%%%%%%%%%%%%%%%%%%%%
\subsection{Modes of Vortex Shedding}
\label{sec:VIVmodes}
%%%%%%%%%%%%%%%%%%%%%%%%%%%%%%%%%%%%%%%%%%%%%%%%%%
%
As seen in Figs.~\ref{fig:VIV_amp} and~\ref{fig:VIV_yaw}, the amplitude jumps
up from the {\em initial excitation} branch to the {\em upper} branch, and
jumps down from the {\em upper} branch to the {\em lower} branch as $V_{R,n}$
is increased. These jumps occur as the mode in which the vortex shedding occurs
switches between two possible configurations. These modes are illustrated in
Fig.~\ref{fig:VIV_Q} using schematics and iso-surface of the Q-criterion.
In the `2S' mode, two single vortices shed alternately from either
side of the cylinder in one cycle of cylinder motion, while in the `2P' mode,
two pairs of vortices are shed from each side of the cylinder in one cycle.
The 2P mode has been observed in the smoke visualizations of~\citet{brika1993vortex}.
In the {\em initial excitation} branch, vortex shedding occurs in the 2S mode,
while in the {\em lower} branch, it switches to the 2P mode. In the {\em
upper} branch of the lock-in regime, the vortex shedding switches between the
2S and 2P modes. Other modes can also be observed (e.g., 2P+S and P+S modes) in
forced vibration motion.
%
\begin{figure}[htb!]
  \subcaptionbox{2S mode; $V_{R,n}=4$}{\incfig[width=.48\textwidth]{Figures/2S_RV4.png}} \qquad
  \subcaptionbox{2P mode; $V_{R,n}=8$}{\incfig[width=.48\textwidth]{Figures/2P_RV8.png}} \\
%
  \subcaptionbox{$\beta=0^\circ$; $V_{R,n}=4$}{\incfig[width=.48\textwidth]{Figures/Q_normal_RV4_1.png}} \qquad
  \subcaptionbox{$\beta=0^\circ$; $V_{R,n}=8$}{\incfig[width=.48\textwidth]{Figures/Q_Normal_RV8_1.png}} \\
%
  \subcaptionbox{$\beta=45^\circ$; $V_{R,n}=4$}{\incfig[width=.48\textwidth]{Figures/Q_45_RV4_1.png}} \qquad
  \subcaptionbox{$\beta=45^\circ$; $V_{R,n}=8$}{\incfig[width=.48\textwidth]{Figures/Q_45_RV8_1.png}} \\
    \caption{Illustration of the two modes of vortex shedding observed in the
    simulations using schematics in (a) \& (b), and iso-surfaces of the
    Q-criterion for $\beta=0^\circ$ in (c) \& (d) and $\beta=45^\circ$ in (e) \&
    (f).}
  \label{fig:VIV_Q}
\end{figure}
%
%The force coefficients are shown in Fig.~\ref{fig:force_VIV}. It is well known
%that the vibration motion can significantly increase the fluctuation of forces.
%Mean transverse force coefficient $\bar{C}_{x,n}$ for reduce velocities
%$V_{R,N} < 5.9$ have a very similar curve as mean amplitude $\bar{A}/D$ in
%Fig.~\ref{fig:VIV_amp}. However, mean amplitude $\bar{A}/D$ is almost constant
%in the {\em lower} branch while $\bar{C}_{x,n}$ continuously declines. The predicted
%rms of longitudinal force coefficients $C_{y,n,rms}$ has very sharp peak at
%$V_{R,n}=4$, which is slightly different from Exp. II. Similar to $\bar{A}/D$
%at $V_{R,n}=4$, $C_{y,n,rms}$ for $\beta=45^\circ$ is much larger than
%$\beta=0^\circ$. Overall, two yawed angle flow simulations have very similar
%results for small amplitude vibrations. Independent principle can be applied to
%forces coefficients except for the {\em upper} branch ($4 \leqslant V_{R,n}\leqslant
%5.9$).
%%
%\begin{figure}[htb!]
%  \subcaptionbox {$C_{x,n}$}
%    [.48\linewidth]{\incfig[width=.48\textwidth]{Figures/Cd_VIV.png}}
%  \hspace*{\fill}
%  \subcaptionbox{$C_{y,n,rms}$ }
%    [.48\linewidth]{\incfig[width=.48\textwidth]{Figures/Cl_VIV.png}}
%    \caption{Mean transverse and rms of longitudinal force coefficients, $\bar{C}_{x,n}$ and $C_{y,n,rms}$ for
%      $\beta=0^\circ$ and $45^\circ$ cases.}
%  \label{fig:force_VIV}
%\end{figure}
%%%%%%%%%%%%%%%%%%%%%%%%%%%%%%%%%%%%%%%%%%%%%%%%%%
\subsection{Independence Principle in VIV}
\label{sec:IPinVIV}
%%%%%%%%%%%%%%%%%%%%%%%%%%%%%%%%%%%%%%%%%%%%%%%%%%
%
This section investigates the validity of the {\em independence principle} on
displacement amplitude and oscillating frequency of an elastically-mounted
cylinder undergoing VIV. Prior experiments (e.g., ~\citet{jain2013vortex} and
\citet{franzini2013one}) have investigated IP for a circular cylinder.
\citet{zhao2015validity} investigated IP using DNS at Reynolds numbers
($Re_D$=150 and $1,000$). Much higher $Re_D$ is considered here and the DES
methodology used here can scale to significantly larger $Re_D$ without a
substantial increase in computation cost.

The data from \citet{franzini2013one} (Exp II) is plotted in
Fig.~\ref{fig:VIV_yaw} (a) and the DES predictions in Fig.~\ref{fig:VIV_yaw}
(b) for the two yaw angles evaluated ($\beta=0^\circ$ and $45^\circ$). The DES
predictions show little difference between $\beta=0^\circ$ and $45^\circ$
except around $V_{R,n}=4$ and $6$, where $\bar{A}/D$ is highly sensitive to
changes in $V_{R,n}$. As far as the general variation of $\bar{A}/D$ with
$V_{R,n}$ (identified by solid lines in Fig.~\ref{fig:VIV_yaw} (b)) is
considered, the predictions for both yaw angles show the same behavior,
suggesting that the independence principle also holds for VIV. From the
measurements however, one can only conclude that IP holds primarily in the
Initial Excitation branch. The large difference in the measurements between the
Up and Dn configurations for $\beta=45^\circ$ limits the verification of IP in
VIV to the {\em initial excitation}
branch.
%
\begin{figure}[htb!]
  \subcaptionbox{Exp II}{\incfig[width=0.45\textwidth]{fig/viv_independence_Exp}}
  \qquad
  \subcaptionbox{DES}   {\incfig[width=0.47\textwidth]{fig/viv_independence}} \\
  \caption{Scaled mean displacement amplitude $\bar{A}/D$ for $\beta=0^\circ$
    and $45^\circ$ degrees for a range of $V_{R,n}$ obtained from a) measurements
    from Exp II, and b) DES predictions with the data from Exp III for
    $\beta=0^\circ$ as a guide.}
  \label{fig:VIV_yaw}
\end{figure}

Figure~\ref{fig:VIV_freq} compares the normalized frequency ($f/f_N$) between
the DES predictions and the measurements from Exp III over a range of
$V_{R,n}$. The purple dashed line corresponds to the peak vortex-shedding
frequency for a static cylinder ($St_p=0.21$) and the black dashed line
corresponds to $f=f_N$. The simulations capture the variation of
vortex-shedding frequency over the entire range of $V_{R,n}$ tested. In the
{\em initial excitation} branch, the vortex shedding frequency ($f$), and hence
$f/f_N$ increases linearly with the reduced velocity $V_{R,n}$. Beyond
resonance, which occurs around $f=f_N$, the vortex shedding frequency gets
``locked-in'' with the natural frequency of the system.  This occurs in the
{\em upper} and {\em lower} branches identified in Fig.~\ref{fig:VIV_amp} (a).
The added mass of the fluid causes the lock-in frequency to be higher than
$f_N$, particularly for low-$m^*$ systems. As $V_{R,n}$ increases beyond
lock-in, the vortex shedding frequency desynchronizes with the natural
frequency and $f/f_N$ again follows the purple line in the figure.
Figure~\ref{fig:VIV_freq} also plots the simulation results for
$\beta=45^\circ$, which are nearly coincident with the results for
$\beta=0^\circ$, verifying IP.
%
\begin{figure}[htb!]
  \incfig[width=.5\textwidth]{fig/viv_freq.pdf}
  \caption{Non-dimensional frequency $f/f_N$ for various reduced velocities
    $V_{R,n}$}
  \label{fig:VIV_freq}
\end{figure}
%%%%%%%%%%%%%%%%%%%%%%%%%%%%%%%%%%%%%%%%%%%%%%%%%%

