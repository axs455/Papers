%%%%%%%%%%%%%%%%%%%%%%%%%%%%%%%%%%%%%%%%%%%%%%%%%%%%%%%%%%%%%
\section{Static-Cylinder Results}
\label{sec:comp_results}
%%%%%%%%%%%%%%%%%%%%%%%%%%%%%%%%%%%%%%%%%%%%%%%%%%%%%%%%%%%%%
%
The objective of this paper is to demonstrate the capability of DES to predict
aerodynamic loads on a static (tethered) cylinder and an elastically-mounted
cylinder in normal and yawed flow.  This section discusses the results of the
simulations of flow over a static cylinder.

Simulations are performed for (1) flow normal to the cylinder axis, and (2)
flow at an angle to the cylinder axis (inclined/yawed cylinder); three yaw
angles ($\beta$) are analyzed in this study. Smooth inflow is used -- zero
turbulence in the numerical simulations and the minimum possible inflow
turbulence intensity ($\sim$0.2\%) in the tunnel. Cylinder surface is very
smooth and hence surface roughness is not modeled in the simulations.
Verification is performed with experimental data available in the literature,
as well as data from new experiments conducted as a part of this study.

%%%%%%%%%%%%%%%%%%%%%%%%%%%%%%%%%%%%%%%%%%
\subsection{Normally-Incident Flow}
\label{sec:normally_incident}
%%%%%%%%%%%%%%%%%%%%%%%%%%%%%%%%%%%%%%%%%%
%
Table~\ref{tab:comparisonRe20k} summarizes the simulation results for the
static cases and compares them with two sets of experimental data. Exp-I refers
to the data from~\cite{norberg2013pressure} and Exp-ISU is from our
measurements. Simulations are performed at $Re_D=20,000$, which is the same as
Exp-I, but the $Re_D$ in Exp-ISU is higher ($=51,500$).  The Strouhal number
($St$), the mean drag coefficient, $\overline{C}_d$ and the mean back pressure
coefficient, $\overline{C}_{pb}$ are compared in the table. Strouhal number is
defined as $St = f_v\,D/V_\infty$, where $f_v$ is the vortex-shedding
frequency, $D$ is the cylinder diameter, and $V_\infty$ is the freestream flow
speed.

Figure~\ref{fig:Cp_compared_Re20k} compares the predicted mean aerodynamic
pressure coefficient, $\overline{C}_p$ and the root mean square of perturbation
pressure coefficient, $C_{p'rms} = \sqrt{(\,\overline{C^2_p} -
\overline{C}^2_p\,)}$ with the data from the two experiments. The predicted
$\overline{C}_p$ agrees very well with the data from Exp-I; Exp-ISU data shows
slightly lower $\overline{C}_p$ than observed in Exp-I and the simulation,
after $100^\circ$, and the mean back pressure, $\overline{C}_{pb}$ is lower as
well. The predicted $C_{p'rms}$ distribution lies in between the two
measurements. Both measurements as well as the simulation show the peak of
$C_{p'rms}$ to be around $80^\circ$, which indicates the location of flow
separation. The predicted distribution over the cylinder surface agrees well
with the measurements.
%
\begin{table}[htb!]
  \caption{Summary of results for normally-incident flow simulations} 
  \label{tab:comparisonRe20k} 
  \begin{center}
  \begin{tabular}{c|c|c|c|c}
      $\boldsymbol{Re_D}$ & \textbf{Method} & $\boldsymbol{St}$ & $\boldsymbol{\overline{C}_d}$ & $\boldsymbol{\overline{C}_{pb}}$ \\ \hline
      \hline
      20,000 & $k$-$\omega$ DDES & 0.21 & 1.13 & -1.16 \\ \hline
      20,000 & Exp-I    & 0.19 & 1.22 & -1.1  \\ \hline
      51,500 & Exp-ISU         & 0.21 & 1.14 & -1.3  \\ \hline
      \hline
  \end{tabular}
  \end{center}
\end {table}

\begin{figure}[htb!]
  \centering
  \subcaptionbox{mean pressure coeff., $\overline{C}_p$}
    [.48\linewidth]{\incfig[width=.48\textwidth]{Figures/Cp_Compared.png}}
  \hspace*{\fill}
  \subcaptionbox{r.m.s. of pressure coeff., $C_{p'rms}$ }
    [.48\linewidth]{\incfig[width=.48\textwidth]{Figures/CpRMS_Compared.png}}
  \caption{Comparisons of mean and rms of aerodynamic pressure coefficient
  between the simulation and experimental measurements.}
  \label{fig:Cp_compared_Re20k}
\end{figure}

Figure~\ref{fig:velocity_Normal_Re20k} plots the predicted and measured wake
velocity profiles at the axial station, $x/D=2$; the cylinder axis is located
at $x/D=0$. The peak wake deficit and the wake profile are predicted
accurately. The measured data shows a slight asymmetry in the data, which is
perhaps due to an asymmetry in the experimental setup (the distance from the
tunnel wall between the top and bottom surfaces of the cylinder is slightly
different). The simulation data is averaged over 120 wake shedding cycles and
the experimental data is averaged over 540 cycles.
%
\begin{figure}[htb!]
  \incfig[width=.5\textwidth]{Figures/velocity_Normal_Re20k.png}
  \caption{Comparison of predicted and measured velocity profiles in the
    cylinder wake $2D$ downstream of the cylinder axis. $u$ is the streamwise
    component of velocity and $y$ is normal to the flow direction and the cylinder
    span.}
  \label{fig:velocity_Normal_Re20k}
\end{figure}

Figure~\ref{fig:force_20k} presents predicted temporal variation of sectional
lift and drag coefficients ($C_l$ and $C_d$). As expected for a circular
cylinder, the mean lift coefficient ($\overline{C}_l$) is zero but the mean
drag coefficient ($\overline{C}_d$) is finite. The high-frequency oscillations
which are apparent in $C_l$ time history, are due to K\'arm\'an vortex
shedding. The Strouhal number ($St$) is $\sim 0.2$ as expected for bluff bodies
for the $Re_D$ considered here (see~\cite{travin2000detached}
and~\cite{norberg2013pressure}). In addition to the oscillations at the
vortex-shedding frequency ($f_v$), the entire signal appears to modulate at a
frequency which is an order of magnitude lower than $f_v$. This modulation has
a certain randomness to it and is not perfectly periodic. This modulation
phenomenon has been reported elsewhere, see e.g.,~\cite{travin2000detached}. 
%
\begin{figure}[htb!]
  \incfig[width=0.9\textwidth]{Figures/force_20k.png}
  \caption{Predicted temporal variations of lift and drag coefficients}
  \label{fig:force_20k}
\end{figure}

Figure~\ref{fig:St_Compared_Re20k} (a) compares the power spectral densities of
$C_l$ between data from Exp-ISU and the simulation. Non-dimensional frequency,
$k=fD/V_\infty$ is used to plot the spectra. The lift in the measurements is
obtained by integrating the surface pressure measured using the pressure taps.
Figure~\ref{fig:St_Compared_Re20k} (b) presents the DES computed spectra of
$C_d$. Because vortex shedding alternates between the top and bottom sides of
the cylinder, one vortex shedding period contains two cycles of drag but only
one cycle of lift. This can be seen in Figure~\ref{fig:St_Compared_Re20k},
where the spectral peak for $C_l$ occurs at $f_v$ while the spectral peak for
drag is at $2\,f_v$. Both measurement and prediction agree very well with
each other and show the peak for lift to be around $f_v$ corresponding to
$k=St\sim0.2$.
%
\begin{figure}[htb!]
  \centering
  \subcaptionbox{PSD of $C_l$ }
    [.49\linewidth]{\incfig[width=.49\textwidth]{Figures/St_Compared_Cl_Re20k.pdf}}
  \hspace*{\fill}
  \subcaptionbox{PSD of $C_d$}
    [.49\linewidth]{\incfig[width=.49\textwidth]{Figures/St_Compared_Cd_Re20k.pdf}}
    \caption{Comparison of predicted and measured power spectral densities
      (PSDs) of $C_l$ and $C_d$. The
      measured data here is from our experiments (Exp-ISU).}
\label{fig:St_Compared_Re20k}
\end{figure}

The peak frequency and its first three harmonics that occur at $k = 0.4, 0.6,
\& \,0.8$, are identified in the figure using vertical grid lines and labeled
as $2f_v,\,3f_v,\,\&\,4f_v$. The prediction and experiment both show a second,
smaller peak in the lift spectrum at the third harmonic ($k=0.6$).  Since the
lift vector alternates with the side the vortex sheds from, only odd harmonics
of $f_v$ (i.e., $3f_v,~5f_v,\ldots$) are expected in the spectra. Therefore, no
peak is observed in the lift spectra at the second harmonic ($k=0.4$) or higher
{\em even} harmonics in either the measured or the simulated data. The spectral
shape of the PSD of $C_l$ is correctly predicted, although the predicted
magnitude is slightly higher than the measured data.

%%%%%%%%%%%%%%%%%%%%%%%%%%%%%%%%%%%%%
\subsection{Yawed Flow (Inclined Cylinder)}
\label{sec:inclined_cylinder}
%%%%%%%%%%%%%%%%%%%%%%%%%%%%%%%%%%%%%
%
The schematic in Fig.~\ref{fig:yawedModel} illustrates the setup for the
inclined-cylinder simulations. The relative inclination of the cylinder axis
with respect to the flow is obtained by yawing the flow rather than inclining
the cylinder; these simulations are therefore also referred to as yawed-flow
simulations. Other than yawing the inflow, the setup is exactly the same as for
normally-incident flow.

Yaw angle, $\beta$ is defined as the angle between the inflow velocity vector
$\boldsymbol{V_\infty}$ and the $x$ axis; the cylinder is aligned with the $z$
axis (see Fig.~\ref{fig:yawedModel}. The normal component of the flow velocity
is $V_n=V_\infty \cos\,\beta$ and the spanwise component is $V_z = V_\infty
\,\sin\,\beta$, where $V_\infty=\abs{\boldsymbol{V_\infty}}$. The computational
domain is $L=10\times D$ long in the spanwise direction to investigate spanwise
variation of aerodynamic forces.
%
\begin{figure}[htb!]
  \incfig[width=0.6\textwidth]{Figures/yawedModel.png}
  \caption{A schematic of the computational setup for static inclined cylinder
    simulations. The right figure is a cross-sectional view.  The inflow is set
    to an angle with respect to the cylinder axis, which stays aligned with the $z$
    axis of the coordinate system.}
  \label{fig:yawedModel}
\end{figure}

Table~\ref{tab:comparisonYawedRe20k} summarizes the Strouhal number ($St$) and
the back pressure coefficient ($C_{pb}$) for four different flow yaw angles,
$\beta=0,15,30,\,\&\,45$ degrees. The velocity component normal to the cylinder
axis ($V_n$) is used as the reference velocity scale to define a new set of
non-dimensional quantities, such as Reynolds number, $Re_{D,n} = \rho V_n D /
\mu$, Strouhal number, $St_{n}=f_v\,D /V_n$, and aerodynamic pressure
coefficient, $\overline{C}_{p,n}=2 (\overline{p}-p_\infty)/(\rho V_n^2)$. The
the mean back pressure coefficient normalized in this manner is labeled as
$\overline{C}_{pb,n}$. The measured value of $\overline{C}_{pb,n}$ is lower
than that predicted by the simulations (see
Table~\ref{tab:comparisonYawedRe20k}).
%
%%%%%%%%%%%%%%%%
\begin{table}[htb!]
  \caption{Summary of simulation results for four different flow yaw angles
  ($\beta=0, 15, 30,~\&~45$ deg). Experimental data is only shown for
  $\beta=30^\circ$.} 
  \label{tab:comparisonYawedRe20k} 
  \begin{center}
  \begin{tabular}{c|c|c|c|c|c}
      \textbf{Method} & \textbf{flow angle,} $\boldsymbol{\beta}$ & $\boldsymbol{Re_D}$ & $\boldsymbol{Re_{D,n}}$  & $\boldsymbol{St_{n}}$ & $\boldsymbol{\overline{C}_{pb,n}}$  \\ \hline
      \hline
      Simulation  & $0^\circ$  & 20,000 & 20,000 & 0.21 & -1.15 \\ \hline
      Simulation  & $15^\circ$ & 20,000 & 19,318 & 0.21 & -1.11 \\ \hline
      Simulation  & $30^\circ$ & 20,000 & 17,320 & 0.20 & -1.11  \\ \hline
      \rowcolor[gray]{.9}
      Exp-ISU     & $30^\circ$ & 51,500 & 44,600 & 0.19 & -1.27  \\ \hline
      Simulation  & $45^\circ$ & 20,000 & 14,142 & 0.21 & -1.16  \\ \hline
      \hline
  \end{tabular}
  \end{center}
\end{table}

Figure~\ref{fig:Cp_Compared_Yawed_Exp-ISU} compares with measured data the
predicted mean aerodynamic pressure coefficient ($\overline{C}_{p,n}$) and root
mean square of perturbation pressure coefficient, $C_{p'rms,n}$ for
$\beta=30^\circ$. The predicted back pressure ($\overline{C}_{pb,n}$) is found
to be slightly higher than Exp-ISU data, which is consistent with the
observation for the normally-incident flow cases. The predicted $C_{p'rms,n}$
distribution agrees very well with measurement, especially for
$\theta<120^\circ$, where $\theta$ is the angular position on the cylinder
surface measured from upstream. The peak of $C_{p'rms,n}$ is observed around
$80^\circ$ in both experiment and simulation, which is indicative of the
location of separation of the shear layer. For $\theta > 120^\circ$, the
measured data shows higher $C_{p'rms,n}$ than predicted by the simulations. A
similar underprediction is observed in the normally-incident flow case.
%
\begin{figure}[htb!]
  \centering
  \subcaptionbox{mean pressure coeff.,$\overline{C}_{p,n}$}
    [.48\linewidth]{\incfig[width=.48\textwidth]{Figures/Cp_Compared_Yawed_Exp-ISU.png}}
  \hspace*{\fill}
  \subcaptionbox{rms of perturbation pressure, $C_{p'rms,n}$ }
    [.48\linewidth]{\incfig[width=.48\textwidth]{Figures/CpRMS_Compared_Yawed_Exp-ISU.png}}
  \caption{Comparisons between simulation and experimental measurements for
    $\beta=30^\circ$ yawed-flow case.}
\label{fig:Cp_Compared_Yawed_Exp-ISU}
\end{figure}

%Figure~\ref{fig:velocity_Yawed30_Re20k.png} presents predicted and measured
%wake velocity profiles for $\beta=30^\circ$ case at $x/D=2$. The predicted peak
%wake deficit matches remarkably well with Exp-ISU data. However, the
%experimental data shows higher overshoots in streamwise velocity in the shear
%layer region than predicted by the simulations. The experimental data is also
%very slightly asymmetric, which is likely due to the fact that the cylinder is
%located closer to one side of the tunnel wall. It should be noted that the
%asymmetry is very small and the wall effects are minimal.
%%
%\begin{figure}[htb!]
%  \incfig[width=.5\textwidth]{Figures/velocity_Yawed30_Re20k.png}
%  \caption{Comparison of predicted and measured velocity profiles for
%    $\beta=30^\circ$ yawed flow in the cylinder wake $2D$ downstream of the cylinder axis}
%  \label{fig:velocity_Yawed30_Re20k.png}
%\end{figure}
%
%Figure~\ref{fig:St_Compared_Yawed30_Exp-ISU} compares the power spectral
%densities of the transverse force coefficient (along the $y$ axis), $C_{y,n}$
%for $\beta=30^\circ$ case between Exp-ISU data and predictions, where $C_{y,n}
%= 2\, F_y/ \left( \rho V^2_n (D \times L) \right)$, $F_y$ is the net force over
%the entire cylinder; longitudinal force coefficient, $C_{x,n}$ is similarly
%defined. In the simulation, the first peak is observed around $St_{p,n}\sim
%0.2$, which is the same as for the normally-incidence flow case (see
%Figure~\ref{fig:St_Compared_Re20k} (a)). The spectral shape is correctly
%predicted by the simulation although the measured curve appears to be shifted
%along the $x$ axis; this is likely due to a scaling factor in frequency (log
%scale is used for frequency in Fig.~\ref{fig:St_Compared_Yawed30_Exp-ISU}),
%arising perhaps from a slight mismatch in the measurement of the inflow
%velocity in the experiment.
%%
%\begin{figure}[htb!]
%  \incfig[width=.6\textwidth]{Figures/St_Compared_Yawed30_Exp-ISU.png}
%  \caption{Comparison of predicted and experimental power spectral densities
%    (PSDs) of force coefficient $C_{y,n}$ for $\beta=30^\circ$ yawed-flow
%    cases.}
%  \label{fig:St_Compared_Yawed30_Exp-ISU}
%\end{figure}

Figure~\ref{fig:Compared_Yawed_Re20k} compares the predicted mean aerodynamic
pressure coefficient ($\overline{C}_{p,n}$), for four different values of
inflow yaw angle, $\beta$. The distribution of $\overline{C}_{p,n}$ is found to
be very similar irrespective of $\beta$;~\cite{zdravkovich2003flow} refers to
this as `independence principle' (IP). IP is also observed in the power
spectral densities of the transverse force coefficient, $C_{y,n}$ for the same
set of values of $\beta$ analyzed. $C_{y,n}=2\, f_y/ (\rho V^2_n)$, where $f_y$
is force per unit projected area ($f_y = F_y/(L\,D)$) in the $y$ direction, and $C_{x,n}$
is correspondingly defined for the $x-$component of force.
Figure~\ref{fig:Compared_Yawed_Re20k} (b) shows that the spectra collapse when
$V_n$ is used to normalize the force coefficients and the frequency; the
abscissa in Fig.~\ref{fig:Compared_Yawed_Re20k} (b) is $k_n = fD/V_n$.  
%
\begin{figure}[htb!]
  \subcaptionbox{Mean pressure coeff., $\overline{C}_{p,n}$}[0.48\linewidth]
    {\incfig[width=.48\textwidth]{Figures/Cp_Compared_Yawed_Re20k.png}}
  \hspace*{\fill}
  \subcaptionbox{PSDs of $C_{y,n}$}[0.53\linewidth]
    {\incfig[width=.53\textwidth]{Figures/St_Compared_Yawed_Cl_Re20k.pdf}}
  \caption{Independence principle: comparisons of (a) $\overline{C}_{p,n}$, and
    (b) power spectral densities (PSDs) of $C_{y,n}$ between predictions for various $\beta$ values.}
\label{fig:Compared_Yawed_Re20k}
\end{figure}

Figure~\ref{fig:Spatial_temporal_CxCyRe20k} shows spatio-temporal plots of the
force coefficients $C_{x,n}$ and $C_{y,n}$. The contours show that $C_{x,n}$
and $C_{y,n}$ vary along the span, indicating that vortex shedding does not
occur simultaneously along the entire span. A spatial drift from left to right
with increasing time can be seen in the contours (more visible in the $C_{x,n}$
spatio-temporal plot) which is indicative of spanwise flow over the cylinder.
%
\begin{figure}[htb!]
  \subcaptionbox{$C_{y,n}$ }
    [.48\linewidth]{\incfig[width=.48\textwidth]{Figures/Spatial_temporal_Cy_Re20k_Yawed30.png}}
  \subcaptionbox {$C_{x,n}$}
    [.48\linewidth]{\incfig[width=.48\textwidth]{Figures/Spatial_temporal_Cx_Re20k_Yawed30.png}}
  \hspace*{\fill}
  \caption{Spatio-temporal distribution of force coefficients at
  $\beta=30^\circ$}
  \label{fig:Spatial_temporal_CxCyRe20k}
\end{figure}

Figure~\ref{fig:Coherence_Yawed30_Re20k} presents coherence of force
coefficients for $\beta=30^\circ$ case. Magnitude squared coherence,
$\gamma^2(\Delta z)$ is defined as
%
\begin{align}
  \gamma^2(\Delta z) &= \frac{\langle \abs{S_{xy}}^2 \rangle}{\langle
    S_{xx}\rangle \langle S_{yy} \rangle},
  \label{eq:coherence}
\end{align}
%
where $S_{xy}$ denotes cross-spectral density of the quantity ($C_{x,n}$ or
$C_{y,n}$) at two points separated by a distance $\Delta z$ along the span, and
$S_{xx},~S_{yy}$ are auto-spectral densities; angular brackets denote ensemble
averaging, however ergodicity assumption is used to relate that to time
averaging. The coherence plot of $C_{y,n}$ indicates that spanwise correlation
is very high (over nearly the entire cylinder span) at the vortex shedding
frequency, but is small at other frequencies, which is expected based on
literature. $C_{x,n}$ however is not that highly correlated along the span even
at the peak vortex shedding frequency.
%
\begin{figure}[htb!]
  \subcaptionbox{$C_{y,n}$ }
    [.48\linewidth]{\incfig[width=.48\textwidth]{Figures/Cl_coherence_Yawed30_Re20k.png}}
  \subcaptionbox {$C_{x,n}$}
    [.48\linewidth]{\incfig[width=.48\textwidth]{Figures/Cd_coherence_Yawed30_Re20k.png}}
  \hspace*{\fill}
    \caption{Magnitude squared coherence, $\gamma^2(\Delta z)$ of transverse
      and longitudinal force coefficients - $C_{y,n}$ and $C_{x,n}$ for
      $\beta=30^\circ$ case.}
  \label{fig:Coherence_Yawed30_Re20k}
\end{figure}
%%%%%%%%%%%%%%%%%%%%%%%%%%%%%%%%%%%%%%%%%%%%%%%%%%
