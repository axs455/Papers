%%%%%%%%%%%%%%%%%%%%%%%%%%%%%%%%%%%%%%%%%%%%%%%%%%%%%%%%%%%%%
\section{Vortex-Induced Vibration Results}
\label{sec:VIV}
%%%%%%%%%%%%%%%%%%%%%%%%%%%%%%%%%%%%%%%%%%%%%%%%%%%%%%%%%%%%%
%
A schematic of the computational setup for vortex-induced vibration (VIV)
simulations is presented in Figure~\ref{fig:VIVmodel}. The setup is the same as
for the static simulations except for an additional mass-spring-damper system
that determines the motion of the cylinder. The cylinder is constrained to move
only in the cross-stream ($y$) direction. The $y$-component of the integrated
aerodynamic surface force on the cylinder drives the mass-spring-damper system
given by Eq.~\ref{eq:solidBodyDynamics}. Simulations are performed for eight
inflow reduced velocities and two values of yaw angle, $\beta$ ($=0^\circ$ and
$45^\circ$).
 
In the simulations presented here, the mass ratio $m^*=m/(\rho {\cal V})=2.6$,
where $m$ is the mass of the cylinder, ${\cal V}=\pi (D^2/4) . L$ is the volume
of the cylinder, $L$ is the cylinder span, and $\rho$ is the density of the
fluid flowing over the cylinder. The mechanical damping ratio of the system
$\zeta = c/c_c$ is $0.001$ where, $c_c=2\sqrt{k.m}$ is the coefficient of
critical damping, and the spring stiffness $k_s$ is related to the natural
frequency, $f_N$ by $k_s=m(2\pi\,f_N)^2$.  Reduced velocity, $V_{R,n} =
V_n/(f_N\,D)$, where $f_N$ is the natural frequency of the system and the
subscript $n$ refers to the component of the velocity vector normal to the
cylinder axis to accommodate yawed flow. Two different yaw angle flows are
evaluated, $\beta=0^\circ$ and $45^\circ$, at $Re_{D,n}=20,000$. The
non-dimensional parameters, $m^*$, $\zeta$, and $V_{R,n}$ are selected to match
the measurements of~\citet{franzini2013one}. This measurement dataset is
referred as Exp II in this paper. 

$V_{R,n}$ is varied in the experiments by changing the freestream flow speed,
which changes $Re_D$. In the simulations, $V_{R,n}$ is varied by changing the
spring stiffness, and $Re_D$ is held constant. In addition to Exp II, the
simulation results are also compared to another dataset reported
in~\cite{khalak1997fluid}, which is referred as Exp III here. The values of
$m^*$ ($=2.4$) and $\zeta$ ($=0.0045$) in Exp III are slightly different from
Exp II and the simulations. 
%
\begin{figure}[htb!]
  \incfig[width=.6\textwidth]{Figures/VIV_setup.jpg}
  \caption{A schematic of the computational setup for oscillating cylinder
    simulations. The right figure shows a cross-sectional view. The inflow is set
    to an angle with respect to the cylinder axis, which is aligned with the
    $z$-axis of the coordinate system.}
  \label{fig:VIVmodel}
\end{figure}

Figure~\ref{fig:VIV_amp} compares the predicted scaled mean displacement
amplitude ($\bar{A}/D$) with the measurements of Exp II and III over a wide
range of $V_{R,n}$. \citet{khalak1997fluid} identified the following four
distinct branches in the $\bar{A}/D$ versus $V_{R,n}$ plot of their VIV
measurements for the zero-yaw case: the ``initial excitation'' branch, the
``upper'' branch, the ``lower'' branch, and the ``desynchronization'' branch.
These are labeled and identified with solid black lines as best curve fits of
the measured data in Fig.~\ref{fig:VIV_amp} (a). Note that the variation of
$\bar{A}/D$ with $V_{R,n}$ is topologically different for systems with high
mass and damping, e.g., measurements of \citet{feng1968measurement} show two
branches (multi-valued solution) in the lock-in regime. For the selected low
mass-damping system, $\bar{A}/D$ grows rapidly with $V_{R,n}$ in the {\em
initial excitation} branch, reaches a peak in the {\em upper} branch, then
reduces to 60\% of the peak value in the {\em lower} branch, and finally drops
to a negligible value at higher $V_{R,n}$ in the {\em desynchronization}
branch. The current DES results agree very well with the data (particularly
with Exp III) in the {\em initial excitation} and {\em upper} branches. The
peak amplitude is well captured and occurs around $V_{R,n}=4.76$, which
corresponds to $St\sim0.2$ for a stationary cylinder. The predicted amplitude
is slightly lower than the measurements in the {\em lower} and {\em
desynchronization} branches.  Considering the relatively large differences in
the two sets of measurements (Exp II and Exp III), which provides an estimate
of uncertainity/repeatibility, the prediction accuracy of the simulations is
very good.
%
\begin{figure}[htb!]
  \centering
  \subcaptionbox{$\beta=0^\circ$} {\incfig[width=.47\textwidth]{fig/viv_amp_noyaw.pdf}}
  \qquad
  \subcaptionbox{$\beta=45^\circ$}{\incfig[width=.45\textwidth]{fig/viv_amp_yaw45_wInset.pdf}} \\
    \caption{Comparison of predicted and measured non-dimensional mean
      amplitude, $\bar{A}/D$ over a range of reduced velocities $V_{R,n}$ for
      a) $\beta=0^\circ$, and b) $\beta=45^\circ$. The inset in the plot on the
      right shows the two setups (UP and DN) used in Exp II for yawed-flow
      measurements.} 
  \label{fig:VIV_amp}
\end{figure}

Figure~\ref{fig:VIV_amp} (b) compares the predicted VIV amplitude with the data
from Exp II for $\beta=45^\circ$. The measurements were taken for two
different configurations of the cylinder, which are shown in the inset in the
figure. Since the top and bottom surfaces are not the same, the two
configurations are not identical and the measured data for the two
configurations shows a large difference. Since the setup in the experiment is
asymmetric (wall at the bottom and free surface on the top) and the fact that
end plates were not used, end effects (finite-span effect) might be the reason
for the observed differences between the two configurations. The predictions
agree better with the ``Up'' configuration in the {\em initial excitation} and
{\em upper} branches, and with the ``Dn'' configuration in the {\em lower}
branch. Exp II did not collect any data at higher $V_{R,n}$ to test the
prediction accuracy in the {\em desynchronization} branch.

%%%%%%%%%%%%%%%%%%%%%%%%%%%%%%%%%%%%%%%%%%%%%%%%%%
\subsection{Modes of Vortex Shedding}
\label{sec:VIVmodes}
%%%%%%%%%%%%%%%%%%%%%%%%%%%%%%%%%%%%%%%%%%%%%%%%%%
%
As seen in Figs.~\ref{fig:VIV_amp} and~\ref{fig:VIV_yaw}, the amplitude jumps
up from the {\em initial excitation} branch to the {\em upper} branch, and
jumps down from the {\em upper} branch to the {\em lower} branch as $V_{R,n}$
is increased. These jumps occur as the mode in which the vortex shedding occurs
switches between two possible configurations. These modes are illustrated in
Fig.~\ref{fig:VIV_Q} using schematics and iso-surface of the Q-criterion.  In
the `2S' mode, two single vortices shed alternately from either side of the
cylinder in one cycle of cylinder motion, while in the `2P' mode, two pairs of
vortices are shed from each side of the cylinder in one cycle.  The 2P mode has
been observed in the smoke visualizations of~\citet{brika1993vortex}.  In the
{\em initial excitation} branch, vortex shedding occurs in the 2S mode, while
in the {\em lower} branch, it switches to the 2P mode. In the {\em upper}
branch of the lock-in regime, the vortex shedding switches between the 2S and
2P modes. Other modes of vortex shedding have also been observed (e.g., 2P+S
and P+S modes) in forced vibration motion (see e.g.,
\cite{williamson1988vortex}). 
%
\begin{figure}[htb!]
  \subcaptionbox{2S mode; $V_{R,n}=4$}{\incfig[width=.48\textwidth]{Figures/2S_RV4.png}} \qquad
  \subcaptionbox{2P mode; $V_{R,n}=8$}{\incfig[width=.48\textwidth]{Figures/2P_RV8.png}} \\
%
  \subcaptionbox{$\beta=0^\circ$; $V_{R,n}=4$}{\incfig[width=.48\textwidth]{Figures/Q_normal_RV4_1.png}} \qquad
  \subcaptionbox{$\beta=0^\circ$; $V_{R,n}=8$}{\incfig[width=.48\textwidth]{Figures/Q_Normal_RV8_1.png}} \\
%
  \subcaptionbox{$\beta=45^\circ$; $V_{R,n}=4$}{\incfig[width=.48\textwidth]{Figures/Q_45_RV4_1.png}} \qquad
  \subcaptionbox{$\beta=45^\circ$; $V_{R,n}=8$}{\incfig[width=.48\textwidth]{Figures/Q_45_RV8_1.png}} \\
    \caption{Illustration of the two modes of vortex shedding observed in the
    simulations using schematics in (a) \& (b), and iso-surfaces of the
    Q-criterion for $\beta=0^\circ$ in (c) \& (d) and $\beta=45^\circ$ in (e) \&
    (f).}
  \label{fig:VIV_Q}
\end{figure}
%
%The force coefficients are shown in Fig.~\ref{fig:force_VIV}. It is well known
%that the vibration motion can significantly increase the fluctuation of forces.
%Mean transverse force coefficient $\bar{C}_{x,n}$ for reduce velocities
%$V_{R,N} < 5.9$ have a very similar curve as mean amplitude $\bar{A}/D$ in
%Fig.~\ref{fig:VIV_amp}. However, mean amplitude $\bar{A}/D$ is almost constant
%in the {\em lower} branch while $\bar{C}_{x,n}$ continuously declines. The predicted
%rms of longitudinal force coefficients $C_{y,n,rms}$ has very sharp peak at
%$V_{R,n}=4$, which is slightly different from Exp. II. Similar to $\bar{A}/D$
%at $V_{R,n}=4$, $C_{y,n,rms}$ for $\beta=45^\circ$ is much larger than
%$\beta=0^\circ$. Overall, two yawed angle flow simulations have very similar
%results for small amplitude vibrations. Independent principle can be applied to
%forces coefficients except for the {\em upper} branch ($4 \leqslant V_{R,n}\leqslant
%5.9$).
%%
%\begin{figure}[htb!]
%  \subcaptionbox {$C_{x,n}$}
%    [.48\linewidth]{\incfig[width=.48\textwidth]{Figures/Cd_VIV.png}}
%  \hspace*{\fill}
%  \subcaptionbox{$C_{y,n,rms}$ }
%    [.48\linewidth]{\incfig[width=.48\textwidth]{Figures/Cl_VIV.png}}
%    \caption{Mean transverse and rms of longitudinal force coefficients, $\bar{C}_{x,n}$ and $C_{y,n,rms}$ for
%      $\beta=0^\circ$ and $45^\circ$ cases.}
%  \label{fig:force_VIV}
%\end{figure}
%%%%%%%%%%%%%%%%%%%%%%%%%%%%%%%%%%%%%%%%%%%%%%%%%%
\subsection{Independence Principle in VIV}
\label{sec:IPinVIV}
%%%%%%%%%%%%%%%%%%%%%%%%%%%%%%%%%%%%%%%%%%%%%%%%%%
%
This section investigates the validity of the {\em independence principle} on
frequency and amplitude of displacement of an elastically-mounted cylinder
undergoing VIV. Prior experiments (e.g., ~\citet{jain2013vortex} and
\citet{franzini2013one}) have investigated IP for a circular cylinder.
\citet{zhao2015validity} investigated IP using DNS at Reynolds numbers
($Re_D$=150 and $1,000$). Much higher $Re_D$ ($=20,000$) is considered here and
the DES methodology used here can scale to significantly higher $Re_D$ without
a substantial increase in computation cost.

The data from \citet{franzini2013one} (Exp II) is plotted in
Fig.~\ref{fig:VIV_yaw} (a) and the DES predictions in Fig.~\ref{fig:VIV_yaw}
(b) for the two yaw angles evaluated ($\beta=0^\circ$ and $45^\circ$). The DES
predictions show little difference between $\beta=0^\circ$ and $45^\circ$
except around $V_{R,n}=4$ and $6$, where $\bar{A}/D$ is highly sensitive to
changes in $V_{R,n}$. As far as the general variation of $\bar{A}/D$ with
$V_{R,n}$ (identified by solid lines in Fig.~\ref{fig:VIV_yaw} (b)) is
considered, the predictions for both yaw angles show the same behavior,
suggesting that the independence principle also holds for VIV. From the
measurements however, one can only conclude that IP holds primarily in the
Initial Excitation branch. The large difference in the measurements between the
Up and Dn configurations for $\beta=45^\circ$ limits the verification of IP in
VIV to the {\em initial excitation}
branch.
%
\begin{figure}[htb!]
  \subcaptionbox{Exp II}{\incfig[width=0.45\textwidth]{fig/viv_independence_Exp}}
  \qquad
  \subcaptionbox{DES}   {\incfig[width=0.47\textwidth]{fig/viv_independence}} \\
  \caption{Scaled mean displacement amplitude $\bar{A}/D$ for $\beta=0^\circ$
    and $45^\circ$ degrees for a range of $V_{R,n}$ obtained from a) measurements
    from Exp II, and b) DES predictions with the data from Exp III for
    $\beta=0^\circ$ as a guide.}
  \label{fig:VIV_yaw}
\end{figure}

Figure~\ref{fig:VIV_freq} compares the vortex shedding frequency normalized by
the natural frequency of the system ($f_v/f_N$) between the DES predictions and
the measurements from Exp III over a range of $V_{R,n}$. The purple dashed line
corresponds to the peak vortex-shedding frequency for a static cylinder
($St\sim0.2$) and the black dashed line corresponds to $f_v=f_N$. The
simulations capture the variation of vortex-shedding frequency over the entire
range of $V_{R,n}$ tested. In the {\em initial excitation} branch, $f_v$
increases linearly with the reduced velocity $V_{R,n}$. Beyond resonance, which
occurs around $f_v=f_N$, the vortex shedding frequency gets ``locked-in'' with
the natural frequency of the system.  This occurs in the {\em upper} and {\em
lower} branches identified in Fig.~\ref{fig:VIV_amp} (a).  The added mass of
the fluid causes the lock-in frequency to be higher than $f_N$, particularly
for low-$m^*$ systems. As $V_{R,n}$ increases beyond lock-in, the vortex
shedding frequency desynchronizes with the natural frequency and $f_v/f_N$
again follows the purple line in the figure.  Figure~\ref{fig:VIV_freq} also
plots the simulation results for $\beta=45^\circ$, which are nearly coincident
with the results for $\beta=0^\circ$, verifying IP.
%
\begin{figure}[htb!]
  \incfig[width=.5\textwidth]{fig/viv_freq.pdf}
  \caption{Non-dimensional frequency $f_v/f_N$ for various reduced velocities
    $V_{R,n}$}
  \label{fig:VIV_freq}
\end{figure}
%%%%%%%%%%%%%%%%%%%%%%%%%%%%%%%%%%%%%%%%%%%%%%%%%%
