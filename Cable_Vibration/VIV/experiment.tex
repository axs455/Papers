\section{Experimental Setup and Measurements}
\label{sec:experiments}
%%%%%%%%%%%%%%%%%%%%%%%%%%%%%%%%%%%%%%%%%%%%%%%%%%%%%%%
%
Static wind tunnel experiments were conducted on a smooth cylinder of circular
cross section representing a stay cable to measure the aerodynamic forces and
the velocity distribution in its wake. These experiments were performed in the
Aerodynamic/Atmospheric Boundary Layer (AABL) Wind and Gust Tunnel located in
the Department of Aerospace Engineering at Iowa State University. This wind
tunnel has an aerodynamic test section of $2.44$ m ($8.0$ ft) width $\times$
$1.83$ m ($6.0$ ft) height and a design maximum wind speed of $53$ m/s ($173.9$
ft/s). A polished aluminum tube with diameter, $D=0.127$ m and length, $L=1.52$
m was selected as the smooth cylinder model. Although the aspect ratio
($L/D=12$) is sufficiently large to minimize edge effects at the mid-span of
the circular cylinder, two circular plates of diameter $4\,D$ were attached to
the ends of the cylinder. These plates were adjusted for each cylinder yaw
angle to be parallel to the incoming airflow so that nearly 2D flow could be
achieved over the cylinder.  The blockage ratio in the tunnel with the model
was approximately 5\% for all measurements.  Figure~\ref{fig:ExpSetup} displays
the model setup in the AABL tunnel with the cylinder in normal-flow and
yawed-flow configurations. An innovative multi-functional static setup was
designed to measure the pressure distributions and velocity profiles for
different yaw angles. As shown in Fig.~\ref{fig:ExpSetup}, this setup properly
secures the model for different yaw angles.
%
\begin{figure}[htb!]
  \centering
  \subcaptionbox{setup for normal-flow measurements}
    [.48\linewidth]{\incfig[width=.48\textwidth]{Figures/Experiment_NormalFlow.png}}
  \hspace*{\fill}
  \subcaptionbox{setup for yawed flow measurements}
    [.48\linewidth]{\incfig[width=.48\textwidth]{Figures/Experiment_YawedFlow.png}}
  \caption{Pictures showing the model setup used to allow measurements at
    arbitrary inflow angles. The Cobra probe used to measure the wake is shown in
    (a).}
  \label{fig:ExpSetup}
\end{figure}

The model has 128 pressure taps distributed on its surface to measure local
instantaneous pressure (see Fig.~\ref{fig:experimentTaps}). These pressure
values are used to compute aerodynamic lift and drag (viscous part ignored) on
the cylinder as well as pressure correlations along the span. There are 36
pressure taps at equal angular spacing of 10 degrees along each of the three
rings located on the cylinder. The three rings are labeled Right (R), Middle
(M), and Left (L) as seen in Fig.~\ref{fig:experimentTaps} (a) and are spaced
$4D$ and $5D$ apart from each other along the span. Another set of pressure
taps are located at a fixed angular location at equal spacing of $1D$ along the
span between the rings (see Fig.~\ref{fig:experimentTaps} (a,b)).

%%%%%%%%%%%%%%%%%%%%%%%%%%%%%%%%%%%%%%%%
\subsection{Data Acquisition System}
\label{sec:data_acquisition}
%%%%%%%%%%%%%%%%%%%%%%%%%%%%%%%%%%%%%%%%
%
For wake measurement, one Cobra Probe (4-hole velocity probe) mounted on a
traverse system was used to measure the velocity field behind the model (see
Fig.~\ref{fig:ExpSetup} (a)). In order to minimize the blockage effect of the
traverse system, its cross section was streamlined by using an airfoil section.
For velocity measurements, the sampling rate was $1250$ Hz and the sampling
time was $60$ s. Wake measurements were made $2.5 D$ downstream of the model
(measured from the cylinder axis), where the turbulence intensity was lower
than the maximum allowable value (overall 30\%) for the Cobra Probe.  

Two 64-channel pressure modules (Scanivalve ZOC 33/64 Px) were utilized to
capture the local pressure. In addition, an Ethernet remote A/D system (ERAD)
was used as a collection system to read information from the ZOC. The sampling
rate and sampling time for all pressure measurements were $250$ Hz and $60$ s
respectively. The Scantel program from Scanivalve was used for pressure data
acquisition. In order to minimize the error of measurement due to the tube
length, both ZOCs were placed inside the wind tunnel near the model
(Fig.~\ref{fig:ExpSetup} (b)). The wake measurement traverse system was removed
when surface pressure measurements were made.
%
\begin{figure}[htb!]
  \centering
  \subcaptionbox{Pressure taps on the cylinder model}
    [.48\linewidth]{\incfig[width=.48\textwidth]{Figures/experiment_PressureTap.png}}
  \hspace*{\fill}
  \subcaptionbox{Distribution of pressure taps in a ring}
    [.48\linewidth]{\incfig[width=.48\textwidth]{Figures/experiment_PressureTapCrossSection.png}}
  \caption{Schematics illustrating the locations of surface pressure taps on
    the cylinder model.}
  \label{fig:experimentTaps}
\end{figure}
%%%%%%%%%%%%%%%%%%%%%%%%%%%%%%%%%%%%%%%%%%%%%%%%%%%%%%%%%%%%%
