\documentclass[12pt,preprint]{elsarticle}
\usepackage[margin=1in]{geometry}
\usepackage{subcaption}
\usepackage{amsmath}
\usepackage{makecell}
\usepackage[table]{xcolor}

% Font size for subcaption
\DeclareCaptionFont{mysize}{\fontsize{9}{9.6}\selectfont}
\captionsetup[sub]{font=mysize}
%\usepackage[mathlines,pagewise]{lineno}
%%\modulolinenumbers[2]
\usepackage{bm}

\journal{Journal of Sound \& Vibration}

\newcommand{\hl}[1]{\textcolor{red}{#1}}
\newcommand{\hb}[1]{\textcolor{blue}{#1}}
\newcommand{\incfig}{\centering\includegraphics}
\newcommand{\abs}[1]{\left| #1 \right|} % for absolute value

\begin{document}

\begin{frontmatter}

%% Title, authors and addresses

%% use the tnoteref command within \title for footnotes;
%% use the tnotetext command for theassociated footnote;
%% use the fnref command within \author or \address for footnotes;
%% use the fntext command for theassociated footnote;
%% use the corref command within \author for corresponding author footnotes;
%% use the cortext command for theassociated footnote;
%% use the ead command for the email address,
%% and the form \ead[url] for the home page:
%% \title{Title\tnoteref{label1}}
%% \tnotetext[label1]{}
%% \author{Name\corref{cor1}\fnref{label2}}
%% \ead{email address}
%% \ead[url]{home page}
%% \fntext[label2]{}
%% \cortext[cor1]{}
%% \address{Address\fnref{label3}}
%% \fntext[label3]{}

\title{Finite-Span Effect on Vortex-Induced Vibration Simulations}

%% use optional labels to link authors explicitly to addresses:
%% \author[label1,label2]{}
%% \address[label1]{}
%% \address[label2]{}

\author[1]{Xingeng Wu}
\author[2]{Anupam Sharma\corref{a}}
\ead{sharma@iastate.edu}
\cortext[a]{Corresponding author}
\fntext[1]{Graduate Student}
\fntext[2]{Associate Professor, Iowa State University}

\address{Department of Aerospace Engineering, Iowa State University, Ames,
Iowa, 50011}


\begin{abstract}
  The effects of using periodic boundary conditions when simulating vortex
  induced vibration (VIV) on a finite-span model are explored. When spanwise
  coherence is smaller than the simulated span, the peak VIV amplitude is
  under-predicted due to spreading of energy in frequencies around the Strouhal
  number.
\end{abstract}

\begin{keyword}
  Vortex-Induced Vibration \sep 
  Detached Eddy Simulations \sep 
  spanwise coherence
\end{keyword}

\end{frontmatter}

%\linenumbers

%\begin{linenumbers}
%% main text
%%%%%%%%%%%%%%%%%%%%%%%%
\section{Introduction}
\label{sec:intro}
%%%%%%%%%%%%%%%%%%%%%%%%
%
Vortex-induced vibrations (VIV) are commonly observed in bridge decks, cables,
power conductors, risers in oil rigs, etc. The Karman vortex shedding in the
wake of the cylinder produces periodic forcing on the cylinder. In certain
conditions the vortex-shedding frequency synchronizes (``lock-in'') with the
natural frequency of the system which results in high-amplitude oscillations
(VIV) limited by the system damping. Experimental investigations of VIV are
typically performed with finite-span cylinders with aspect ratios greater than
$10$ to minimize ``end effects'' (*cite*), where end-effect refers to the
three-dimensional flow that results due to the finiteness of the model. End
plates have been used in experiments, which reduce but not completely eliminate
the end effect as horse-shoe vortices develop at the intersection of the cable
and the end plates. 

This problem is avoided in simulations by using periodic boundaries in the span
direction. While the periodic boundaries imply an infinitely-long cylinder, the
finite size of the computational domain in the span direction imposes a
periodicity in the flow which may not exist in reality. If spanwise variations
occur in the problem and the length scale of these variations is larger than
the simulated span, then the flow simulations with spanwise periodicity will
likely yield incorrect results. For stochastic (turbulent) flows, this length
scale can be measured using two-point correlation; in particular, spanwise
coherence is used.

In this paper, we present detached eddy simualtion (DES) results of four models
of a circular cylinder experiencing VIV. The aspect ratio of these models are,
$L/D=1,~2,~5,$ and $10$. Spanwise coherence is analyzed and it is observed that
while the coherence is high around the vortex-shedding frequency ($f_v$),  

when 

the reduced frequency, $k=fD/V_\infty$ matches
the Strouhal number, $f_v D/V_\infty$


Introduction of VIV, review

Aspect ratio of cylinder, history

Motivation for this paper

Figure~\ref{fig:Coherence_staic} presents a lift coherence contour for a
$20\times D$ static cylinder. As the figure shown, the cylinder is not
corelated for the most frequency when it is long enough, exact for $St\sim
0.2$. In here, the peak frequency ($St\sim 0.2$) is the Kármán vortex
shedding frequency. It is well known that the amplitude of rigit cylinder will
significant increase if the natural frequency is close to the vortex shedding
frequency. 

\begin{figure}[htb!]
  \incfig[width=.6\textwidth]{Figures/Cl_Coherence_Z20_Static.png}
  \caption{Lift coherence for static cylinder}
  \label{fig:Coherence_staic}
\end{figure}

\hl{change Z to L/D}


%%%%%%%%%%%%%%%%%%%%%%%%%%%%%%%%%%%%%%%%%%
\section{Computational Methodology}
\label{sec:methodology}
%%%%%%%%%%%%%%%%%%%%%%%%%%%%%%%%%%%%%%%%%%
%
\hl{add A schematic of the setup for the vortex-induced vibration(VIV) simulations}
As Figur~\ref{fig:VIV_setup} shown, an elastically-mounted cylinder with different aspect ratio 
is used in the simulations. The flow is considering as the incompressible flow since similar experiments were carried
out in water runnel(\hl{ref}).  

\begin{figure}[htb!]
  \incfig[width=.6\textwidth]{Figures/VIV_setup}
  \caption{A schematic of the simulated setup for VIV simulations.}
  \label{fig:VIV_setup}
\end{figure}


\hl{numerical method, mesh}

In order to study the aspect ratio of the circular cylinder, four different aspect ratios ($L/D = 1, 2, 5,$ and $10$)
have been chosen and simulated at reduced velocity $V_{R,n}=2, 3, 4, 5, 5.9, 7, 8$. The averaged peak amplitude 
results are shown in 

%%%%%%%%%%%%%%%%%%%%%%%%%%%%%%%%%%%%%%%%%%
\section{Numerical Results and Verification}
\label{sec:results}
%%%%%%%%%%%%%%%%%%%%%%%%%%%%%%%%%%%%%%%%%%
%


Figure~\ref{fig:Amplitude_VIV} compared the predicted non-dimensional mean
amplitude for $10\times D$ cylinder with the experiment
(see e.g., Ref.~\cite{khalak1997fluid}). Overall, the predicted amplitudes agree well with
the experiment. Four branches can be identified on the experiment : initial
excitation, upper branch, lower branch and desynchronization. All four branches
are observed on the simulations. Therefore, the simulated method is able to
simulate vortex induced vibration.

\begin{figure}[htb!]
  \incfig[width=.6\textwidth]{Figures/A_Exp_Z10D_Compared.png}
  \caption{Comparison of predicted and experimental non-dimensional mean
    amplitude $A/D$ for various reduced velocities $V_{R,n}$}
  \label{fig:Amplitude_VIV}
\end{figure}

\begin{figure}[htb!]
  \incfig[width=.6\textwidth]{Figures/A.png}
  \caption{Comparison of predicted non-dimensional mean amplitude $A/D$ for various reduced 
  velocities $V_{R,n}$ for different span lengths}
  \label{fig:Amplitude_Compared}
\end{figure}

Figure~\ref{fig:Coherence_Cl} and \ref{fig:Coherence_Cd} present force
coefficients coherence for different spans cylinders ($Z = 1D, 2D, 5D,$ and
$10D$). For longer cylinders ($Z = 5D$ and $10D$) are only correlated to the
peak vortex shedding frequency and its harmonics. However, the shorter
cylinders ($Z = 1D$ and $2D$) have more wider coherence for more frequencies??

\begin{figure}[htb!]
  \subcaptionbox{$Z=1D$ }
    [.1\linewidth]{\incfig[height=.3\textwidth]{Figures/Coherence_Z1D.png}}
  \hspace*{\fill}
  \subcaptionbox{$Z=2D$ }
    [.1\linewidth]{\incfig[height=.3\textwidth]{Figures/Coherence_Z2D.png}}
  \hspace*{\fill}
  \subcaptionbox{$Z=5D$ }
    [.25\linewidth]{\incfig[height=.3\textwidth]{Figures/Coherence_Z5D.png}}
  \hspace*{\fill}
  \subcaptionbox{$Z=10D$ }
    [.5\linewidth]{\incfig[height=.3\textwidth]{Figures/Coherence_Z10D.png}}
    \caption{Lift coherence for various aspect ratio at $V_{R,n}=5$}
  \label{fig:Coherence_Cl}
\end{figure}

\begin{figure}[htb!]
  \subcaptionbox{$Z=1D$ }
    [.1\linewidth]{\incfig[height=.3\textwidth]{Figures/Coherence_Z1D_Cd.png}}
  \hspace*{\fill}
  \subcaptionbox{$Z=2D$ }
    [.1\linewidth]{\incfig[height=.3\textwidth]{Figures/Coherence_Z2D_Cd.png}}
  \hspace*{\fill}
  \subcaptionbox{$Z=5D$ }
    [.25\linewidth]{\incfig[height=.3\textwidth]{Figures/Coherence_Z5D_Cd.png}}
  \hspace*{\fill}
  \subcaptionbox{$Z=10D$ }
    [.5\linewidth]{\incfig[height=.3\textwidth]{Figures/Coherence_Z10D_Cd.png}}
    \caption{Drag coherence for various aspect ratio at $V_{R,n}=5$}
  \label{fig:Coherence_Cd}
\end{figure}

Figure~\ref{fig:force_VIV} shows ... different peak value, wider, lower psd
smaller $L/D$. 

\begin{figure}[htb!]
  \subcaptionbox {Power spectral densities of $C_l$}
    [.48\linewidth]{\incfig[width=.48\textwidth]{Figures/St_f.png}}
  \hspace*{\fill}
  \subcaptionbox{Zoom-in }
    [.48\linewidth]{\incfig[width=.48\textwidth]{Figures/St_f_2.png}}
    \caption{Comparison of predicted power spectral densities
      (PSDs) of $C_l$ for various frequencies at $V_{R,n}=5$.}
  \label{fig:force_VIV}
\end{figure}



%%%%%%%%%%%%%%%%%%%%%%%%%%%%%%%%%%
\section{Conclusion}
\label{sec:conclusions}
%%%%%%%%%%%%%%%%%%%%%%%%%%%%%%%%%%%%



%% The Appendices part is started with the command \appendix;
%% appendix sections are then done as normal sections
%% \appendix

%% \section{}
%% \label{}

%% If you have bibdatabase file and want bibtex to generate the
%% bibitems, please use
%%
%\end{linenumbers}

\section{References}
\label{sec:references}

\bibliographystyle{elsarticle-harv} 
\bibliography{references}

%% else use the following coding to input the bibitems directly in the
%% TeX file.

%\begin{thebibliography}{00}

%% \bibitem[Author(year)]{label}
%% Text of bibliographic item
%\bibitem[ ()]{}
%\end{thebibliography}

\end{document}
\endinput
