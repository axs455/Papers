%%%%%%%%%%%%%%%%%%%%%%%%%%%%%%%%%%%%%%%%%%
\section{Computational Methodology}
\label{sec:methodology}
%%%%%%%%%%%%%%%%%%%%%%%%%%%%%%%%%%%%%%%%%%
%
The flow is approximated to be incompressible since the flow Mach number is
less than $0.2$. Various degrees of approximations can be utilized to model
flow turbulence: from resolving only time-averaged quantities in Reynolds
Averaged Navier-Stokes or RANS, to resolving the tiniest of turbulent eddies in
Direct Numerical Simulations or DNS. Large eddy simulations (LES) resolve
energy containing eddies but model the net effect of smaller
(unresolved/universal) eddies on larger (resolved) eddies. The detached eddy
simulation (DES) technique~\citep{spalart1997comments} is a hybrid approach
that uses RANS equations to simulate attached flow near solid surfaces and LES
for separated (detached) flow away from them. DES allows computation of high
Reynolds number flows relatively inexpensively by removing the constraint in
LES to have very fine grids near solid boundaries.

Flow over stationary slender structures with circular cross-section has been
studied using unsteady RANS~\citep{pontaza2009three},
DES~\citep{travin2000detached,yeo2011computational,yeo2012aerodynamic},
LES~\citep{breuer1998large,kravchenko2000numerical,catalano2003numerical}, and
DNS~\citep{dong2005dns,zhao2009direct} approaches. Detailed flow simulations
have been performed using DES for a single, stationary, yawed cylinder in
uniform inflow~\citep{yeo2008investigation} and oscillating
inflow~\citep{yeo2012aerodynamic}. 
%DES has also been used to investigate the use of strakes in cables for
%aerodynamic mitigation of wind-induced oscillations
%by~\cite{yeo2011computational}. 
High-fidelity simulations using DNS, LES, and DES have been instrumental in
gaining insights into the problem of flow-induced vibration.

In LES and DES, the equations are spatially filtered (low-pass) and the
numerical procedure solves for the filtered quantities that can be resolved by
the grid. The sub-filter (or sub-grid) quantities exert a ``stress'' on the
filtered flow-field, which is modeled using a sub-grid scale (SGS) stress
model. Denoting spatially filtered quantities by overhead tilde ($^\sim$), the
incompressible flow equations with an eddy-viscosity turbulence model are
%
\begin{align}
  \frac{\partial{\tilde{U_i}}}{\partial{x_i}}&=0,~{\rm and} \nonumber \\
   \frac{\partial{\tilde{U_i}}}{\partial{t}}+
   \frac{\partial{(\tilde{U_j}\tilde{U_i})}}{\partial{x_j}}
   &=-\frac{1}{\rho}\frac{\partial{\tilde{p}}}{\partial{x_i}}+\nu\frac{\partial{^2\tilde{U_i}}}{\partial{x_j^2}}-\frac{\partial{\tau_{ij}}}{\partial{x_j}},
 \label{eq:geqs}
\end{align}
%
where $\tau_{ij} = \widetilde{U_i U_j}-\tilde{U_i}\tilde{U_j} = -2 \, \nu_{SGS}
\, \tilde{S}$ and $\tilde{S} = (\partial{\tilde{U_i}}/\partial{x_j} +
\partial{\tilde{U_j}}/\partial{x_i})/2$. In the above, SGS denotes a sub-grid
scale quantity, $\tau_{ij}^{SGS}$ denotes the sub-grid scale stress tensor
which is modeled as the product of the eddy viscosity, $\nu_{SGS}$ and the
strain rate tensor $S_{ij}$; turbulence models of such type are referred to as
eddy-viscosity models. DES is a non-zonal hybrid RANS-LES method, where a RANS
turbulence model is used to compute the eddy viscosity for the SGS stress
tensor in the corresponding LES. In the original DES formulation
(\cite{spalart1997comments}), the Spalart-Allmaras (SA) LES and SA-RANS models
were used. We use the method developed by~\cite{yin2015dynamic}, which
introduces a dynamic procedure to improve the DES capability by correcting for
modeled stress depletion and log-layer mismatch. This improved model is
referred to as delayed detached eddy simulation or DDES. The model has been
implemented in the open source CFD software OpenFOAM. All the simulations
results in this paper are obtained using OpenFOAM. The numerical scheme uses
second order backward difference for time integration and linear interpolation
with central differencing for spatial discretization of the governing equations.

%%%%%%%%%%%%%%%%%%%%%%%%%%%%%%%%%%%%%
\subsection{Delayed Detached Eddy Simulation (DDES) Model}
\label{sec:DDES}
%%%%%%%%%%%%%%%%%%%%%%%%%%%%%%%%%%%%%
%
A summary of the DDES model used in this study is provided here; details are
available in~\cite{yin2015dynamic}. It uses a $k$-$\omega$ turbulence closure
model in the RANS zones, and the same model is used to calculate $\nu_T$ in the
LES zones. The eddy viscosity in the $k$-$\omega$ DDES can be defined as
$\nu_T=l_{DDES}^2\, \omega$, where $l_{DDES}$ is the DDES length scale. The
different length scales in the $k$-$\omega$ DDES model are defined as
%
\begin{align}  
  l_{DDES} &=l_{RANS}-f_d\, \max( 0,~l_{RANS}-l_{LES}), \nonumber \\
  l_{RANS} &=\sqrt{k}/\omega, \\
  l_{LES} &=C_{DES}\bigtriangleup. \nonumber
  \label{eq:ddes_lscale}
\end{align}
%
In the above, $l_{RANS}$ and $l_{LES}$ are the length scales of the RANS and LES
branches respectively and $\bigtriangleup = f_d \,V^{1/3}+(1-f_d ) \,h_{max}$,
where $h_{max} = \max(dx,dy,dz)$ is the maximum grid size, and $f_d$ is a
shielding function of the DDES model, defined as $f_d = 1 -
\tanh\{(8\,r_d)^3\}$, with
\[
  r_d=\frac{k/\omega+\nu}{\kappa^2 \, d_w^2 \, \sqrt{U_{i,j} U_{i,j} }},
\]
$\nu$ is the molecular viscosity, $\kappa$ is the von K\'arm\'an constant, $d_w$ is
the distance between the cell and the nearest wall, and $U_{i,j}=\partial_j
U_i$ is the velocity gradient.  In the RANS branch, the transport equation for
k and $\omega$ are written as
%
\begin{align}  
  \frac{Dk}{Dt} & = 2\nu_T |S|^2-C_\mu k\omega+\partial{_j [(\nu+\sigma_k \nu_T ) \partial{_j k}]}, \nonumber \\
  \frac{D\omega}{Dt} & = 2C_{\omega1} |S|^2-C_{\omega2} \omega^2 
                    +\partial{_j [(\nu+\sigma_\omega \nu_T )\partial{_j \omega}]},\\
    &~~~~~~~\rm{where}~\nu_T=k^2/\omega. \nonumber
    \label{eq:transportEqs}
\end{align}  
%
In the LES region ($f_d=1,l_{DDES}=C_{DES} \, \bigtriangleup$), the eddy
viscosity switches to
$\nu_T=l_{DDES}^2\,\omega=(C_{DES}\bigtriangleup)^2\omega$, which is similar to
the eddy viscosity in the Smagorinsky model, $\nu_s=(C_s \bigtriangleup)^2
|S|$.

The LES branch of this $k$-$\omega$ DDES model uses a dynamic procedure which
defines the value of $C_{DES}$ as
%
\begin{align}
  C_{DES}   & =  \max( C_{lm},~C_{dyn} ), \nonumber \\
  C_{dyn}^2 & = \max \left( 0,\; 0.5 \frac{L_{i,j} M_{i,j}}{M_{i,j} M_{i,j}}\right), \nonumber \\
  C_{lim}   &=C_{DES}^0 \left[1-\tanh \left(\alpha \exp \left(\frac{-\beta \, h_{max}}{L_k}\right)\right)\right],\\
  C_{DES}^0 &=0.12, \quad   L_k=\left(\frac{\nu^3}{\epsilon}\right)^{\frac{1}{4}},  \quad  \alpha=25,  \quad  \beta=0.05, \nonumber \\
  \epsilon  &= 2 \left(C_{DES}^0 h_{max} \right)^2 \omega\,|S|^2+C_\mu k\,\omega. \nonumber
\end{align}  
%
For further details about the DES model, the reader is referred
to~\cite{yin2015dynamic}.

%%%%%%%%%%%%%%%%%%%%%%%%%%%%%%%%%%%%%%%%%%
\subsection{Solid Body Dynamics and Coupling}
\label{sec:coupling}
%%%%%%%%%%%%%%%%%%%%%%%%%%%%%%%%%%%%%%%%%%
%
Simulations of VIV are performed with the pimpleDyMFoam solver which uses the
{\em sixDoFRigidBodyMotion} feature of OpenFOAM. Single degree of freedom (dof)
motion of the cylinder is considered here with displacement permitted only in
the $y-$direction (orthogonal to the freestream flow direction). At each time
step, pimpleDyMFoam calculates the motion and updates the displacement of the
cylinder by integrating the following equation in time using the Crank-Nicolson
method.
%
\begin{equation}
  m\,\ddot{y} + c\,\dot{y} + k_s \,y = F_{fluid},
  \label{eq:solidBodyDynamics}
\end{equation}
%
where $m$ is the mass of the rigid cylinder, $\ddot{y}$, $\dot{y}$, and $y$ are the
instantaneous acceleration, velocity and displacement of the cylinder,
respectively, $c$ is the damping of the system, $k_s$ is the spring stiffness and
$F_{fluid}$ is the transverse (cross-stream) component of the fluid force
acting on the cylinder surface.

In order to maintain the quality of the meshes inside the boundary layer in
dynamic simulations, the mesh around the cylinder (up to 1 diameter from the
cylinder axis) moves with the cylinder without deforming. The mesh away from
the cylinder is deformed based on the position of the cylinder. The solver then
updates the fluxes according to the motion and uses the PIMPLE method to solve
the incompressible Navier-Stokes equations.

%%%%%%%%%%%%%%%%%%%%%%%%%%%%%%%%%%%%%%%%%%
\subsection{Computational Grid and Mesh Sensitivity Study}
\label{sec:grids}
%%%%%%%%%%%%%%%%%%%%%%%%%%%%%%%%%%%%%%%%%%
%
The outer boundary of the computational domain is circular with a radius of
$25\times D$, where $D$ is the diameter of the cylinder. The cylinder is placed
in the center of the domain and the span dimension is $L=10\times D$ for all
the simulations presented here. Periodic boundary conditions are used in the
span direction, while freestream condition is used on the outer radial
boundary. The domain is discretized using a multi-block grid that has three
blocks: (1) an O-grid is used to resolve the flow around the cylinder, (2) an
H-grid to resolve the wake, and (3) a C-grid for the far field. In order to
accurately capture the aerodynamic forces on the cylinder, the flow around the
cylinder and in the near-wake region has to be resolved with high precision. A
fine mesh is therefore applied in these regions. Figure~\ref{fig:Mesh} shows a
cross-sectional view of the full computational domain as well as a zoom-view to
highlight the grid topology.
%
\begin{figure}[htb!]
  \centering
  \subcaptionbox{Overall CFD domain}%
    [.48\linewidth]{\incfig[width=.48\textwidth]{Figures/Mesh1.png}}
  \hspace*{\fill}
  \subcaptionbox{Zoom view of near-cylinder mesh}%
    [.48\linewidth]{\incfig[width=.48\textwidth]{Figures/Mesh2.png}}
  \caption{Cross-sectional views of the computational grid}
  \label{fig:Mesh}
\end{figure}

A mesh sensitivity study is performed for normally-incident flow.
Table~\ref{tab:meshSize} provides a summary of the three meshes used for this
study. These are labeled `Mesh 1', `Mesh 2', and `Mesh 3'.  Cell counts,
Strouhal number ($St$), mean drag coefficient ($\overline{C}_d$), and mean back
pressure coefficient ($\overline{C}_{p,b}$) are also listed in the table.
Figure~\ref{fig:Cp_compare_LS_Mesh} compares the predicted mean aerodynamic
pressure coefficient, $\overline{C}_p=2 (\overline{p}-p_\infty)/(\rho
V_\infty^2)$ and the r.m.s. of pressure coeff., $C_{p'rms}$ for the different
grids. Mesh 2 and Mesh 3 show consistent distributions of $\overline{C}_p$ and
$C_{p'rms}$ and predict the same location for boundary layer separation
($\theta \sim 80^\circ$); separation location is delayed with Mesh 1.  Based on
these results, Mesh 3 is chosen for the subsequent simulations.
%
\begin{table}[htb!]
  \caption{Summary of the test cases simulated to investigate sensitivity of
    results to mesh size. $N_\theta$, $N_r$, and $N_z$ denote number of cells in 
    circumferential, radial, and span directions respectively. Also tabulated 
    are $St$, $\overline{C}_d$, and $\overline{C}_{p,b}$.}
  \label{tab:meshSize} 
  \begin{center}
    \begin{tabular}{c|c|c|c|c|c}
      \textbf{Mesh name} &  \textbf{Cell counts} ${(N_{\theta} \times N_r \times N_z)}$ & \textbf{Total cell count} & 
      \boldsymbol{$St$} & \boldsymbol{$\overline{C_d}$} & \boldsymbol{$\overline{C_{p,b}}$} \\ \hline
      \hline
      Mesh 1 & $157 \times 233 \times 65$  & 2.37 M & 0.201 & 1.065 & -0.983 \\ \hline
      Mesh 2 & $188 \times 275 \times 80$  & 4.14 M & 0.208 & 1.125 & -1.132 \\ \hline
      Mesh 3 & $236 \times 343 \times 100$ & 8.09 M & 0.208 & 1.137 & -1.148 \\
      \hline \hline
    \end{tabular}
  \end{center}
\end{table}

%
\begin{figure}[htb!]
  \centering
  \subcaptionbox{mean pressure coeff., $\overline{C}_p$}
    [.48\linewidth]{\incfig[width=.48\textwidth]{Figures/Cp_Mesh_Re20k_no_exp.pdf}}
  \hspace*{\fill}
  \subcaptionbox{r.m.s. of pressure coeff., $C_{p'rms}$ }
    [.48\linewidth]{\incfig[width=.48\textwidth]{Figures/CpRMS_Mesh_Re20_k_no_exp.pdf}}
  \caption{Results of mesh refinement study}
  \label{fig:Cp_compare_LS_Mesh}
\end{figure}
%%%%%%%%%%%%%%%%%%%%%%%%%%%%%%%%%%%%%%%%%%%%%%%%%%%%%%%
