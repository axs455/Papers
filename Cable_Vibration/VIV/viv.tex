%%%%%%%%%%%%%%%%%%%%%%%%%%%%%%%%%%%%%%%%%%%%%%%%%%
\subsection{Vortex-Induced Vibration (VIV)}
\label{sec:VIV}
%%%%%%%%%%%%%%%%%%%%%%%%%%%%%%%%%%%%%%%%%%%%%%%%%%
%
A schematic of the computational setup for the vortex-induced vibration (VIV)
simulations is presented in Figure~\ref{fig:VIVmodel}. The setup is the same as
for the static simulations except for an additional forced mass-spring-damper
system. The cylinder is allowed to move only in the $y$ (cross-stream)
direction. The $y$ component of the integrated aerodynamic surface force on the
cylinder (denoted by $F_y$) drives the mass-spring-damper system given by
%
\begin{equation}
  m \frac{{\rm d}^2 x}{{\rm d}t^2} + c \frac{{\rm d} x}{{\rm d}t} + k x = F_y(t).
  \label{eq:mass-spring-damper}
\end{equation}
 
In the simulations the mass ratio $m^*=m/(\rho {\cal V})=2.6$, where $m$ is the
mass of the cylinder, ${\cal V}=\pi (D^2/4) . S$ is the volume of the cylinder,
$S$ is the cylinder span, and $\rho$ is the density of the fluid flowing over
the cylinder. The mechanical damping ratio of the system $\zeta = c/c_c$ is
$0.001$ where, $c_c=2\sqrt{k.m}$ is the critical damping, and the spring
stiffness $k$ is obtained from the natural frequency, $f_N$ as
$k=m(2\pi\,f_N)^2$.  The values of these parameters are selected to match the
measurements presented in~\cite{franzini2013one}.  This measurement dataset is
referred as Exp II in this paper. The predictions are also compared to another
dataset reported in~\cite{khalak1997fluid}, which is referred as Exp III here.
The mass ratio and damping ratio used in Exp III are slightly different
($m^*=2.4$ and $\zeta=0.0045$) from Exp II and the simulations. It should be
noted that the measurement results have end effects due to the finite length of
the cylinder. {\color{red} The experiments were conducted in water channels
\ldots check aspect ratio and differences between Exp II and III} The
simulations use periodic boundary conditions in the span direction, which
theoretically simulates an infinite span. However, span-periodicity can induce
artificial effects if the spanwise coherence is greater than the simulated
span.
%
\begin{figure}[htb!]
  \incfig[width=.6\textwidth]{Figures/VIV_setup.jpg}
  \caption{A schematic of the computational setup for oscillating cylinder
    simulations. The right figure is a cross-sectional view.  The inflow is set
    to an angle with respect to the cylinder axis, which stays aligned with the $z$
    axis of the coordinate system.}
  \label{fig:VIVmodel}
\end{figure}

Figure~\ref{fig:Amplitude_VIV} compares the predicted non-dimensional
mean amplitude $\bar{A}/D$ with the measurements of Exp II and III over a wide
range of reduced velocity $V_{R,n}$, which is defined as
$V_{R,n}=V_n/(f_N\,D)$, where $f_N$ is the natural frequency of the system. The
subscript $n$ refers to the component of the vector normal to the cylinder axis
to accommmodate for yawed flow. Two different yaw angle flows are evaluated --
$\beta=0^\circ$ and $45^\circ$ -- both at $Re_{D,n}=20,000$. Reduced velocity
is the inverse of Strouhal number, $V_{R,n} = 1/St$.

\citet{khalak1997fluid} identified the following four distinct branches in
their VIV measurements for the zero-yaw case: the ``initial excitation''
branch, the ``upper'' branch, the ``lower'' branch, and the
``desynchronization'' branch. These are labeled and identified with solid black
lines as best curve fits of the measured data in Fig.~\ref{fig:Amplitude_VIV}
(a). In the initial excitation branch, the mean amplitude grows rapidly with
$V_{R,n}$. The scaled displacement oscillation amplitude ($A/D$) reaches a peak
in the upper branch, drops to 60\% of the peak value in the lower branch, and
then finally drops to a negligible value at higher $V_{R,n}$ in the
desynchronization branch. The current DES simulations agree very well with the
data (particularly with Exp III) in the initial excitation and upper branches.
The peak amplitude is well captured and occurs around $V_{R,n}=5$, which
corresponds to the peak shedding Strouhal number, $St=0.2$ for a stationary
cylinder. The predicted amplitude is slightly lower than the measurements in
the lower and desynchronization branches. Considering the relatively large
differences in the two sets of measurements (between Exp II and Exp III), the
prediction accuracy of the simulations is very good.
%
\begin{figure}[htb!]
  \centering
  \subcaptionbox{$\beta=0^\circ$}%
    [.48\linewidth]{\incfig[width=.48\textwidth]{fig/viv_amp_noyaw.pdf}}
  \hspace*{\fill}
  \subcaptionbox{$\beta=45^\circ$}%
    [.45\linewidth]{\incfig[width=.45\textwidth]{fig/viv_amp_yaw45_wInset.pdf}}
    \caption{Comparison of predicted and experimental non-dimensional mean
      amplitude $A/D$ over a range of reduced velocities $V_{R,n}$ for a)
      $\beta=0^\circ$, and b) $\beta=45^\circ$. The inset in the plot on the right
      shows the two setups (UP and DN) used in Exp II for yawed-flow
      measurements.} 
  \label{fig:Amplitude_VIV}
\end{figure}

Figure~\ref{fig:Amplitude_VIV} (b) compares the predicted VIV amplitude with
the measurements from Exp II. The measurements were taken for two different
configurations of the cylinder, shown in the inset in the figure. Since the top
and bottom surfaces are not the same, the two configurations are not identical
and the measured data for the two configurations shows a large difference.
Based on the fact that end plates were not used in the simulations, end effects
(finite-span effect) might be the reason for the differences between the two
configurations.  The predictions agree better with the ``Up'' configuration in
the initial excitation and upper branches, and with the ``Dn'' configuration in
the lower branch. Exp II did not collect any data at higher $V_{R,n}$ to test
the prediction accuracy in the desynchronization branch.

Figure~\ref{fig:VIV_yaw} investigates the effect of yaw angle on the amplitude
of VIV. The data from Exp II is plotted in Fig.~\ref{fig:VIV_yaw} (a) and the
DES predictions in the plot on the right.  The DES predictions show little
difference between $\beta=0^\circ$ and $45^\circ$ except around $V_{R,n}=4$ and
$6$, where $A/D$ is highly sensitive to changes in $V_{R,n}$. As far as the
general variation of $A/D$ with $V_{R,n}$ (as shown in the plot with lines) is
considered, the DES predictions for both yaw angles show the same behavior,
suggesting that the independence principle also holds for VIV. From the
measurements however, one can only conclude that yaw-independence holds
primarily in the Initial Excitation branch. The large difference in the data
between the Up and Dn configurations for $\beta=45^\circ$ limits the
verification of yaw-independence in VIV predicted by DES to the initial
excitation branch.
%
\begin{figure}[htb!]
  \centering
  \begin{subfigure}[b]{.5\linewidth}
    \incfig[width=0.45\textwidth]{fig/viv_independence_Exp}
    \caption{Exp II}
  \end{subfigure}
  \begin{subfigure}[b]{.5\linewidth}
    \incfig[width=0.45\textwidth]{fig/viv_independence}
    \caption{DES}
  \end{subfigure}
  \caption{Scaled displacement amplitude ($A/D$) for $\beta=0$ and $45$ degrees
    for a range of $V_{R,n}$ from a) measurements from Exp II, and b) DES
    predictions with the data from Exp III for $\beta=0^\circ$ as a guide.}
  \label{fig:VIV_yaw}
\end{figure}

Figure~\ref{fig:f_VIV} presents non-dimensional frequency $f/f_N$ for various
reduced velocities $V_{R,n}$, with $f$ being the vortex shedding frequency in
the simulation while being the oscillating frequency in Exp. III because vortex
shedding frequency data is not available in Exp. III. 


The blue dash line
indicates vortex shedding frequency for the static cylinder ($St_p=0.21$),
while the red dash line presents the natural frequency.  As the figure shown,
the predicted non-dimensional frequencies for both yawed flows are very similar
and agree very well with Exp.~III.  In the initial excitation branch,
non-dimensional frequency $f/f_N$ increases as reduced velocity $V_{R,n}$
increases, which follows the blue dash line. It suggests that the oscillating
motion is independent of the natural frequency of the system in the initial
excitation branch. After that, the vortex shedding frequency locks onto the
natural frequency $f_N$ in the upper branch and lower branch, which is known as
"lock-in". Because this is a low mass-damping system, the ``lock-in'' frequency
is higher than the natural frequency, while it is significantly smaller than
the vortex shedding frequency of the static cylinder. The similar phenomenon
can also be observed on Exp.~III.  In the desynchronization branch, the vortex
shedding frequency desynchronizes from the natural frequency indicating the
ending of ``lock-in''.
%
\begin{figure}[htb!]
  \incfig[width=.5\textwidth]{fig/viv_freq.pdf}
  \caption{Non-dimensional frequency $f/f_N$ for various reduced velocities
    $V_{R,n}$}
  \label{fig:f_VIV}
\end{figure}

As Fig.~\ref{fig:Amplitude_VIV} shown, there is a jump between the upper branch
and the initial excitation branch and another jump between the upper branch and
the lower branch. This is because there are two different vortex shedding
patterns happened in this reduced velocity region. Figure~\ref{fig:Q_VIV}
displaces the Q-criterion for two reduced velocities $V_{R,n}=4$ and
$V_{R,n}=8$ for $\beta=45^\circ$ case, which represent the typical vortex
shedding modes of the initial excitation branch and the lower branch,
respectively. As the figure shown, when $V_{R,n}=4$, one vortex shedding period
includes two singular vortices (2S mode) shed alternately from either side of
the cylinder. On the contrary, when $V_{R,n}=8$ and ``lock-in'' happens,
periodic vortex shedding pattern switches to two pairs of vortices (2P mode).
Vortex shedding frequency lock onto the natural frequency as a result of 2P
mode. The vortex shedding mode of the upper branch would shift between these
two modes.
%
\begin{figure}[htb!]
  \subcaptionbox {Q-criterion for $V_{R,n}=4$}
    [.48\linewidth]{\incfig[width=.48\textwidth]{Figures/Q_45_RV4_1.png}}
  \hspace*{\fill}
  \subcaptionbox{Q-criterion for $V_{R,n}=8$ }
    [.48\linewidth]{\incfig[width=.48\textwidth]{Figures/Q_45_RV8_1.png}}
  \vskip\baselineskip
   \subcaptionbox {A schematic of 2S mode for $V_{R,n}=4$}
    [.48\linewidth]{\incfig[width=.48\textwidth]{Figures/2S_RV4.png}}
  \hspace*{\fill}
  \subcaptionbox{A schematic of 2P mode $V_{R,n}=8$ }
    [.48\linewidth]{\incfig[width=.48\textwidth]{Figures/2P_RV8.png}}
    \caption{Q-criterion and vortex shedding modes for $\beta=45^\circ$ yawed flow simulation} 
  \label{fig:Q_VIV}
\end{figure}

The force coefficients are shown in Fig.~\ref{fig:force_VIV}. It is well known
that the vibration motion can significantly increase the fluctuation of forces.
Mean transverse force coefficient $\bar{C}_{x,n}$ for reduce velocities
$V_{R,N} < 5.9$ have a very similar curve as mean amplitude $\bar{A}/D$ in
Fig.~\ref{fig:Amplitude_VIV}. However, mean amplitude $\bar{A}/D$ is almost
constant in the lower branch while $\bar{C}_{x,n}$ continuously declines. The
predicted rms of longitudinal force coefficients $C_{y,n,rms}$ has very sharp
peak at $V_{R,n}=4$, which is slightly different from Exp. II. Similar to
$\bar{A}/D$ at $V_{R,n}=4$, $C_{y,n,rms}$ for $\beta=45^\circ$ is much larger
than $\beta=0^\circ$. Overall, two yawed angle flow simulations have very
similar results for small amplitude vibrations. Independent principle can be
applied to forces coefficients except for the upper branch ($4 \leqslant
V_{R,n}\leqslant 5.9$).
%
\begin{figure}[htb!]
  \subcaptionbox {$C_{x,n}$}
    [.48\linewidth]{\incfig[width=.48\textwidth]{Figures/Cd_VIV.png}}
  \hspace*{\fill}
  \subcaptionbox{$C_{y,n,rms}$ }
    [.48\linewidth]{\incfig[width=.48\textwidth]{Figures/Cl_VIV.png}}
    \caption{Mean transverse and rms of longitudinal force coefficients, $\bar{C}_{x,n}$ and $C_{y,n,rms}$ for
      $\beta=0^\circ$ and $45^\circ$ cases.}
  \label{fig:force_VIV}
\end{figure}
