%%%%%%%%%%%%%%%%%%%%%%%%%%%%%%%%%%%%%%%%%%
\section{Computational Methodology}
\label{sec:methodology}
%%%%%%%%%%%%%%%%%%%%%%%%%%%%%%%%%%%%%%%%%%
%
The flow is approximated to be incompressible since the flow Mach number is
less than $0.2$. Various degrees of approximations can be utilized to model
flow turbulence: from resolving only time-averaged quantities in Reynolds
Averaged Navier-Stokes or RANS, to resolving the tiniest of turbulent eddies in
Direct Numerical Simulations or DNS. Large eddy simulations (LES) resolve
energy containing eddies but model the net effect of smaller
(unresolved/universal) eddies on larger (resolved) eddies. The detached eddy
simulation (DES) technique~\citep{spalart1997comments} is a hybrid approach
that uses RANS equations to simulate attached flow near solid surfaces and LES
for separated (detached) flow away from the surfaces. DES allows computation of
high Reynolds number flows relatively inexpensively by removing the constraint
in LES to have very fine grids near solid boundaries.

Flow over slender structures with circular cross-section has been studied using
unsteady RANS~\citep{pontaza2009three},
DES~\citep{travin2000detached,yeo2012aerodynamic},
LES~\citep{breuer1998large,kravchenko2000numerical,catalano2003numerical}, and
DNS~\citep{dong2005dns,zhao2009direct} approaches. Latest numerical efforts in
simulating aerodynamics of cable vibration have utilized
DES~\citep{yeo2007characteristics,yeo2008investigation,yeo2012aerodynamic,yeo2011computational}
as the primary numerical approach. Detailed flow simulations have been
performed with a single, stationary, yawed cylinder in uniform
inflow~\citep{yeo2007characteristics,yeo2008investigation} and oscillating
inflow~\citep{yeo2012aerodynamic}. It has been concluded
by~\cite{yeo2012aerodynamic} that oblique wind-induced aerodynamic forces play
an important role in initiating and increasing the vibration at low
frequencies. DES has also been used to investigate the use of strakes in cables
for aerodynamic mitigation of wind-induced oscillations
by~\cite{yeo2011computational}. In essence, high-fidelity simulations have been
instrumental in gaining insights into the problem of flow-induced cylinder
vibration.

In LES and DES, the equations are spatially filtered (low-pass) and the
numerical procedure solves for the filtered quantities that can be resolved by
the grid. The sub-filter (or sub-grid) quantities exert a ``stress'' on the
filtered flow-field, which is modeled using a sub-grid scale (SGS) stress
model. Denoting spatially filtered quantities by overhead tilde ($^\sim$), the
incompressible flow equations with an eddy-viscosity turbulence model are
%
\begin{gather*}
  \frac{\partial{\tilde{U_i}}}{\partial{x_i}}=0,~{\rm and} \\
   \frac{\partial{\tilde{U_i}}}{\partial{t}}+
   \frac{\partial{(\tilde{U_j}\tilde{U_i})}}{\partial{x_j}}
   =-\frac{1}{\rho}\frac{\partial{\tilde{p}}}{\partial{x_i}}+\nu\frac{\partial{^2\tilde{U_i}}}{\partial{x_j^2}}-\frac{\partial{\tau_{ij}}}{\partial{x_j}}, \\
 \label{eq:geqs}
\end{gather*}
%
where $\tau_{ij} = \widetilde{U_i U_j}-\tilde{U_i}\tilde{U_j} = -2 \, \nu_{SGS}
\, \tilde{S}$ and $\tilde{S} = (\partial{\tilde{U_i}}/\partial{x_j} +
\partial{\tilde{U_j}}/\partial{x_i})/2$.  In the above, SGS denotes a sub-grid
scale quantity, $\tau_{ij}^{SGS}$ denotes the sub-grid scale stress tensor
which is modeled as the product of eddy viscosity, $\nu_{SGS}$ and the strain
rate tensor $S_{ij}$; turbulence models of such type are referred to as
eddy-viscosity models. DES is a non-zonal hybrid RANS-LES method, where a RANS
turbulence model is used to compute the eddy viscosity for the SGS stress
tensor in the corresponding LES. In the original DES formulation
(\cite{spalart1997comments}), the Spalart-Allmaras (SA) LES and SA-RANS models
were used. We use the method developed by~\cite{yin2015dynamic}, which
introduces a dynamic procedure to improve the DES capability by correcting for
modeled stress depletion and log-layer mismatch. This model has been
implemented in the open source CFD software OpenFOAM. All the simulations in
this paper are obtained using OpenFOAM. The numerical scheme uses second order
backward difference for time integration and linear interpolation with central
differencing for spatial discretization of the governing equations.

Simulations of VIV are using pimpleDyMFoam solver with sixDoFRigidBodyMotion
feature on OpenFOAM. The incompressible Navier-Stokes equations are solved by
pimpleDyMFoam solver. For VIV cases, the incompressible Navier-Stokes equations
includes an addition body force term due to interaction between the moving
cylinder and fluid. 

\[ m\ddot{Y}+c\dot{Y}+kY=F_{fluid}, \] where $m$ is the mass of the rigid body,
$\ddot{Y}$, $\dot{Y}$, and $Y$ are the instantaneous acceleration, velocity and
displacement of the cylinder, respectively, $c$ is spring damping, $k$ is the
spring stiffness and $F_{fluid}$ is the fluid forces applying on the cylinder
solved by Navier-Stokes equations. 


%%%%%%%%%%%%%%%%%%%%%%%%%%%%%%%%%%%%%
\subsection{Detached Eddy Simulation Model}
\label{sec:DDES}
%%%%%%%%%%%%%%%%%%%%%%%%%%%%%%%%%%%%%
%
A summary of the DES model used in this study is provided here; details are
available in~\cite{yin2015dynamic}. It uses a $k-\omega$ turbulence closure
model in the RANS zones, and the same model is used to calculate $\nu_T$ in the
LES zones. The eddy viscosity in the $k-\omega$ DDES can be defined as
$\nu_T=l_{DDES}^2\, \omega$, where $l_{DDES}$ is the DDES length scale. The
different length scales in the $k-\omega$ DDES model are defined as
%
\begin{align*}  
  l_{DDES} &=l_{RANS}-f_d\, \max( 0,~l_{RANS}-l_{LES}), \\
  l_{RANS} &=\sqrt{k}/\omega,\\
  l_{LES} &=C_{DES}\bigtriangleup.
\end{align*}
%
In the above, $l_{RANS}$ and $l_{LES}$ are the length scales of the RANS and LES
branches respectively and $\bigtriangleup = f_d \,V^{1/3}+(1-f_d ) \,h_{max}$,
where $h_{max} = \max(dx,dy,dz)$ is the maximum grid size, and $f_d$ is a
shielding function of the DDES model, defined as $f_d = 1 -
\tanh\{(8\,r_d)^3\}$, with
\[
  r_d=\frac{k/\omega+\nu}{\kappa^2 \, d_w^2 \, \sqrt{U_{i,j} U_{i,j} }},
\]
$\nu$ is the molecular viscosity, $\kappa$ is the von Karman constant, $d_w$ is
the distance between the cell and the nearest wall, and $U_{i,j}=\partial_j
U_i$ is the velocity gradient.  In the RANS branch, the transport equation for
k and $\omega$ are written as
%
\begin{align*}  
  \frac{Dk}{Dt} & = 2\nu_T |S|^2-C_\mu k\omega+\partial{_j [(\nu+\sigma_k \nu_T ) \partial{_j k}]},\\
  \frac{D\omega}{Dt} & = 2C_{\omega1} |S|^2-C_{\omega2} \omega^2 
                    +\partial{_j [(\nu+\sigma_\omega \nu_T )\partial{_j \omega}]},\\
    &~~~~~~~\rm{where}~\nu_T=k^2/\omega.   
\end{align*}  
%
In the LES region ($f_d=1,l_{DDES}=C_{DES} \, \bigtriangleup$), the eddy viscosity
switches to $\nu_T=l_{DDES}^2\,\omega=(C_{DES}\bigtriangleup)^2\omega$, which is
similar to the eddy viscosity in the Smagorinsky model, $\nu_s=(C_s
\bigtriangleup)^2 |S|$.

The LES branch of this $k-\omega$ DDES model uses a dynamic procedure which
defines the value of $C_{DES}$ as
%
\begin{align*}  
  C_{DES}   & =  \max( C_{lm},~C_{dyn} ), \\
  C_{dyn}^2 & = \max \left( 0,\; 0.5 \frac{L_{i,j} M_{i,j}}{M_{i,j} M_{i,j}}\right),\\
  C_{lim}   &=C_{DES}^0 \left[1-\tanh \left(\alpha \exp \left(\frac{-\beta \, h_{max}}{L_k}\right)\right)\right],\\
  C_{DES}^0 &=0.12, \quad   L_k=\left(\frac{\nu^3}{\epsilon}\right)^{\frac{1}{4}},  \quad  \alpha=25,  \quad  \beta=0.05,\\
  \epsilon  &= 2 \left(C_{DES}^0 h_{max} \right)^2 \omega\,|S|^2+C_\mu k\,\omega.
\end{align*}  
%
For further details about the DES model, the reader is referred to~\cite{yin2015dynamic}.


%%%%%%%%%%%%%%%%%%%%%%%%%%%%%%%%%%%%%%%%%%
\subsection{Computational Grids}
\label{sec:grids}
%%%%%%%%%%%%%%%%%%%%%%%%%%%%%%%%%%%%%%%%%%
%
The outer boundary of the computational domain is circular with a radius of
$25\times D$, where $D$ is the diameter of the cylinder. The cylinder is placed
in the center of the domain and the span dimension is $10\times D$ for all
simulations. Periodic boundary conditions are used in the span direction, while
freestream condition is used on the outer radial boundary. The domain is
discretized using a multi-block grid that has three blocks: (1) an O-grid is
used to resolve the flow around the cylinder, (2) an H-grid to resolve the
wake, and (3) a C-grid for the far field. In order to accurately capture the
aerodynamic forces on the cylinder, the flow around the cylinder and in the
near-wake region has to be resolved with high precision. A fine mesh is
therefore applied in these regions.  Figure~\ref{fig:Mesh} shows a
cross-sectional view of the full computational domain as well as a zoom-view to
highlight the grid topology. Results of a mesh sensitivity study are presented
in Section~\ref{sec:mesh_sensitivity}.
%
\begin{figure}[htb!]
  \centering
  \subcaptionbox{Overall CFD domain}%
    [.48\linewidth]{\incfig[width=.48\textwidth]{Figures/Mesh1.png}}
  \hspace*{\fill}
  \subcaptionbox{Zoom view of near-cylinder mesh}%
    [.48\linewidth]{\incfig[width=.48\textwidth]{Figures/Mesh2.png}}
  \caption{Cross-sectional views of the computational grid}
  \label{fig:Mesh}
\end{figure}

%%%%%%%%%%%%%%%%%%%%%%%%%%%%%%%%%%%%%%%%%%%%%%%%%%%%%%%
