\documentclass[12pt,authoryear]{elsarticle}
\usepackage{graphicx}
\usepackage{subcaption}
\usepackage{makecell}
\usepackage[table]{xcolor}
\usepackage[margin=1in]{geometry}
\usepackage{amsmath}
\usepackage{amssymb}
\usepackage{bm}
\linespread{1.5} %line spacing

\newcommand{\incfig}{\centering\includegraphics}
\newcommand{\abs}[1]{\left| #1 \right|} % for absolute value
% Font size for subcaption
\DeclareCaptionFont{mysize}{\fontsize{9}{9.6}\selectfont}
\captionsetup[sub]{font=mysize}
% For editing
\newcommand{\hl}[1]{\textcolor{red}{#1}}
\newcommand{\hb}[1]{\textcolor{blue}{#1}}

%\captionsetup{compatibility=false}
%\journal{Journal of Fluids and Structures}


\begin{document}

\begin{frontmatter}

\title{Numerical Study of Static and Vortex-Induced Vibrations of Normal and
Yawed Cylinder with Detached Eddy Simulations}

%% use optional labels to link authors explicitly to addresses:
%% \author[label1,label2]{}
%% \address[label1]{}
%% \address[label2]{}

\author[1]{Xingeng Wu}
\author[1]{Mohammad Jafari} 
\author[2]{Partha Sarkar}
\author[3]{Anupam Sharma\corref{a}}
\ead{sharma@iastate.edu}
\cortext[a]{Corresponding author}
\fntext[1]{Graduate Student}
\fntext[2]{Professor, Iowa State University}
\fntext[3]{Associate Professor, Iowa State University}

\address{Department of Aerospace Engineering, Iowa State University, Ames, Iowa, 50011}

\begin{abstract}
A computational approach based on a $k-\omega$ delayed detached eddy simulation
model for predicting aerodynamic loads on a smooth circular cylinder is
verified against experiments. Comparisons with experiments are performed for
flow over a stationary cylinder and for a cylinder oscillating in the
transverse direction due to flow-induced forces (vortex-induced vibration or
VIV). For the static cases, measurement data from the literature is used to
validate the predictions for normally-incident flow. New experiments are
conducted as a part of this study for yawed flow, where the cylinder axis is
inclined with respect to the inflow velocity at the desired yaw angle,
$\beta=30^\circ$. Good agreement is observed between the predictions and
measurements for mean and rms surface pressure and wake velocity deficit.
However the lift spectrum prediction shows a slight offset in frequency from
the measurements. Three yawed flow cases ($\beta=15^\circ, 30^\circ, \,\&\,
45^\circ$) are simulated and the results are found to be independent of $\beta$
(dynamic scaling) when the flow speed normal to the cylinder axis is selected
as the reference velocity scale.  

Dynamic (VIV) simulations are performed by coupling the flow solver with a
solid dynamics solver where the cylinder motion in the transverse direction is
modeled as a spring-mass-damper system. The simulations accurately predict the
displacement amplitude and unsteady loading over a wide range of reduced
velocity, including the region where lock-in occurs. VIV simulations are
performed at two yaw angles, $\beta=0^\circ$ and $45^\circ$ and the
appropriately-scaled results are found to be similar over the range of reduced
velocities tested with slightly higher discrepancy around the reduced velocity
corresponding to the natural frequency of the system.
\end{abstract}

\begin{keyword}
  Vortex-Induced Vibrations \sep Detached Eddy Simulations \sep Yawed Cylinder \sep Independence Principle
\end{keyword}
\end{frontmatter}

%\modulolinenumbers[2] %\linenumbers

%\begin{linenumbers}
%% main text
%%%%%%%%%%%%%%%%%%%%%%%%
\section{Introduction}
\label{sec:intro}
%%%%%%%%%%%%%%%%%%%%%%%%
%
Inclined and horizontally- or vertically-spanned cylinder are used in various
engineering applications: cable-stayed, suspension, and tied-arch bridges,
power transmission lines, offshore risers, sub-sea pipelines etc. These
cylinders are prone to large-amplitude flow-induced vibration, which can
lead to catastrophic failure of the cylinders and the structures supported by
them. The vibration mechanisms involve complex aeroelastic (motion-induced)
interactions that depend on the spatial orientation, geometry,
surface-characteristics, and dynamic properties of cylinders.~\cite{davenport1995dynamics} 
shows that large amplitude vibrations can lead to
either catastrophic- or fatigue failure of the cylinders and/or the adjoining
structures, which poses a significant threat to the safety and serviceability of
these systems.

Vortex-induced vibration(VIV) is an oscillating motion due to the interaction
between structures and periodical vortices. VIV is a very common vibration
phenomenon happened in many engineering areas, such as offshore structures,
transmission lines, stacks, and bridges. Due to its practical significance in
many engineering structures, VIV has raised many interests and been extensively
investigated in past decades.
\cite{bearman1984vortex},~\cite{sarpkaya2004critical} and
~\cite{williamson2004vortex} have done some detailed reviews of the these work.
One of the most important observations of VIV is called "lock-in" or
"synchronization". "Lock-in" is a regime where shedding vortices induce large
response amplitude of the structures because the oscillating frequency along
with vortex shedding frequency lock onto the natural frequency of the
structures. ~\cite{feng1968measurement} observed two amplitude branches in the
``lock-in'' regime in experiment. Because this experiment was carried in wind
tunnels, the system has a very large mass ratio. The jump between two branches
was suspected due to the change of vortex shedding mode.    However, in recent
years, other different amplitude branches have been found in water tunnel tests
which have much small mass ratio. Four branches have been identified and
labeled as ``initial excitation'', ``upper branch'', ``lower branch'' and
``desynchronization''.(\cite{khalak1997fluid}). 

Past investigations of VIV were based primarily on experiments and a limited
number of computational fluid dynamics (CFD). These studies, spanning over the
past several decades, have resulted in an improved understanding of the
phenomenon of the VIV, particularly when the flow is perpendicular to the
cylinder in wind tunnel.  However, most of these work investigated relatively
large mass and damping system.  In recent years, due to increasing applications
of marine engineering, VIV of low mass and damping system has raised many
interests and been widely studied in experiments. Though a lot of experimental
studies have been done in this decade, only a very few studies were related to
the VIV of a yawed cylinder with low mass and damping.  These experiments
(~\cite{jain2013vortex}, ~\cite{franzini2013one}) were performed to validate
the independence principle on the oscillating frequency, amplitude and force
coefficients of an elastically-mounted cylinder with various yawed angles. The
experiments are conducted in water tunnels with a partially submerged yawed
cylinder. Due to the effects of the end conditions, these experiments observed
different results between the cylinder slanted towards upstream and downstream.
Ideally, the independence principle should be investigated without any end
effect because the vortex shedding of a yawed stationary cylinder and the
response of an elastically-mounted cylinder are both very sensitive to the end
conditions. Compared with experiments, simulations are much easier to achieve
the end-effect-free result.  ~\cite{zhao2015validity} studied the independence
principle of an elastically-mounted cylinder for two Reynolds numbers (Re=150
and Re=1,000) by using direct numerical simulation (DNS) and periodic boundary
conditions.  Independence principle for the response amplitude and frequency is
observed on this numerical study, but the Reynolds numbers of this study are
very small.
	
This paper presents the results of our progress towards the goal of developing
an aeroelastic load model for cylinder vibration. The objective is to be able
to accurately predict aerodynamic loading on a single, smooth circular cylinder
operating in smooth inflow (normal and yawed flow) for static and dynamic
conditions. This paper focuses on verifying the computational methodology
against experimental data for normally-incident flow and yawed flow.


%%%%%%%%%%%%%%%%%%%%%%%%%%%%%%%%%%%%%%%%%%
\section{Computational Methodology}
\label{sec:methodology}
%%%%%%%%%%%%%%%%%%%%%%%%%%%%%%%%%%%%%%%%%%
%
The flow is approximated to be incompressible since the flow Mach number is
less than $0.2$. Various degrees of approximations can be utilized to model
flow turbulence: from resolving only time-averaged quantities in Reynolds
Averaged Navier-Stokes or RANS, to resolving the tiniest of turbulent eddies in
Direct Numerical Simulations or DNS. Large eddy simulations (LES) resolve
energy containing eddies but model the net effect of smaller
(unresolved/universal) eddies on larger (resolved) eddies. The detached eddy
simulation (DES) technique~\citep{spalart1997comments} is a hybrid approach
that uses RANS equations to simulate attached flow near solid surfaces and LES
for separated (detached) flow away from the surfaces. DES allows computation of
high Reynolds number flows relatively inexpensively by removing the constraint
in LES to have very fine grids near solid boundaries.

Flow over slender structures with circular cross-section has been studied using
unsteady RANS~\citep{pontaza2009three},
DES~\citep{travin2000detached,yeo2012aerodynamic},
LES~\citep{breuer1998large,kravchenko2000numerical,catalano2003numerical}, and
DNS~\citep{dong2005dns,zhao2009direct} approaches. Latest numerical efforts in
simulating aerodynamics of cable vibration have utilized
DES~\citep{yeo2007characteristics,yeo2008investigation,yeo2012aerodynamic,yeo2011computational}
as the primary numerical approach. Detailed flow simulations have been
performed with a single, stationary, yawed cylinder in uniform
inflow~\citep{yeo2007characteristics,yeo2008investigation} and oscillating
inflow~\citep{yeo2012aerodynamic}. It has been concluded
by~\cite{yeo2012aerodynamic} that oblique wind-induced aerodynamic forces play
an important role in initiating and increasing the vibration at low
frequencies. DES has also been used to investigate the use of strakes in cables
for aerodynamic mitigation of wind-induced oscillations
by~\cite{yeo2011computational}. In essence, high-fidelity simulations have been
instrumental in gaining insights into the problem of flow-induced cylinder
vibration.

In LES and DES, the equations are spatially filtered (low-pass) and the
numerical procedure solves for the filtered quantities that can be resolved by
the grid. The sub-filter (or sub-grid) quantities exert a ``stress'' on the
filtered flow-field, which is modeled using a sub-grid scale (SGS) stress
model. Denoting spatially filtered quantities by overhead tilde ($^\sim$), the
incompressible flow equations with an eddy-viscosity turbulence model are
%
\begin{gather*}
  \frac{\partial{\tilde{U_i}}}{\partial{x_i}}=0,~{\rm and} \\
   \frac{\partial{\tilde{U_i}}}{\partial{t}}+
   \frac{\partial{(\tilde{U_j}\tilde{U_i})}}{\partial{x_j}}
   =-\frac{1}{\rho}\frac{\partial{\tilde{p}}}{\partial{x_i}}+\nu\frac{\partial{^2\tilde{U_i}}}{\partial{x_j^2}}-\frac{\partial{\tau_{ij}}}{\partial{x_j}}, \\
 \label{eq:geqs}
\end{gather*}
%
where $\tau_{ij} = \widetilde{U_i U_j}-\tilde{U_i}\tilde{U_j} = -2 \, \nu_{SGS}
\, \tilde{S}$ and $\tilde{S} = (\partial{\tilde{U_i}}/\partial{x_j} +
\partial{\tilde{U_j}}/\partial{x_i})/2$.  In the above, SGS denotes a sub-grid
scale quantity, $\tau_{ij}^{SGS}$ denotes the sub-grid scale stress tensor
which is modeled as the product of eddy viscosity, $\nu_{SGS}$ and the strain
rate tensor $S_{ij}$; turbulence models of such type are referred to as
eddy-viscosity models. DES is a non-zonal hybrid RANS-LES method, where a RANS
turbulence model is used to compute the eddy viscosity for the SGS stress
tensor in the corresponding LES. In the original DES formulation
(\cite{spalart1997comments}), the Spalart-Allmaras (SA) LES and SA-RANS models
were used. We use the method developed by~\cite{yin2015dynamic}, which
introduces a dynamic procedure to improve the DES capability by correcting for
modeled stress depletion and log-layer mismatch. This model has been
implemented in the open source CFD software OpenFOAM. All the simulations in
this paper are obtained using OpenFOAM. The numerical scheme uses second order
backward difference for time integration and linear interpolation with central
differencing for spatial discretization of the governing equations.

Simulations of VIV are using pimpleDyMFoam solver with sixDoFRigidBodyMotion
feature on OpenFOAM. The incompressible Navier-Stokes equations are solved by
pimpleDyMFoam solver. For VIV cases, the incompressible Navier-Stokes equations
includes an addition body force term due to interaction between the moving
cylinder and fluid. 

\[
  m\ddot{Y}+c\dot{Y}+kY=F_{fluid},
\]
where $m$ is the mass of the rigid body, $\ddot{Y}$, $\dot{Y}$, and $Y$ are the
instantaneous acceleration, velocity and displacement of the cylinder,
respectively, $c$ is spring damping, $k$ is the spring stiffness and
$F_{fluid}$ is the fluid forces applying on the cylinder solved by
Navier-Stokes equations. 

%%%%%%%%%%%%%%%%%%%%%%%%%%%%%%%%%%%%%
\subsection{Detached Eddy Simulation Model}
\label{sec:DDES}
%%%%%%%%%%%%%%%%%%%%%%%%%%%%%%%%%%%%%
%
A summary of the DES model used in this study is provided here; details are
available in~\cite{yin2015dynamic}. It uses a $k-\omega$ turbulence closure
model in the RANS zones, and the same model is used to calculate $\nu_T$ in the
LES zones. The eddy viscosity in the $k-\omega$ DDES can be defined as
$\nu_T=l_{DDES}^2\, \omega$, where $l_{DDES}$ is the DDES length scale. The
different length scales in the $k-\omega$ DDES model are defined as
%
\begin{align*}  
  l_{DDES} &=l_{RANS}-f_d\, \max( 0,~l_{RANS}-l_{LES}), \\
  l_{RANS} &=\sqrt{k}/\omega,\\
  l_{LES} &=C_{DES}\bigtriangleup.
\end{align*}
%
In the above, $l_{RANS}$ and $l_{LES}$ are the length scales of the RANS and LES
branches respectively and $\bigtriangleup = f_d \,V^{1/3}+(1-f_d ) \,h_{max}$,
where $h_{max} = \max(dx,dy,dz)$ is the maximum grid size, and $f_d$ is a
shielding function of the DDES model, defined as $f_d = 1 -
\tanh\{(8\,r_d)^3\}$, with
\[
  r_d=\frac{k/\omega+\nu}{\kappa^2 \, d_w^2 \, \sqrt{U_{i,j} U_{i,j} }},
\]
$\nu$ is the molecular viscosity, $\kappa$ is the von Karman constant, $d_w$ is
the distance between the cell and the nearest wall, and $U_{i,j}=\partial_j
U_i$ is the velocity gradient.  In the RANS branch, the transport equation for
k and $\omega$ are written as
%
\begin{align*}  
  \frac{Dk}{Dt} & = 2\nu_T |S|^2-C_\mu k\omega+\partial{_j [(\nu+\sigma_k \nu_T ) \partial{_j k}]},\\
  \frac{D\omega}{Dt} & = 2C_{\omega1} |S|^2-C_{\omega2} \omega^2 
                    +\partial{_j [(\nu+\sigma_\omega \nu_T )\partial{_j \omega}]},\\
    &~~~~~~~\rm{where}~\nu_T=k^2/\omega.   
\end{align*}  
%
In the LES region ($f_d=1,l_{DDES}=C_{DES} \, \bigtriangleup$), the eddy viscosity
switches to $\nu_T=l_{DDES}^2\,\omega=(C_{DES}\bigtriangleup)^2\omega$, which is
similar to the eddy viscosity in the Smagorinsky model, $\nu_s=(C_s
\bigtriangleup)^2 |S|$.

The LES branch of this $k-\omega$ DDES model uses a dynamic procedure which
defines the value of $C_{DES}$ as
%
\begin{align*}  
  C_{DES}   & =  \max( C_{lm},~C_{dyn} ), \\
  C_{dyn}^2 & = \max \left( 0,\; 0.5 \frac{L_{i,j} M_{i,j}}{M_{i,j} M_{i,j}}\right),\\
  C_{lim}   &=C_{DES}^0 \left[1-\tanh \left(\alpha \exp \left(\frac{-\beta \, h_{max}}{L_k}\right)\right)\right],\\
  C_{DES}^0 &=0.12, \quad   L_k=\left(\frac{\nu^3}{\epsilon}\right)^{\frac{1}{4}},  \quad  \alpha=25,  \quad  \beta=0.05,\\
  \epsilon  &= 2 \left(C_{DES}^0 h_{max} \right)^2 \omega\,|S|^2+C_\mu k\,\omega.
\end{align*}  
%
For further details about the DES model, the reader is referred to~\cite{yin2015dynamic}.



%%%%%%%%%%%%%%%%%%%%%%%%%%%%%%%%%%%%%%%%%%
\subsection{Computational Grids}
\label{sec:grids}
%%%%%%%%%%%%%%%%%%%%%%%%%%%%%%%%%%%%%%%%%%
%
The outer boundary of the computational domain is circular with a radius of
$25\times D$, where $D$ is the diameter of the cylinder. The cylinder is placed
in the center of the domain and the span dimension is $10\times D$ for
all simulations. Periodic boundary
conditions are used in the span direction, while freestream condition is used
on the outer radial boundary. The domain is discretized using a multi-block
grid that has three blocks: (1) an O-grid is used to resolve the flow around
the cylinder, (2) an H-grid to resolve the wake, and (3) a C-grid for the far
field. In order to accurately capture the aerodynamic forces on the cylinder,
the flow around the cylinder and in the near-wake region has to be resolved
with high precision. A fine mesh is therefore applied in these regions.
Figure~\ref{fig:Mesh} shows a cross-sectional view of the full computational
domain as well as a zoom-view to highlight the grid topology. Results of a mesh
sensitivity study are presented in Section~\ref{sec:mesh_sensitivity}.
%
\begin{figure}[htb!]
  \centering
  \subcaptionbox{Overall CFD domain}%
    [.48\linewidth]{\incfig[width=.48\textwidth]{Figures/Mesh1.png}}
  \hspace*{\fill}
  \subcaptionbox{Zoom view of near-cylinder mesh}%
    [.48\linewidth]{\incfig[width=.48\textwidth]{Figures/Mesh2.png}}
  \caption{Cross-sectional views of the computational grid}
  \label{fig:Mesh}
\end{figure}

%%%%%%%%%%%%%%%%%%%%%%%%%%%%%%%%%%%%%%%%%%%%%%%%%%%%%%%
\section{Experimental Setup and Measurements}
\label{sec:experiments}
%%%%%%%%%%%%%%%%%%%%%%%%%%%%%%%%%%%%%%%%%%%%%%%%%%%%%%%
%
Static wind tunnel experiments were conducted on a smooth cylinder of circular
cross section representing a stay cable to measure the aerodynamic forces and
the velocity distribution in its wake. These experiments were performed in the
Aerodynamic/Atmospheric Boundary Layer (AABL) Wind and Gust Tunnel located in
the Department of Aerospace Engineering at Iowa State University. This wind
tunnel has an aerodynamic test section of $2.44$ m ($8.0$ ft) width $\times$
$1.83$ m ($6.0$ ft) height and a design maximum wind speed of $53$ m/s ($173.9$
ft/s). A polished aluminum tube with diameter, $D=0.127$ m and length, $L=1.52$
m was selected as the smooth cylinder model. Although the aspect ratio
($L/D=12$) is sufficiently large to minimize edge effects at the mid-span of
the circular cylinder, two circular plates of diameter $4\,D$ were attached to
the ends of the cylinder. These plates were adjusted for each cylinder yaw
angle to be parallel to the incoming airflow so that nearly 2D flow could be
achieved over the cylinder.  The blockage ratio in the tunnel with the model
was approximately 5\% for all measurements.  Figure~\ref{fig:ExpSetup} displays
the model setup in the AABL tunnel with the cylinder in normal-flow and
yawed-flow configurations. An innovative multi-functional static setup was
designed to measure the pressure distributions and velocity profiles for
different yaw angles. As shown in Fig.~\ref{fig:ExpSetup}, this setup properly
secures the model for different yaw angles.
%
\begin{figure}[htb!]
  \centering
  \subcaptionbox{setup for normal-flow measurements}
    [.48\linewidth]{\incfig[width=.48\textwidth]{Figures/Experiment_NormalFlow.png}}
  \hspace*{\fill}
  \subcaptionbox{setup for yawed flow measurements}
    [.48\linewidth]{\incfig[width=.48\textwidth]{Figures/Experiment_YawedFlow.png}}
  \caption{Pictures showing the model setup used to allow measurements at
    arbitrary inflow angles. The Cobra probe used to measure the wake is shown in
    (a).}
  \label{fig:ExpSetup}
\end{figure}

The model has 128 pressure taps distributed on its surface to measure local
instantaneous pressure (see Fig.~\ref{fig:experimentTaps}). These pressure
values are used to compute aerodynamic lift and drag (viscous part ignored) on
the cylinder as well as pressure correlations along the span. There are 36
pressure taps at equal angular spacing of 10 degrees along each of the three
rings located on the cylinder. The three rings are labeled Right (R), Middle
(M), and Left (L) as seen in Fig.~\ref{fig:experimentTaps} (a) and are spaced
$4D$ and $5D$ apart from each other along the span. Another set of pressure
taps are located at a fixed angular location at equal spacing of $1D$ along the
span between the rings (see Fig.~\ref{fig:experimentTaps} (a,b)).

%%%%%%%%%%%%%%%%%%%%%%%%%%%%%%%%%%%%%%%%
\subsection{Data Acquisition System}
\label{sec:data_acquisition}
%%%%%%%%%%%%%%%%%%%%%%%%%%%%%%%%%%%%%%%%

For wake measurement, one Cobra Probe (4-hole velocity probe) mounted on a
traverse system was used to measure the velocity field behind the model (see
Fig.~\ref{fig:ExpSetup} (a)). In order to minimize the blockage effect of the
traverse system, its cross section was streamlined by using an airfoil section.
For velocity measurements, the sampling rate was $1250$ Hz and the sampling
time was $60$ s. Wake measurements were made $2.5 D$ downstream of the model
(measured from the cylinder axis), where the turbulence intensity was lower
than the maximum allowable value (overall 30\%) for the Cobra Probe.  

Two 64-channel pressure modules (Scanivalve ZOC 33/64 Px) were utilized to
capture the local pressure. In addition, an Ethernet remote A/D system (ERAD)
was used as a collection system to read information from the ZOC. The sampling
rate and sampling time for all pressure measurements were $250$ Hz and $60$ s
respectively. The Scantel program from Scanivalve was used for pressure data
acquisition. In order to minimize the error of measurement due to the tube
length, both ZOCs were placed inside the wind tunnel near the model
(Fig.~\ref{fig:ExpSetup} (b)). The wake measurement traverse system was removed
when surface pressure measurements were made.
%
\begin{figure}[htb!]
  \centering
  \subcaptionbox{Pressure taps on the cylinder model}
    [.48\linewidth]{\incfig[width=.48\textwidth]{Figures/experiment_PressureTap.png}}
  \hspace*{\fill}
  \subcaptionbox{Distribution of pressure taps in a ring}
    [.48\linewidth]{\incfig[width=.48\textwidth]{Figures/experiment_PressureTapCrossSection.png}}
  \caption{Schematics illustrating the locations of surface pressure taps on
    the cylinder model.}
  \label{fig:experimentTaps}
\end{figure}

%%%%%%%%%%%%%%%%%%%%%%%%%%%%%%%%%%%%%%%%%%%%%%%%%%%%%%%%%%%%%
\section{Numerical Results and Verification with Measured Data}
\label{sec:comp_results}
%%%%%%%%%%%%%%%%%%%%%%%%%%%%%%%%%%%%%%%%%%%%%%%%%%%%%%%%%%%%%

The objective of this paper is to demonstrate the capability of detached eddy
simulations to predict aerodynamic loads on a static cylinder and an elastically-mounted cylinder.
Verification with existing experimental data in the literature, and data from
new experiments conducted as a part of this study, are presented for both the static
cylinder and the elastically-mounted cylinder. Smooth inflow is used -- zero
turbulence in the numerical simulations and the minimum possible inflow
turbulence intensity ($\sim$0.2\%) in the tunnel. Cylinder surface is very
smooth and hence surface roughness is not modeled in the simulations.

Static testing is performed for (1) flow normal to the cylinder axis, and (2)
flow at an angle to the cylinder axis (inclined cylinder); three inclination
angles are analyzed in this study. These cases are simulated at Reynolds
number $Re_D=20,000$ which is corresponding to laminar separation (LS), where 
flows are laminar before separation and transition to
turbulence occurs in the shear layer.

Dynamic testing is performed for an elastically-mounted cylinder in eight different 
inflow reduce velocities with 
(1) flow normal to the cylinder axis, and (2) flow at an angle to the 
cylinder axis (inclined cylinder). However, due to high computational cost of simulations,
only one inclination angle is analyzed in the dynamic study($45^\circ$). The cylinder
is limited to vibrate along the direction which is perpendicular to both 
the flow direction and the cylinder axis. All
dynamic cases are simulated at the same Reynolds number, $Re_{D,n}=20,000$.

%%%%%%%%%%%%%%%%%%%%%%%%%%%%%%%%%%%%%%%%%%
\subsection{Mesh Sensitivity Study}
\label{sec:mesh_sensitivity}
%%%%%%%%%%%%%%%%%%%%%%%%%%%%%%%%%%%%%%%%%%
%

A mesh sensitivity study is performed for the normally-incident flow category
for $Re_D=20,000$. Table~\ref{tab:meshSize}
presents a summary of the different cases simulated for this study.  Three
different meshes, labeled `Mesh 1', `Mesh 2', and `Mesh 3' are shown in the table.
The computational domain is
$25\times D$ in the radial direction and $10\times D$ in the span direction.
%
\begin{table}[htb!]
  \caption{Summary of the test cases simulated to investigate sensitivity of results to mesh size} 
  \label{tab:meshSize} 
  \begin{center}
  \begin{tabular}{c|c|c|c}
    $\boldsymbol{Re_D}$ & \textbf{Mesh name} &  \textbf{Cell count}$\boldsymbol{(\theta,r,z)}$ \\ \hline
    \hline
    20,000  & Mesh 1 & 157, 233, 65\\ \hline
    20,000  & Mesh 2 & 236, 343, 100\\ \hline
    20,000  & Mesh 3 & 354, 507, 150\\ \hline \hline
    \hline
  \end{tabular}
  \end{center}
\end {table}
%

Figure~\ref{fig:Cp_compare_LS_Mesh} compares the mean aerodynamic pressure
coefficient, $\overline{C}_p=2 (\overline{p}-p_\infty)/(\rho V_\infty^2)$
obtained using simulations with different grids, and experimental data.
Experiment I refers to data from~\cite{norberg2013pressure} and Exp-ISU is from
our measurements. Simulations are simulated at `cylinder diameter based'
Reynolds number, $Re_D=20,000$, which is the same as Experiment I, but the
$Re_D$ in Exp-ISU is higher ($Re_D=51,500$) due to the limitation of the wind
tunnel and measurement equipment.

Consistent with the results of~\cite{travin2000detached}
and~\cite{breuer2000challenging}, the simulation results are found to be
sensitive to mesh size even with the highest mesh resolution attempted. Mesh 2
and Mesh 3 capture the flow separation location correctly (same as in the
experiment), but separation is delayed with Mesh 1. Results of Mesh 2 are in
closer agreement with Experiment I, however they differ from the results of
Mesh 3 in the separated flow region. Interestingly, the Exp-ISU data agrees
well with Mesh 3 results in the same region. Based on this study, Mesh 2 
is chosen for the subsequent simulations because it can predict aerodynamic
loading and resolve wake turbulence reasonably accurately within a reasonable 
computational cost.
%
\begin{figure}[htb!]
  \centering
  \incfig[width=.5\textwidth]{Figures/Cp_Compared_Mesh_Re20k.png} 
  \caption{$\overline{C}_p$ comparisons between simulation results using
  different meshes.}
  \label{fig:Cp_compare_LS_Mesh}
\end{figure}


%%%%%%%%%%%%%%%%%%%%%%%%%%%%%%%%%%%%%%%%%%
\subsection{Static Cylinder with Normally-Incident Flow}
\label{sec:normally_incident}
%%%%%%%%%%%%%%%%%%%%%%%%%%%%%%%%%%%%%%%%%%


Table~\ref{tab:comparisonRe20k} summarizes the simulation results for the
static case and compares them with two sets of experimental
data. The peak shedding Strouhal number ($St_p$), the mean drag coefficient,
$\overline{C}_d$ and the mean back pressure coefficient, $\overline{C}_{pb}$
are compared in the table. Strouhal number is a non-dimensional frequency,
defined here as $St = f\,D/V_\infty$, and $St_p$ corresponds to the peak vortex
shedding frequency of the cylinder.

Figure~\ref{fig:Cp_compared_Re20k} compares the predicted mean aerodynamic
pressure coefficient, $\overline{C}_p$ and the root mean square of perturbation
pressure coefficient, $C_{p'rms} = \left(\,\overline{C^2_p -
\overline{C}^2_p}\,\right)^{1/2}$ with the data from the two experiments. The
predicted $\overline{C}_p$ agrees very well with the data from Experiment I;
Exp-ISU data shows slightly lower $\overline{C}_p$ than observed in
Experiment I and the simulation, especially after $100^\circ$, and the mean
back pressure, $\overline{C}_{pb}$ is lower as well. The prediction
$C_{p'rms}$ distribution lies in between the two measurements. Both
measurements as well as the simulation show the peak to be around $80^\circ$,
which indicates the location of flow separation. The
predicted distribution over the cylinder surface agrees well with the
measurements.
%
\begin{table}[htb!]
  \caption{Summary of results for normally-incident flow simulations} 
  \label{tab:comparisonRe20k} 
  \begin{center}
  \begin{tabular}{c|c|c|c|c|c}
      $\boldsymbol{Re_D}$ & \textbf{Method} & $\boldsymbol{St_p}$ & $\boldsymbol{\overline{C}_d}$ & $\boldsymbol{\overline{C}_{pb}}$ \\ \hline
      \hline
      20,000 & $k-\omega$ DDES & 0.21 & 1.13 & -1.16 \\ \hline
      20,000 & Experiment I    & 0.19 & 1.22 & -1.1  \\ \hline
      51,500 & Exp-ISU   & 0.21 & 1.14 & -1.3  \\ \hline
      \hline
  \end{tabular}
  \end{center}
\end {table}

\begin{figure}[htb!]
  \centering
  \subcaptionbox{Mean aerodynamic pressure coeff., $\overline{C}_p$}
    [.48\linewidth]{\incfig[width=.48\textwidth]{Figures/Cp_Compared.png}}
  \hspace*{\fill}
  \subcaptionbox{r.m.s. of pressure coeff., $C_{p'rms}$ }
    [.48\linewidth]{\incfig[width=.48\textwidth]{Figures/CpRMS_Compared.png}}
  \caption{Comparisons of mean and rms of aerodynamic pressure coefficient
  between the simulation and experimental measurements.}
  \label{fig:Cp_compared_Re20k}
\end{figure}

Figure~\ref{fig:velocity_Normal_Re20k} plots the predicted and measured wake
velocity profiles at the axial station, $x/D=2$; the cylinder axis is located
at $x/D=0$. The peak wake deficit and the wake profile are predicted
accurately. The measured data shows a slight asymmetry in the data, which is
perhaps due to asymmetry in the experimental setup (e.g., distance from the
tunnel wall between the top and bottom surfaces of the cylinder). The simulation data is
averaged over 120 wake shedding cycles and experimental data is averaged over
540 cycles.
%
\begin{figure}[htb!]
  \incfig[width=.5\textwidth]{Figures/velocity_Normal_Re20k.png}
  \caption{Comparison of predicted and measured velocity profiles in the
    cylinder wake $2D$ downstream of the cylinder axis.}
  \label{fig:velocity_Normal_Re20k}
\end{figure}

Figure~\ref{fig:force_20k} presents predicted temporal variation of sectional
lift and drag coefficients ($C_l$ and $C_d$). As expected for a circular
cylinder, the mean lift coefficient ($\overline{C}_l$) is zero but the mean
drag coefficient ($\overline{C}_d$) is finite. The high-frequency oscillations,
more apparent in $C_l$ time history are due to Karman vortex shedding, which
occurs at $St=fd/U \sim 0.2$ for bluff bodies in the range of $Re_D$ considered
here. In addition to the oscillations at the Karman vortex shedding frequency, the
entire signal appears to modulate at a frequency which is an order of magnitude
lower than that corresponding to $St=0.2$. This modulation has a certain
randomness to it and is not perfectly periodic. This modulation phenomenon has
been reported elsewhere, see e.g.,~\cite{travin2000detached}. 
%
\begin{figure}[htb!]
  \incfig[width=0.9\textwidth]{Figures/force_20k.png}
  \caption{Predicted temporal variations of lift and drag coefficients}
  \label{fig:force_20k}
\end{figure}

Figure~\ref{fig:St_Compared_Re20k} (a) compares the power spectral densities of
lift coefficient ($C_l$) between data from Exp-ISU and the simulation. 
The lift in the measurements is obtained by integrating the surface
pressure measured using the pressure taps. Figure~\ref{fig:St_Compared_Re20k} (b)
presents the DES computed spectra of $C_d$.  Because vortex shedding alternates
between the top and bottom sides of the cylinder, one vortex shedding period contains two cycles
of drag but only one cycle of lift. This can be seen in
Figure~\ref{fig:St_Compared_Re20k}, where the spectral peak for $C_l$ occurs at
$f_p$ while the spectral peak for drag is at $2\times f_p$, where $f_p$ is determined
by the peak shedding Strouhal number, $St_p = f_p\,D/V_\infty$. Based on existing
literature (see~\cite{travin2000detached} and~\cite{norberg2013pressure}),
$St_p \sim 0.2$. Both measurement and prediction agree very well with each
other and show the peak for lift to be around $f_p$ corresponding to $St_p$.
%
\begin{figure}[htb!]
  \centering
  \subcaptionbox{PSD of $C_l$ }
    [.48\linewidth]{\incfig[width=.48\textwidth]{Figures/St_Compared_Cl_Re20k.png}}
  \hspace*{\fill}
  \subcaptionbox{PSD of $C_d$}
    [.48\linewidth]{\incfig[width=.48\textwidth]{Figures/St_Compared_Cd_Re20k.png}}
    \caption{Comparison of predicted and measured power spectral densities
      (PSDs) of $C_l$ and $C_d$. The
      measured data here is from our experiments (Exp-ISU).}
\label{fig:St_Compared_Re20k}
\end{figure}

The peak frequency and its first three harmonics that occur at $St = 0.4, 0.6,
\& \,0.8$, are identified in the figure using vertical grid lines and labeled
as $2f_p,\,3f_p,\,\&\,4f_p$. The prediction and experiment both show a second,
smaller peak in the lift spectrum at the third harmonic ($St=0.6$).  Since the
lift vector alternates with the side the vortex sheds from, only odd harmonics
of $f_p$ (i.e., $3f_p,~5f_p,\ldots$) are expected in the spectra. Therefore, no
peak is observed in the lift spectra at the second harmonic ($St=0.4$) or
higher {\em even} harmonics in either the measured or the simulated data. The
spectral shape of the PSD of $C_l$ is correctly predicted, even though the
magnitude is slightly higher than the measured data.



%%%%%%%%%%%%%%%%%%%%%%%%%%%%%%%%%%%%%
\subsection{Static Cylinder with Yawed Flow (Inclined Cylinder)}
\label{sec:inclined_cylinder}
%%%%%%%%%%%%%%%%%%%%%%%%%%%%%%%%%%%%%
%
Figure~\ref{fig:yawedModel} is a schematic that illustrates the setup for the
inclined-cylinder simulations. The relative inclination of the cylinder axis
with respect to the flow is obtained by yawing the flow rather than inclining
the cylinder; these simulations are therefore also referred to as yawed flow
simulations. Other than yawing the inflow, the setup is exactly the same as for
the normally-incident flow cases.

The yaw angle, $\beta$ is defined as the angle between the inflow velocity
vector $\boldsymbol{V_\infty}$ and the $x$ axis; the cylinder is aligned with
the $z$ axis. The normal component of the flow velocity, $V_n$ and the spanwise
component, $V_z$ are defined as $V_n=V_\infty \cos\,\beta$ and $V_z = V_\infty
\,\sin\,\beta$, where $V_\infty=\abs{\boldsymbol{V_\infty}}$. The computational
domain is $10\times D$ in the spanwise direction to resolve and investigate 
spanwise variation of aerodynamic forces.
Three different yaw angles $\beta=15,\,30,\,\&\,45$ degrees are investigated
in these simulations.
%
\begin{figure}[htb!]
  \incfig[width=0.6\textwidth]{Figures/yawedModel.png}
  \caption{A schematic of the computational setup for static inclined cylinder
    simulations. The right figure is a cross-sectional view.  The inflow is set
    to an angle with respect to the cylinder axis, which stays aligned with the $z$
    axis of the coordinate system.}
  \label{fig:yawedModel}
\end{figure}


Table~\ref{tab:comparisonYawedRe20k} summarizes the peak Strouhal number
($St_p$) and the back pressure coefficient ($C_{pb}$) for four different flow
yaw angles, $\beta=0,15,30,\,\&\,45$ degrees. 
The component of the freestream velocity normal to the cylinder axis is used as the reference
velocity scale to define a new set of non-dimensional quantities, such as
Reynolds number, $Re_{D,n} = \rho V_n D / \mu$, Strouhal number, $St_{p,n}=f\,D
/V_n$, and aerodynamic pressure coefficient, $\overline{C}_{p,n}=2
(\overline{p}-p_\infty)/(\rho V_n^2)$. The peak shedding Strouhal number and
the mean back pressure coefficient normalized in this manner are labeled
respectively as $St_{p,n}$ and $\overline{C}_{pb,n}$. The measured value of
$\overline{C}_{pb,n}$ is lower than predicted by the simulations (see
Table~\ref{tab:comparisonYawedRe20k}), suggesting that the measured mean
velocity at the back of the cylinder is slightly higher than predicted.
%
%%%%%%%%%%%%%%%%
\begin{table}[htb!]
  \caption{Summary of simulation results for four different flow yaw angles
  ($\beta=0, 15, 30,~\&~45$ deg). Experimental data is only shown for
  $\beta=30^\circ$.} 
  \label{tab:comparisonYawedRe20k} 
  \begin{center}
  \begin{tabular}{c|c|c|c|c|c}
      \textbf{Method} & \textbf{flow angle,} $\boldsymbol{\beta}$ & $\boldsymbol{Re_D}$ & $\boldsymbol{Re_{D,n}}$  & $\boldsymbol{St_{p,n}}$ & $\boldsymbol{\overline{C}_{pb,n}}$  \\ \hline
      \hline
      Simulation  & $0^\circ$  & 20,000 & 20,000 & 0.21 & -1.15 \\ \hline
      Simulation  & $15^\circ$ & 20,000 & 19,318 & 0.21 & -1.11 \\ \hline
      Simulation  & $30^\circ$ & 20,000 & 17,320 & 0.20 & -1.11  \\ \hline
      \rowcolor[gray]{.9}
      Exp-ISU     & $30^\circ$ & 51,500 & 44,600 & 0.19 & -1.27  \\ \hline
      Simulation  & $45^\circ$ & 20,000 & 14,142 & 0.21 & -1.16  \\ \hline
      \hline
  \end{tabular}
  \end{center}
\end{table}

Figure~\ref{fig:Cp_Compared_Yawed_Exp-ISU} compares with measured data the
predicted mean aerodynamic pressure coefficient ($\overline{C}_{p,n}$) and root
mean square of perturbation pressure coefficient, $C_{p'rms,n}$ normalized
using $V_n$, for $\beta=30^\circ$ case. The predicted distribution of
$\overline{C}_{p,n}$ and mean back pressure, $\overline{C}_{pb,n}$ are found to be
slightly higher than Exp-ISU data, which is consisted with the observation for
the normally-incident flow cases. The predicted $C_{p'rms,n}$ distribution
agrees very well with measurement, especially before $120^\circ$. The peak of
$C_{p'rms,n}$ is observed around $80^\circ$ in both experiment and simulation,
which is indicative of the location of separation of the shear layer. Beyond
$\theta=120^\circ$, the measured data shows higher $C_{p'rms,n}$ than
predicted by the simulations. A similar underprediction is observed in the
normally-incident flow case.
%
\begin{figure}[htb!]
  \centering
  \subcaptionbox{Mean pressure distribution,$\overline{C}_{p,n}$}
    [.48\linewidth]{\incfig[width=.48\textwidth]{Figures/Cp_Compared_Yawed_Exp-ISU.png}}
  \hspace*{\fill}
  \subcaptionbox{rms of perturbation pressure, $C_{p'rms,n}$ }
    [.48\linewidth]{\incfig[width=.48\textwidth]{Figures/CpRMS_Compared_Yawed_Exp-ISU.png}}
  \caption{Comparisons between simulation and experimental measurements for
    $\beta=30^\circ$ yawed-flow case.}
\label{fig:Cp_Compared_Yawed_Exp-ISU}
\end{figure}
%

%%%%%%%%%%%%%%%%
Figure~\ref{fig:velocity_Yawed30_Re20k.png} presents predicted and measured
wake velocity profiles for $\beta=30^\circ$ case at $x/D=2$. The predicted peak
wake deficit matches remarkably well with Exp-ISU data. However, the
experimental data shows higher overshoots in streamwise velocity in the shear
layer region than predicted by the simulations. The experimental data is also
very slightly asymmetric, which is likely due to the fact that the cylinder is
located closer to one side of the tunnel wall. It should be noted that the
asymmetry is very small and the wall effects are minimal.
%
\begin{figure}[htb!]
  \incfig[width=.5\textwidth]{Figures/velocity_Yawed30_Re20k.png}
  \caption{Comparison of predicted and measured velocity profiles for
    $\beta=30^\circ$ yawed flow in the cylinder wake $2D$ downstream of the cylinder axis}
  \label{fig:velocity_Yawed30_Re20k.png}
\end{figure}
%

%%%%%%%%%%%%%%%%
Figure~\ref{fig:St_Compared_Yawed30_Exp-ISU} compares the power spectral
densities of the transverse force coefficient (along the $y$ axis), $C_{y,n}$
for $\beta=30^\circ$ case between Exp-ISU data and predictions, where $C_{y,n}
= 2\, F_y/ \left( \rho V^2_n (D \times L) \right)$, $F_y$ is the net force over
the entire cylinder; longitudinal force coefficient, $C_{x,n}$ is similarly
defined. In the simulation, the first peak is observed around $St_{p,n}\sim
0.2$, which is the same as for the normally-incidence flow case (see
Figure~\ref{fig:St_Compared_Re20k} (a)). The spectral shape is correctly
predicted by the simulation although the measured curve appears to be shifted
along the $x$ axis; this is likely due to a scaling factor in frequency (log
scale is used for frequency in Fig.~\ref{fig:St_Compared_Yawed30_Exp-ISU}),
arising perhaps from a slight mismatch in the measurement of the inflow
velocity in the experiment.
%
\begin{figure}[htb!]
  \incfig[width=.6\textwidth]{Figures/St_Compared_Yawed30_Exp-ISU.png}
  \caption{Comparison of predicted and experimental power spectral densities
    (PSDs) of force coefficient $C_{y,n}$ for $\beta=30^\circ$ yawed-flow
    cases.}
  \label{fig:St_Compared_Yawed30_Exp-ISU}
\end{figure}
%
%%%%%%%%%%%%%%%%%%%%%%
Figure~\ref{fig:Compared_Yawed_Re20k} compares the predicted mean aerodynamic
pressure coefficient ($\overline{C}_{p,n}$) normalized using $V_n$, for four
different values of inflow yaw angle, $\beta$. The distribution of
$\overline{C}_{p,n}$ is found to be very similar irrespective of
$\beta$;~\cite{zdravkovich2003flow} refers to this as `independence principle'.
The independence principle is also observed in the power spectral densities of
the transverse force coefficient, $C_{y,n}$ for the same set of values of $\beta$
analyzed. Figure~\ref{fig:Compared_Yawed_Re20k} (b) shows that the spectra
collapse when $V_n$ is used to normalize the force coefficients and the frequency;
the abscissa in Fig.~\ref{fig:Compared_Yawed_Re20k} (b) is $St_n$.  
%
%
\begin{figure}[htb!]
  \subcaptionbox{Mean pressure coeff., $\overline{C}_{p,n}$}[0.48\linewidth]
    {\incfig[width=.48\textwidth]{Figures/Cp_Compared_Yawed_Re20k.png}}
  \hspace*{\fill}
  \subcaptionbox{PSDs of $C_{y,n}$}[0.48\linewidth]
    {\incfig[width=.48\textwidth]{Figures/St_Compared_Yawed_Cl_Re20k.png}}
  \caption{Independence principle: comparisons of (a) $\overline{C}_{p,n}$, and
    (b) power spectral densities (PSDs) of $C_{y,n}$ between predictions for various $\beta$ values.}
\label{fig:Compared_Yawed_Re20k}
\end{figure}

Figure~\ref{fig:Spatial_temporal_CxCyRe20k} shows spatio-temporal plots of the
the force coefficients $C_{x,n}$ and $C_{y,n}$. Here, $C_{x,n}$ and $C_{y,n}$
are functions of spanwise location and are computed by normalizing the
sectional forces in $x$ and $y$ directions respectively ($C_{x,n} = 2\, f_x/
(\rho V^2_n)$, where $f_x$ is force per unit area in the $x$
direction). The coefficients are plotted as functions of span ($z$) and time to
obtain the contours shown in the figure. The contours clearly show that the
force coefficients vary along the span, indicating that vortex shedding does
not occur simultaneously along the entire span. In fact, a spatial drift from
left to right with increasing time can be seen in the contours (more visible in
the $C_{x,n}$ spatio-temporal plot) which is indicative of spanwise flow over
the cylinder.

%
\begin{figure}[htb!]
  \subcaptionbox {$C_{x,n}$}
    [.48\linewidth]{\incfig[width=.48\textwidth]{Figures/Spatial_temporal_Cx_Re20k_Yawed30.png}}
  \hspace*{\fill}
  \subcaptionbox{$C_{y,n}$ }
    [.48\linewidth]{\incfig[width=.48\textwidth]{Figures/Spatial_temporal_Cy_Re20k_Yawed30.png}}
  \caption{Spatio-temporal distribution of force coefficients at
  $\beta=30^\circ$}
  \label{fig:Spatial_temporal_CxCyRe20k}
\end{figure}

Figure~\ref{fig:Coherence_Yawed30_Re20k} presents coherence of force
coefficients for $\beta=30^\circ$ case. Magnitude squared coherence,
$\gamma^2(\Delta z)$ is defined as
%
\begin{align}
  \gamma^2(\Delta z) &= \frac{\langle \abs{S_{xy}}^2 \rangle}{\langle
    S_{xx}\rangle \langle S_{yy} \rangle},
  \label{eq:coherence}
\end{align}
%
where $S_{xy}$ denotes cross-spectral density of the quantity ($C_{x,n}$ or
$C_{y,n}$) at two points separated by a distance $\Delta z$, and
$S_{xx},~S_{yy}$ are auto-spectral densities; angular brackets denote ensemble
averaging, however ergodicity assumption is used to relate that to time
averaging. The coherence plot of $C_{y,n}$ indicates that spanwise correlation
is very high (over nearly the entire cylinder span) at the vortex shedding
frequency, but is small at other frequencies, which is expected based on
literature. $C_{x,n}$ however is not that highly correlated along the span even
at the peak vortex shedding frequency.
%
\begin{figure}[htb!]
  \subcaptionbox {$C_{x,n}$}
    [.48\linewidth]{\incfig[width=.48\textwidth]{Figures/Cd_coherence_Yawed30_Re20k.png}}
  \hspace*{\fill}
  \subcaptionbox{$C_{y,n}$ }
    [.48\linewidth]{\incfig[width=.48\textwidth]{Figures/Cl_coherence_Yawed30_Re20k.png}}
    \caption{Magnitude squared coherence, $\gamma^2(\Delta z)$ of transverse
      and longitudinal force coefficients, $C_{x,n}$ and $C_{y,n}$ for
      $\beta=30^\circ$ case.}
  \label{fig:Coherence_Yawed30_Re20k}
\end{figure}


%%%%%%%%%%%%%%%%%%%%%%%%%%%%%%%%%%%%%%%%%%%%%%%%%%
\subsection{Vortex-Induced Vibration (VIV)}
\label{sec:VIV}
%%%%%%%%%%%%%%%%%%%%%%%%%%%%%%%%%%%%%%%%%%%%%%%%%%
%
A schematic of the computational setup for the vortex-induced vibration (VIV)
simulations is presented in Figure~\ref{fig:VIVmodel}. The setup is the same as
for the static simulations except for an additional forced mass-spring-damper
system. The cylinder is allowed to move only in the $y$ (cross-stream)
direction. The $y$ component of the integrated aerodynamic surface force on the
cylinder (denoted by $F_y$) drives the mass-spring-damper system given by
%
\begin{equation}
  m \frac{{\rm d}^2 x}{{\rm d}t^2} + c \frac{{\rm d} x}{{\rm d}t} + k x = F_y(t).
  \label{eq:mass-spring-damper}
\end{equation}
%
In the simulations the mass ratio $m^*=m/(\rho {\cal V})=2.6$, where $m$ is the
mass of the cylinder, ${\cal V}=\pi (D^2/4) . S$ is the volume of the cylinder,
$S$ is the cylinder span, and $\rho$ is the density of the fluid flowing over
the cylinder. The mechanical damping ratio of the system $\zeta = c/c_c$ is
$0.001$ where, $c_c=2\sqrt{k.m}$ is the critical damping, and
$k={\color{red}???}$ is the spring stiffness. The values of these parameters
are selected to match the measurements presented in~\cite{franzini2013one}.
This measurement dataset is referred as Exp II in this paper. The predictions
are also compared to another dataset reported in~\cite{khalak1997fluid}, which
is referred as Exp III here. The mass ratio and damping ratio used in Exp III
are slightly different ($m^*=2.4$ and $\zeta=0.0045$) from Exp II and the
simulations. It should be noted that the measurement results have end effects
due to the finite length of the cylinder. {\color{red} summarize here} The
simulations use periodic boundary conditions in the span direction, which
theoretically simulates an infinite span. However, span-periodicity can induce
artificial effects if the spanwise coherence is greater than the simulated
span.
%
\begin{figure}[htb!]
  \incfig[width=.6\textwidth]{Figures/VIV_setup.jpg}
  \caption{A schematic of the computational setup for oscillating cylinder
    simulations. The right figure is a cross-sectional view.  The inflow is set
    to an angle with respect to the cylinder axis, which stays aligned with the $z$
    axis of the coordinate system.}
  \label{fig:VIVmodel}
\end{figure}

Figure~\ref{fig:Amplitude_VIV} compares the predicted non-dimensional
mean amplitude $\bar{A}/D$ with the measurements of Exp II and III over a wide
range of reduced velocity $V_{R,n}$, which is defined as
$V_{R,n}=V_n/(f_N\,D)$, where $f_N$ is the natural frequency of the system. The
subscript $n$ refers to the component of the vector normal to the cylinder axis
to accommmodate for yawed flow. Two different yaw angle flows are evaluated --
$\beta=0^\circ$ and $45^\circ$ -- both at $Re_{D,n}=20,000$.

\citet{khalak1997fluid} identified the following four distinct branches in
their VIV measurements for the zero-yaw case: the ``initial excitation''
branch, the ``upper'' branch, the ``lower'' branch, and the
``desynchronization'' branch. These are labeled and identified with solid black
lines as best curve fits of the measured data in Fig.~\ref{fig:Amplitude_VIV}
(a). In the initial excitation branch, the mean amplitude grows rapidly with
$V_{R,n}$. The scaled displacement oscillation amplitude ($A/D$) reaches a peak
in the upper branch, drops to 60\% of the peak value in the lower branch, and
then finally drops to a negligible value at higher $V_{R,n}$ in the
desynchronization branch. The current DES simulations agree with the data
(particularly with Exp III) very well in the initial excitation and upper
branches. The predicted amplitude is slightly lower than the measurements in
the lower and desynchronization branches. Considering the relatively large
differences in the two sets of measurements (between Exp II and Exp III), the
prediction accuracy of the simulations is very good.
%
\begin{figure}[htb!]
  \centering
  \subcaptionbox{$\beta=0^\circ$}%
    [.48\linewidth]{\incfig[width=.48\textwidth]{fig/viv_amp_noyaw.pdf}}
  \hspace*{\fill}
  \subcaptionbox{$\beta=45^\circ$}%
    [.45\linewidth]{\incfig[width=.45\textwidth]{fig/viv_amp_yaw45_wInset.pdf}}
    \caption{Comparison of predicted and experimental non-dimensional mean
      amplitude $A/D$ over a range of reduced velocities $V_{R,n}$ for a)
      $\beta=0^\circ$, and b) $\beta=45^\circ$. The inset in the plot on the right
      shows the two setups (UP and DN) used in Exp II for yawed-flow
      measurements.} 
  \label{fig:Amplitude_VIV}
\end{figure}

Even though both yawed angle inflow simulations have a similar peak in the
upper branch, the predicted non-dimensional mean amplitude $\bar{A}/D$ quickly
drops to around 0.6 at $V_{R,n}=5.9$ for $\beta=45^\circ$ case. Exp II presents
very different results for $0^\circ$ and $45^\circ$ in the upper branch.
Because the end effect plays an important role in the experiment, that might be
the reason to cause this difference.  A lower branch region where the mean
amplitude $\bar{A}/D$ is almost constant can be observed on simulations and
Exp. III.  Though similar lower branch is not being found on Exp. II, it is
widely known and observed on other experiments by ~\cite{govardhan2000modes}.
Overall, the predicted mean amplitude agrees very well with Exp. III, but
presents some difference from Exp. II. Other than the upper branch, both yawed
flow simulations have observed very similar amplitude for various reduced
velocities. 

Figure~\ref{fig:f_VIV} presents non-dimensional frequency $f/f_N$ for various
reduced velocities $V_{R,n}$, with $f$ being the vortex shedding frequency in
the simulation while being the oscillating frequency in Exp.~III because vortex
shedding frequency data is not available in Exp.~III. The blue dash line
indicates vortex shedding frequency for the static cylinder ($St_p=0.21$),
while the red dash line presents the natural frequency.  As the figure shown,
the predicted non-dimensional frequencies for both yawed flows are very similar
and agree very well with Exp.~III.  In the initial excitation branch,
non-dimensional frequency $f/f_N$ increases as reduced velocity $V_{R,n}$
increases, which follows the blue dash line. It suggests that the oscillating
motion is independent of the natural frequency of the system in the initial
excitation branch. After that, the vortex shedding frequency locks onto the
natural frequency $f_N$ in the upper branch and lower branch, which is known as
"lock-in". Because this is a low mass-damping system, the ``lock-in'' frequency
is higher than the natural frequency, while it is significantly smaller than
the vortex shedding frequency of the static cylinder. The similar phenomenon
can also be observed on Exp.~III.  In the desynchronization branch, the vortex
shedding frequency desynchronizes from the natural frequency indicating the
ending of ``lock-in''.
%
\begin{figure}[htb!]
  \incfig[width=.5\textwidth]{fig/viv_freq.pdf}
  \caption{Non-dimensional frequency $f/f_N$ for various reduced velocities
    $V_{R,n}$}
  \label{fig:f_VIV}
\end{figure}

As Fig.~\ref{fig:Amplitude_VIV} shown, there is a jump between the upper branch
and the initial excitation branch and another jump between the upper branch and
the lower branch. This is because there are two different vortex shedding
patterns happened in this reduced velocity region. Figure~\ref{fig:Q_VIV}
displaces the Q-criterion for two reduced velocities $V_{R,n}=4$ and
$V_{R,n}=8$ for $\beta=45^\circ$ case, which represent the typical vortex
shedding modes of the initial excitation branch and the lower branch,
respectively. As the figure shown, when $V_{R,n}=4$, one vortex shedding period
includes two singular vortices (2S mode) shed alternately from either side of
the cylinder. On the contrary, when $V_{R,n}=8$ and ``lock-in'' happens,
periodic vortex shedding pattern switches to two pairs of vortices (2P mode).
Vortex shedding frequency lock onto the natural frequency as a result of 2P
mode. The vortex shedding mode of the upper branch would shift between these
two modes.

\begin{figure}[htb!]
  \subcaptionbox {Q-criterion for $V_{R,n}=4$}
    [.48\linewidth]{\incfig[width=.48\textwidth]{Figures/Q_45_RV4_1.png}}
  \hspace*{\fill}
  \subcaptionbox{Q-criterion for $V_{R,n}=8$ }
    [.48\linewidth]{\incfig[width=.48\textwidth]{Figures/Q_45_RV8_1.png}}
  \vskip\baselineskip
   \subcaptionbox {A schematic of 2S mode for $V_{R,n}=4$}
    [.48\linewidth]{\incfig[width=.48\textwidth]{Figures/2S_RV4.png}}
  \hspace*{\fill}
  \subcaptionbox{A schematic of 2P mode $V_{R,n}=8$ }
    [.48\linewidth]{\incfig[width=.48\textwidth]{Figures/2P_RV8.png}}
    \caption{Q-criterion and vortex shedding modes for $\beta=45^\circ$ yawed flow simulation} 
  \label{fig:Q_VIV}
\end{figure}

The force coefficients are shown in Fig.~\ref{fig:force_VIV}. It is well known
that the vibration motion can significantly increase the fluctuation of forces.
Mean transverse force coefficient $\bar{C}_{x,n}$ for reduce velocities
$V_{R,N} < 5.9$ have a very similar curve as mean amplitude $\bar{A}/D$ in
Fig.~\ref{fig:Amplitude_VIV}. However, mean amplitude $\bar{A}/D$ is almost
constant in the lower branch while $\bar{C}_{x,n}$ continuously declines. The
predicted rms of longitudinal force coefficients $C_{y,n,rms}$ has very sharp
peak at $V_{R,n}=4$, which is slightly different from Exp. II. Similar to
$\bar{A}/D$ at $V_{R,n}=4$, $C_{y,n,rms}$ for $\beta=45^\circ$ is much larger
than $\beta=0^\circ$. Overall, two yawed angle flow simulations have very
similar results for small amplitude vibrations. Independent principle can be
applied to forces coefficients except for the upper branch ($4 \leqslant
V_{R,n}\leqslant 5.9$).

\begin{figure}[htb!]
  \subcaptionbox {$C_{x,n}$}
    [.48\linewidth]{\incfig[width=.48\textwidth]{Figures/Cd_VIV.png}}
  \hspace*{\fill}
  \subcaptionbox{$C_{y,n,rms}$ }
    [.48\linewidth]{\incfig[width=.48\textwidth]{Figures/Cl_VIV.png}}
    \caption{Mean transverse and rms of longitudinal force coefficients, $\bar{C}_{x,n}$ and $C_{y,n,rms}$ for
      $\beta=0^\circ$ and $45^\circ$ cases.}
  \label{fig:force_VIV}
\end{figure}

%%%%%%%%%%%%%%%%%%%%%%%%%%%%%%%%%%
\section{Conclusion}
\label{sec:conclusions}
%%%%%%%%%%%%%%%%%%%%%%%%%%%%%%%%%%%%
A computational methodology based on a $k-\omega$ delayed detached eddy
simulation (DDES) model and in-house experiments are used to investigate
aerodynamic loading on a smooth circular cylinder. Simulations are performed
for the cylinder in normally-incident flow (static and dynamic) and yawed flow
(3 cases). The computational methodology for predicting aerodynamic loading on
the cylinder is verified against experimental data in normally-incident flow
($\beta=0^\circ$) and yawed flow ($\beta=30^\circ$). The agreement between the
simulations and the experiments for normally-incident flow is very good, and
the results of the yawed flow simulation with $\beta=30^\circ$ in reasonable
agreement with the experiment. Overall, these comparisons show that the
computational methodology is able to accurately predict aerodynamic loading on
a static, smooth circular cylinder in smooth inflow.

Comparisons of simulation results for different flow angles ($\beta$) show that
the aerodynamic loads do not vary with yaw angle when the loads and frequency
are non-dimensionalized using the component of the flow velocity normal to the
cylinder axis. This indifference to yaw angle, referred to as the independence
principle, is observed for yawed flow up to $45^\circ$.

The numerical study of VIV of an elastically-mounted cylinder in
normally-incident flow agrees well with Exp III, but show some differences with
Exp II. The difference might be caused by the end effects of the experiment.
Other than the upper branch, the numerical results for two different yawed flow
($\beta=0^\circ$ and $45^\circ$) show reasonable agreement, which indicates
independence principle is applicable for the most regime of VIV, except for the
upper branch.

%%%%%%%%%%%%%%%%%%%%%%%%%%%%%%%%%%%%
\section{Acknowledgments}
\label{sec:acknowledgement}
%%%%%%%%%%%%%%%%%%%%%%%%%%%%%%%%%%%%
Funding for this research is provided by the National Science Foundation (Grant
\#NSF/ CMMI-1537917). Computational resources are provided by NSF XSEDE (Grant
\#TG-CTS130004) and the Argonne Leadership Computing Facility, which is a DOE
Office of Science User Facility supported under Contract DE-AC02-06CH11357.


%%%%%%%%%%%%%%%%%%%%%%%%%%%%%%%%%%%%
%\section{References}
%\label{sec:references}
\bibliographystyle{elsarticle-harv} 
\bibliography{references}
%%%%%%%%%%%%%%%%%%%%%%%%%%%%%%%%%%%%
\end{document}
