\documentclass[12pt,preprint]{elsarticle}
\usepackage[margin=1in]{geometry}
\usepackage{wrapfig}
\usepackage{amsmath}
\usepackage{bm}
\usepackage{makecell}
\usepackage[table]{xcolor}

\usepackage{subcaption}
% Font size for subcaption
\DeclareCaptionFont{mysize}{\fontsize{9}{9.6}\selectfont}
\captionsetup[sub]{font=mysize}

%\usepackage[mathlines,pagewise]{lineno}
%%\modulolinenumbers[2]

%\journal{Journal of Sound \& Vibration}

\newcommand{\hl}[1]{\textcolor{red}{#1}}
\newcommand{\hb}[1]{\textcolor{blue}{#1}}
\newcommand{\incfig}{\centering\includegraphics}
\newcommand{\abs}[1]{\left| #1 \right|} % for absolute value
\newcommand{\etal}{\textit{et al}. }
\newcommand{\ie}{\textit{i}.\textit{e}., }
\newcommand{\eg}{\textit{e}.\textit{g}., }
\usepackage{natbib}

\usepackage[colorlinks=true,citecolor=red]{hyperref}

\begin{document}

\begin{frontmatter}

%% Title, authors and addresses

%% use the tnoteref command within \title for footnotes;
%% use the tnotetext command for theassociated footnote;
%% use the fnref command within \author or \address for footnotes;
%% use the fntext command for theassociated footnote;
%% use the corref command within \author for corresponding author footnotes;
%% use the cortext command for theassociated footnote;
%% use the ead command for the email address,
%% and the form \ead[url] for the home page:
%% \title{Title\tnoteref{label1}}
%% \tnotetext[label1]{}
%% \author{Name\corref{cor1}\fnref{label2}}
%% \ead{email address}
%% \ead[url]{home page}
%% \fntext[label2]{}
%% \cortext[cor1]{}
%% \address{Address\fnref{label3}}
%% \fntext[label3]{}

\title{Finite-Span Effect on Vortex-Induced Vibration Simulations}

%% use optional labels to link authors explicitly to addresses:
%% \author[label1,label2]{}
%% \address[label1]{}
%% \address[label2]{}

\author[1]{Xingeng Wu}
\author[2]{Anupam Sharma\corref{a}}
\ead{sharma@iastate.edu}
\cortext[a]{Corresponding author}
\fntext[1]{Graduate Student}
\fntext[2]{Associate Professor, Iowa State University}

\address{Department of Aerospace Engineering, Iowa State University, Ames,
Iowa, 50011}


\begin{abstract}
  The effects of spanwise periodic boundary conditions to simulate vortex
  induced vibration (VIV) of finite-span cylinders are investigated. Flow over
  four elastically-mounted rigid circular cylinder models of varying span
  lengths are studied using detached eddy simulations. Spectra of integrated
  transverse loading and spanwise coherence of sectional force coefficient are
  analyzed to explain the observed differences. Aspect ratio of ten is found to
  be sufficient to accurately simulate VIV.
\end{abstract}

%\begin{keyword}
%  Vortex-Induced Vibration \sep 
%  Detached Eddy Simulations \sep 
%  spanwise coherence
%\end{keyword}
%
\end{frontmatter}

%\linenumbers

%\begin{linenumbers}
%% main text
%%%%%%%%%%%%%%%%%%%%%%%%
\section{Introduction}
\label{sec:intro}
%%%%%%%%%%%%%%%%%%%%%%%%
%
Vortex-induced vibrations (VIV) are commonly observed in bridge decks, cables,
power conductors, risers in oil rigs, etc. The K\'arm\'an vortex shedding in
the wake of the cylinder produces periodic forcing on the cylinder. In certain
conditions the vortex-shedding frequency synchronizes (``lock-in'') with the
natural frequency of the system which results in high-amplitude oscillations
limited only by the system damping. Experimental investigations of VIV are
typically performed with finite-span cylinders with aspect ratios greater than
$10$ to minimize ``end effects'' (*cite*), where end-effects refer to effects
due to three-dimensional flow at span ends. End plates have been used in
experiments (*cite*), which reduce but not completely eliminate the end effects
as horse-shoe vortices develop at the intersection of the cylinder and the end
plates.

This problem is avoided in simulations by using periodic boundaries in the span
direction. While the periodic boundaries imply an infinitely-long cylinder, the
finite size of the computational domain in the span direction imposes
artificial periodicity in the flow. If spanwise variations are present in the
flow and the length scale of these variations is larger than the simulated
span, then span periodicity will likely yield incorrect results. For a
turbulent flow (stochastic system), this length scale can be measured using
two-point correlations; in particular, spanwise coherence. Magnitude-squared
coherence is defined as 
%
$\gamma^2(\Delta z,f) = \langle \abs{S_{xy}(f)}^2 \rangle/(\langle
    S_{xx}(f)\rangle \langle S_{yy}(f) \rangle)$,
%
where $S_{xy}(f)$ denotes cross-spectral density of the desired quantity (e.g.,
sectional force) at points $\bm{x}$ and $\bm{y}$ separated by a distance $\Delta
z$ (along the span). $S_{xx}(f)$ and $S_{yy}(f)$ are auto-spectral densities at
$\bm{x}$ and $\bm{y}$ respectively, and angle brackets denote ensemble
averaging.
%
\begin{wrapfigure}[13]{r}{0.4\textwidth}
  \incfig[width=0.4\textwidth]{Figures/Coherence_Cl_Z20}
  \caption{$\gamma^2(\Delta z,f)$ for a static cylinder}
  \label{fig:Coherence_staic}
\end{wrapfigure}
%
Figure~\ref{fig:Coherence_staic} presents contours of $\gamma^2(\Delta
z,\omega)$ of sectional transverse force for a static circular cylinder (aspect
ratio, $L/D$=20) simulation. Nondimensional frequency, $k = f\,D/V_\infty$ is
used to plot coherence. Spanwise coherence is small everywhere except at the
K\'arm\'an vortex-shedding frequency ($f_v$), which is given by the Strouhal
number, $St = f_v D/V_\infty$ ($\sim 0.2$).

In this paper, we investigate the effect of span periodicity when simulating a
circular cylinder experiencing VIV. While the effect of aspect ratio (span
length to diameter ratio, $L/D$) have been investigated in VIV experiments
(*cite*), such an investigation is lacking for simulations. (* provide a very
brief summary of study on aspect-ratio-effects for flow over a cylinder *) We
present detached eddy simualtion (DES) results of four models with
$L/D=1,~2,~5,$ and $10$. Displacement amplitude, oscillation frequency, lift
spectra and spanwise coherence are analyzed to study the effect of span
periodicity.


%%%%%%%%%%%%%%%%%%%%%%%%%%%%%%%%%%%%%%%%%%
\section{Computational Methodology and Verification} 
\label{sec:methodology}
%%%%%%%%%%%%%%%%%%%%%%%%%%%%%%%%%%%%%%%%%%
%
A coupled fluid-solid dynamics solver is used to simulate an
elastically-mounted rigid circular cylinder experiencing VIV. The $k$-$\omega$
detached eddy simulation (DES) technique~\cite{yin2015dynamic} is used to model
the flow and a forced single-degree of freedom mass-spring-damper system is
solved to model the dynamics of the cylinder. The details of the numerical
methodology are described in Wu \etal \citep{wu2019}.

Figure~\ref{fig:VIV_verification} (a) shows a schematic of the simulation setup
and a comparison with measured data from Ref.~\cite{khalak1997fluid} for a low
mass-damping cylinder undergoing VIV. The following nondimensional numbers are
matched between the experiment and the simulations: mass ratio, $m^*=2.6$,
mechanical damping ratio, $\zeta=0.001$, and reduced velocity, $V_R =
V_\infty/(f_N D)$, where $f_N$ is the natural frequency of the system (mounted
cylinder). The flow Reynolds number based on the cylinder diameter,
$Re_D=2\times 10^4$. The simulations accurately predict the displacement
amplitude and the vortex shedding frequency of the cylinder (see
Fig.~\ref{fig:VIV_verification}) over a wide range of $V_R$ which includes the
four branches identified in Ref.~\cite{khalak1997fluid}: {\em Initial
Excitation}, {\em Upper}, {\em Lower}, and {\em Desynchronization}. Of
particular interest is the ``lock-in'' phenomenon which occurs in the {\em
Upper} and {\em Lower} branches where the chances of structural damage are
highest. 

\begin{figure}[htb!]
  \subcaptionbox{simulation setup}{\incfig[width=.32\textwidth]{Figures/plot/VIV_setup}} \;
  \subcaptionbox{mean amplitude}  {\incfig[width=.32\textwidth]{Figures/validation/viv_amp_noyaw.pdf}} \;
  \subcaptionbox{frequency}       {\incfig[width=.32\textwidth]{Figures/validation/viv_freq_noyaw.pdf}}
  \caption{Verification of the DES approach to predict VIV: (a) a schematic
    showing the simulation setup, (b) mean nondimensional displacement
    amplitude ($\bar{A}/D$), and (c) normalized oscillation frequency
    ($f_v/f_N$). Measured data is from Ref.~\cite{khalak1997fluid}.}
  \label{fig:VIV_verification}
\end{figure}


%%%%%%%%%%%%%%%%%%%%%%%%%%%%%%%%%%%%%%%%%%
\section{Effect of Aspect Ratio}
\label{sec:periodicity}
%%%%%%%%%%%%%%%%%%%%%%%%%%%%%%%%%%%%%%%%%%
%
Four cylinder models with aspect ratio, $L/D$ $= 1, 2, 5,$ and $10$ are
evaluated using DES with periodic span boundaries. For each configuration,
seven values of reduced velocity $V_{R}$ ($=2, 3, 4, 5, 5.9, 7, 8$) are
evaluated; this wide range of $V_R$ covers the {\em Initial Excitation}, {\em
Upper}, and {\em Lower} branches. Figure~\ref{fig:span_amp_freq} compares the
predicted scaled mean amplitude ($\bar{A}/D$) and the normalized oscillation
frequency ($f_v/f_N$) for the different models. The convergence of the results
for models with $L/D=$ $5$ and $10$ shows that span length of $10\times D$ is
adequate for VIV simulations. While the smaller-span models exhibit the same
qualitative trend, moderate-to-large differences are observed between them in
the {\em Initial Excitation} and {\em Upper} branches including
underprediction of the peak amplitude. 
%
\begin{figure}[htb!]
  \centering
    \subcaptionbox{mean amplitude}{\incfig[width=.4\textwidth]{Figures/validation/viv_amp_span}} \hfill
    \subcaptionbox{frequency     }{\incfig[width=.4\textwidth]{Figures/validation/viv_freq_span}}
  \caption{Comparison of predicted (a) ${\bar A}/D$, and (b) normalized
    oscillation frequency, using cylinder models of different span lengths.}
  \label{fig:span_amp_freq}
\end{figure}

Peak amplitude is observed at resonance when $f_v$ matches $f_N$ and is
expected to occur at $V_R=1/St$. Since the $St$ at the simulated $Re_D$ is
approximately $0.2$, peak amplitude occurs at $V_R\sim 5$. Power spectral
density (PSD) of the transverse aerodynamic force coefficient, $C_y = 2
F_y/(\rho V^2_\infty L)$ of the four models are compared in Fig.~\ref{fig:RV_5}
(a) for $V_R=5$. A linear scale is used to plot the PSDs to accentuate the
differences. The spectral peak is higher and narrower for the models with
$L\ge5D$. For shorter-span models, the energy is distributed over a wider
frequency range, resulting in a broader peak. A sharp peak in $C_y$ at $k \sim
St$ results in greater excitation, hence higher $\bar{A}/D$, for the
larger-span models at $V_R=5$.

Spanwise coherence of transverse sectional force coefficient, $c_y(z) = 2
f_y(z)/(\rho V^2_\infty)$ are plotted in Fig.~\ref{fig:RV_5} (b) for $V_R=5$.
For the larger-span cylinders ($L\ge5D$), high coherence ($\gamma^2$) is
limited to a very small band of $k$ around $St$ ($\sim0.2$), whereas for the
smaller-span models, $\gamma^2$ is very high over the entire span for
$0.1<k<0.3$. High coherence at frequencies away from $k=St$ is the reason for
the broader but shorter peaks observed in the $C_y$ spectra 
%
%and consequently, a lower peak $\bar{A}/D$ is observed at $k=St$ 
%
for the smaller-span models.

\begin{figure}[htb!]
  \subcaptionbox{$C_y$ spectra}     {\incfig[height=0.3\textwidth]{Figures/Coherence_St/RV5/spectra_RV5.pdf}} \quad
  \subcaptionbox{coherence of $c_y$}{\incfig[height=0.3\textwidth]{Figures/plot/RV5_Coherence.png}}
  \caption{Results at $V_R$=5: (a) PSD of $C_y$, and (b)
  $\gamma^2(\Delta z,k)$ of $c_y$ for models with $L/D$=1,2,5,\&10.}
  \label{fig:RV_5}
\end{figure}

Smaller-span models underpredict $\bar{A}/D$ throughout the {\em Upper} branch.
The solutions at $V_R=5.9$ are probed to investigate this underprediction.
Figure~\ref{fig:RV_5p9} plots the $C_y$ spectra and the spanwise coherence of
$c_y$ for the different models. The $k$ corresponding to the natural frequency
is $k_N = 1/V_R \sim 0.17$ for $V_R=5.9$. The $C_y$ spectra for the small-span
models peak at $k \sim 0.2$ and ``lock-in'' with $k_N$ does not occur. The
spectrum of the largest-span model peaks at around $k_N$; the peak is slightly
shifted from $k_N$ due to the added-mass effect, which can be substantial for
low-$m^*$ systems. High coherence in the small-span models forces the vortex
shedding at $k\sim 0.2$ and does not allow ``lock-in'' at $k_N$.

\begin{figure}[htb!]
  \subcaptionbox{$C_y$ spectra}     {\incfig[height=0.3\textwidth]{Figures/Coherence_St/RV5.9/spectra_RV5p9.pdf}} \quad
  \subcaptionbox{coherence of $c_y$}{\incfig[height=0.3\textwidth]{Figures/plot/RV5p9_Coherence.png}}
  \caption{Results at $V_R$=5.9: (a) PSD of $C_y$, and (b)
  $\gamma^2(\Delta z,k)$ of $c_y$ for models with $L/D$=1,2,5,\&10.}
  \label{fig:RV_5p9}
\end{figure}

In contrast to underprediction in the ``lock-in'' region, the smaller-span
models overpredict $\bar{A}/D$ in the {\em Initial Excitation} branch. This is
consistent with the $C_y$ spectra which shows a higher peak at $k \sim 0.2$ for
the smaller-span models (see Fig.~\ref{fig:RV_3} (a)). Another peak is observed
with the smaller-span models at $k$ near $k_N$ ($\sim 0.33$ for $V_R=3$).
Figure~\ref{fig:RV_3} (b) shows increased coherence at $k_N$. The periodicity
in smaller span models reinforces the excitation at this frequency leading to
the additional peak in the $C_y$ spectra. In fact, the smallest-span model
oscillates at this frequency (see Fig.~\ref{fig:span_amp_freq} (b)), suggesting
a ``lock-in'' in the {\em Initial Excitation} branch. 

\begin{figure}[htb!]
  \subcaptionbox{$C_y$ spectra}     {\incfig[height=0.3\textwidth]{Figures/Coherence_St/RV3/spectra_RV3.pdf}} \quad
  \subcaptionbox{coherence of $c_y$}{\incfig[height=0.3\textwidth]{Figures/plot/RV3_Coherence.png}}
  \caption{Results at $V_R$=3: (a) PSD of $C_y$, and (b)
  $\gamma^2(\Delta z,k)$ of $c_y$ for models with $L/D$=1,2,5,\&10.}
  \label{fig:RV_3}
\end{figure}

%\begin{figure}[htb!]
%  %\subcaptionbox{mean amplitude}{\incfig[width=.48\textwidth]{Figures/A.png}} \;
%  \subcaptionbox{frequency     }{\incfig[width=.48\textwidth]{Figures/Coherence_St/f_A}}
%  \caption{Comparison of predicted (a) ${\bar A}/D$, and (b) frequency using}
%\end{figure}

%%%%%%%%%%%%%%%%%%%%%%%%%%%%%%%%%%
\section{Conclusion}
\label{sec:conclusions}
%%%%%%%%%%%%%%%%%%%%%%%%%%%%%%%%%%%%
%
DES of four finite-span models with periodic span boundaries shows that a
minimum span length of $10\, D$ is required to accurately simulate VIV.
Simulations with smaller-span domains underpredict the peak amplitude of
vibration and do not capture the ``lock-in'' phenomenon accurately. Spanwise
coherence of sectional transverse force coefficient provides insights into the
observed behavior.

%%%%%%%%%%%%%%%%%%%%%%%%%%%%%%%%%%%%
\section{Acknowledgments}
\label{sec:acknowledgement}
%%%%%%%%%%%%%%%%%%%%%%%%%%%%%%%%%%%%
Funding for this research is provided by the National Science Foundation (Grant
\#NSF/ CMMI-1537917). Computational resources are provided by NSF XSEDE (Grant
\#TG-CTS130004) and the Argonne Leadership Computing Facility, which is a DOE
Office of Science User Facility supported under Contract DE-AC02-06CH11357.

%\bibliographystyle{elsarticle-num} 
\bibliography{references}
\bibliographystyle{model3a-num-names}

%% else use the following coding to input the bibitems directly in the
%% TeX file.
%\begin{thebibliography}{00}
%% \bibitem[Author(year)]{label}
%% Text of bibliographic item
%\bibitem[ ()]{}
%\end{thebibliography}

\end{document}
\endinput
