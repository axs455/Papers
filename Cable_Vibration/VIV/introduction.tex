%%%%%%%%%%%%%%%%%%%%%%%%
\section{Introduction}
\label{sec:intro}
%%%%%%%%%%%%%%%%%%%%%%%%
%
Inclined and horizontally- or vertically-spanned cylinder are used in various
engineering applications: cable-stayed, suspension, and tied-arch bridges,
power transmission lines, offshore risers, sub-sea pipelines etc. These
cylinders are prone to large-amplitude flow-induced vibration, which can lead
to catastrophic failure of the cylinders and the structures supported by them.
The vibration mechanisms involve complex aeroelastic (motion-induced)
interactions that depend on the spatial orientation, geometry,
surface-characteristics, and dynamic properties of
cylinders.~\cite{davenport1995dynamics} shows that large amplitude vibrations
can lead to either catastrophic- or fatigue failure of the cylinders and/or the
adjoining structures, which poses a significant threat to the safety and
serviceability of these systems.

Vortex-induced vibration(VIV) is an oscillating motion due to the interaction
between structures and periodical vortices. VIV is a very common vibration
phenomenon happened in many engineering areas, such as offshore structures,
transmission lines, stacks, and bridges. Due to its practical significance in
many engineering structures, VIV has raised many interests and been extensively
investigated in past decades.
\cite{bearman1984vortex},~\cite{sarpkaya2004critical} and
~\cite{williamson2004vortex} have done some detailed reviews of the these work.
One of the most important observations of VIV is called "lock-in" or
"synchronization". "Lock-in" is a regime where shedding vortices induce large
response amplitude of the structures because the oscillating frequency along
with vortex shedding frequency lock onto the natural frequency of the
structures. ~\cite{feng1968measurement} observed two amplitude branches in the
``lock-in'' regime in experiment. Because this experiment was carried in wind
tunnels, the system has a very large mass ratio. The jump between two branches
was suspected due to the change of vortex shedding mode.    However, in recent
years, other different amplitude branches have been found in water tunnel tests
which have much small mass ratio. Four branches have been identified and
labeled as ``initial excitation'', ``upper branch'', ``lower branch'' and
``desynchronization''.(\cite{khalak1997fluid}). 

Past investigations of VIV were based primarily on experiments and a limited
number of computational fluid dynamics (CFD). These studies, spanning over the
past several decades, have resulted in an improved understanding of the
phenomenon of the VIV, particularly when the flow is perpendicular to the
cylinder in wind tunnel.  However, most of these work investigated relatively
large mass and damping system.  In recent years, due to increasing applications
of marine engineering, VIV of low mass and damping system has raised many
interests and been widely studied in experiments. Though a lot of experimental
studies have been done in this decade, only a very few studies were related to
the VIV of a yawed cylinder with low mass and damping.  These experiments
(~\cite{jain2013vortex}, ~\cite{franzini2013one}) were performed to validate
the independence principle on the oscillating frequency, amplitude and force
coefficients of an elastically-mounted cylinder with various yawed angles. The
experiments are conducted in water tunnels with a partially submerged yawed
cylinder. Due to the effects of the end conditions, these experiments observed
different results between the cylinder slanted towards upstream and downstream.
Ideally, the independence principle should be investigated without any end
effect because the vortex shedding of a yawed stationary cylinder and the
response of an elastically-mounted cylinder are both very sensitive to the end
conditions. Compared with experiments, simulations are much easier to achieve
the end-effect-free result.  ~\cite{zhao2015validity} studied the independence
principle of an elastically-mounted cylinder for two Reynolds numbers (Re=150
and Re=1,000) by using direct numerical simulation (DNS) and periodic boundary
conditions.  Independence principle for the response amplitude and frequency is
observed on this numerical study, but the Reynolds numbers of this study are
very small.
	
This paper presents the results of our progress towards the goal of developing
an aeroelastic load model for cylinder vibration. The objective is to be able
to accurately predict aerodynamic loading on a single, smooth circular cylinder
operating in smooth inflow (normal and yawed flow) for static and dynamic
conditions. This paper focuses on verifying the computational methodology
against experimental data for normally-incident flow and yawed flow.

%%%%%%%%%%%%%%%%%%%%%%%%%%%%%%%%%%%%%%%%%%
