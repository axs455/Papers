\documentclass[12pt,preprint]{elsarticle}
\usepackage[margin=1in]{geometry}
\usepackage{wrapfig}
\usepackage{amsmath}
\usepackage{bm}
\usepackage{makecell}
\usepackage[table]{xcolor}

\usepackage{subcaption}
% Font size for subcaption
\DeclareCaptionFont{mysize}{\fontsize{9}{9.6}\selectfont}
\captionsetup[sub]{font=mysize}

%\usepackage[mathlines,pagewise]{lineno}
%%\modulolinenumbers[2]

%\journal{Journal of Sound \& Vibration}

\newcommand{\hl}[1]{\textcolor{red}{#1}}
\newcommand{\hb}[1]{\textcolor{blue}{#1}}
\newcommand{\incfig}{\centering\includegraphics}
\newcommand{\abs}[1]{\left| #1 \right|} % for absolute value
\newcommand{\etal}{\textit{et al}. }
\newcommand{\ie}{\textit{i}.\textit{e}., }
\newcommand{\eg}{\textit{e}.\textit{g}., }
\usepackage{natbib}

\usepackage[colorlinks=true,citecolor=red]{hyperref}

\begin{document}

\begin{frontmatter}

%% Title, authors and addresses

%% use the tnoteref command within \title for footnotes;
%% use the tnotetext command for theassociated footnote;
%% use the fnref command within \author or \address for footnotes;
%% use the fntext command for theassociated footnote;
%% use the corref command within \author for corresponding author footnotes;
%% use the cortext command for theassociated footnote;
%% use the ead command for the email address,
%% and the form \ead[url] for the home page:
%% \title{Title\tnoteref{label1}}
%% \tnotetext[label1]{}
%% \author{Name\corref{cor1}\fnref{label2}}
%% \ead{email address}
%% \ead[url]{home page}
%% \fntext[label2]{}
%% \cortext[cor1]{}
%% \address{Address\fnref{label3}}
%% \fntext[label3]{}

\title{Finite-Span Effect on Vortex-Induced Vibration Simulations}

%% use optional labels to link authors explicitly to addresses:
%% \author[label1,label2]{}
%% \address[label1]{}
%% \address[label2]{}

\author[1]{Xingeng Wu}
\author[2]{Anupam Sharma\corref{a}}
\ead{sharma@iastate.edu}
\cortext[a]{Corresponding author}
\fntext[1]{Graduate Student}
\fntext[2]{Associate Professor, Iowa State University}

\address{Department of Aerospace Engineering, Iowa State University, Ames,
Iowa, 50011}


\begin{abstract}
  The effects of using periodic boundary conditions when simulating vortex
  induced vibration (VIV) on a finite-span model are explored. When spanwise
  coherence is smaller than the simulated span, the peak VIV amplitude is
  under-predicted due to spreading of energy in frequencies around the Strouhal
  number.
\end{abstract}

\begin{keyword}
  Vortex-Induced Vibration \sep 
  Detached Eddy Simulations \sep 
  spanwise coherence
\end{keyword}

\end{frontmatter}

%\linenumbers

%\begin{linenumbers}
%% main text
%%%%%%%%%%%%%%%%%%%%%%%%
\section{Introduction}
\label{sec:intro}
%%%%%%%%%%%%%%%%%%%%%%%%
%
Vortex-induced vibrations (VIV) are commonly observed in bridge decks,
cables, power conductors, risers in oil rigs, etc. The K\'arm\'an vortex
shedding in the wake of the cylinder produces periodic forcing on the
cylinder. In certain conditions the vortex-shedding frequency
synchronizes (``lock-in'') with the natural frequency of the system
which results in high-amplitude oscillations (VIV) limited by the
system damping. Experimental investigations of VIV are typically
performed with finite-span cylinders with aspect ratios greater than
$10$ to minimize ``end effects'' (*cite*), where end-effect refers to
the three-dimensional flow that results due to the finiteness of the
model. End plates have been used in experiments (*cite*), which reduce
but not completely eliminate the end effects as horse-shoe vortices
develop at the intersection of the cable and the end plates. 

This problem is avoided in simulations by using periodic boundaries in the span
direction. While the periodic boundaries imply an infinitely-long cylinder, the
finite size of the computational domain in the span direction imposes
artificial periodicity in the flow. If spanwise variations are present in the
flow and the length scale of these variations is larger than the simulated
span, then span periodicity will likely yield incorrect results. For stochastic
(turbulent) flows, this length scale can be measured using two-point
correlations; in particular, spanwise coherence. Magnitude-squared coherence is
defined as 
%
$\gamma^2(\Delta z,f) = \langle \abs{S_{xy}(f)}^2 \rangle/(\langle
    S_{xx}(f)\rangle \langle S_{yy}(f) \rangle)$,
%
where $S_{xy}(f)$ denotes cross-spectral density of the desired quantity
(e.g., sectional lift) at points $\bm{x}$ and $\bm{y}$ separated by a distance
$\Delta z$ (along the span). $S_{xx}(f)$ and $S_{yy}(f)$ are
auto-spectral densities at $\bm{x}$ and $\bm{y}$ respectively and angular
brackets denote ensemble averaging.

Figure~\ref{fig:Coherence_staic} presents contours of $\gamma^2(\Delta
z,\omega)$ of sectional lift for a static circular cylinder (aspect ratio,
$L/D=20$) simulation. Nondimensional frequency, $k = f\,D/V_\infty$ is used to
plot coherence. Spanwise coherence is small everywhere except at the K\'arm\'an
vortex-shedding frequency ($f_v$), given by the Strouhal number, $St = f_v
D/V_\infty$ of around $0.2$.
%
\begin{wrapfigure}{R}{0.5\textwidth}
  \vspace*{-0.2in}
  \incfig[width=.5\textwidth]{Figures/Cl_Coherence_Z20_Static.png}
  \caption{$\gamma^2(\Delta z,f)$ of lift for a static cylinder}
  \label{fig:Coherence_staic}
\end{wrapfigure}

In this paper, we investigate the effect of span periodicity when simulating a
circular cylinder experiencing VIV. While the effect of aspect ratio (span
length to diameter ratio, $L/D$) have been investigated in VIV experiments
(*cite*), such an investigation is lacking for simulations. (* provide a very
brief summary of study on aspect-ratio-effects for flow over a cylinder *) We
present detached eddy simualtion (DES) results of four models with
$L/D=1,~2,~5,$ and $10$. Displacement amplitude, lift spectra and spanwise
coherence are analyzed to study the effect of span periodicity.


%%%%%%%%%%%%%%%%%%%%%%%%%%%%%%%%%%%%%%%%%%
\section{Computational Methodology and Verification} 
\label{sec:methodology}
%%%%%%%%%%%%%%%%%%%%%%%%%%%%%%%%%%%%%%%%%%
%
A coupled fluid-solid dynamics solver is used to simulate an
elastically-mounted rigid circular cylinder experiencing VIV. The detached eddy
simulation (DES) technique is used to model the flow and a forced single-degree
of freedom mass-spring-damper system is solved to model the dynamics of the
cylinder. The $k$-$\omega$ delayed detached eddy simulation (DDES) formulation
by Yin \etal \cite{yin2015dynamic} is used here for the flow simulation. The
details of the approach are described in Wu \etal \citep{wu2019}.

Figure~\ref{fig:VIV_verification} (a) shows a schematic of the simulation setup
and a comparison with measured data for a low mass-damping cylinder undergoing
VIV.  The following nondimensional numbers are matched between the experiments
and the simulations: mass ratio, $m^*=2.6$, mechanical damping ratio,
$\zeta=0.001$, and reduced velocity, $V_R = V_\infty/(f_N D)$, where $f_N$ is
the natural frequency of the mass-spring-damper system. As seen in
Fig.~\ref{fig:VIV_verification} (b\&c), the simulations accurately predict the
displacement amplitude and the vortex shedding frequency of the cylinder over a
wide range of $V_R$ which includes the four branches identified in the
experiment~\cite{khalak1997fluid}: {\em initial excitation}, {\em upper}, {\em
lower}, and {\em desynchronization}. Of particular interest is the ``lock-in''
phenomenon which occurs in the {\em upper} and {\em lower} branches. 

\begin{figure}[htb!]
  \subcaptionbox{simulation setup}{\incfig[width=.32\textwidth]{Figures/VIV_setup}} \;
  \subcaptionbox{mean amplitude}  {\incfig[width=.32\textwidth]{Figures/validation/viv_amp_noyaw.pdf}} \;
  \subcaptionbox{frequency}       {\incfig[width=.32\textwidth]{Figures/validation/viv_freq_noyaw.pdf}}
  \caption{Verification of the DES approach to predict VIV: (a) a schematic
    showing the simulation setup, (b) mean nondimensional displacement
    amplitude ($\bar{A}/D$), and (c) vortex shedding frequency ($f_v$)
    normalized by natural frequency of the system ($f_N$). Simulation results
    are compared against measurements from Ref.~\cite{khalak1997fluid} for a
    range of reduced velocity ($V_{R}$).}
  \label{fig:VIV_verification}
\end{figure}


%%%%%%%%%%%%%%%%%%%%%%%%%%%%%%%%%%%%%%%%%%
\section{Effect of Aspect Ratio}
\label{sec:periodicity}
%%%%%%%%%%%%%%%%%%%%%%%%%%%%%%%%%%%%%%%%%%
%
Simulations are performed for four values of aspect ratio, $L/D$ ($= 1, 2, 5,$
and $10$). For each configuration, seven values of reduced velocity $V_{R}$
($=2, 3, 4, 5, 5.9, 7, 8$) are evaluated; this range covers the {\em initial
excitation}, {\em upper}, and {\em lower} branches observed during VIV of a low
mass-damping cylinder.


\begin{figure}[htb!]
  \incfig[width=.5\textwidth]{Figures/A.png}
  \caption{Comparison of predicted non-dimensional mean amplitude ${\bar A}/D$
    using cylinder models of different span lengths.}
  \label{fig:Amplitude_Compared}
\end{figure}

Figure~\ref{fig:Coherence_Cl} and \ref{fig:Coherence_Cd} present force
coefficients coherence for different spans cylinders ($Z = 1D, 2D, 5D,$ and
$10D$). For longer cylinders ($Z = 5D$ and $10D$) are only correlated to the
peak vortex shedding frequency and its harmonics. However, the shorter
cylinders ($Z = 1D$ and $2D$) have more wider coherence for more frequencies??

\begin{figure}[htb!]
  \subcaptionbox{$1D$} {\incfig[height=.33\textwidth]{Figures/Coherence_Z1D.png}} \qquad
  \subcaptionbox{$2D$} {\incfig[height=.33\textwidth]{Figures/Coherence_Z2D.png}} \qquad
  \subcaptionbox{$5D$} {\incfig[height=.33\textwidth]{Figures/Coherence_Z5D.png}} \qquad
  \subcaptionbox{$10D$}{\incfig[height=.33\textwidth]{Figures/Coherence_Z10D.png}}
    \caption{Lift coherence for various aspect ratio at $V_{R,n}=5$}
  \label{fig:Coherence_Cl}
\end{figure}

\begin{figure}[htb!]
  \subcaptionbox{$1D$} {\incfig[height=.33\textwidth]{Figures/Coherence_Z1D_Cd.png}} \qquad
  \subcaptionbox{$2D$} {\incfig[height=.33\textwidth]{Figures/Coherence_Z2D_Cd.png}} \qquad
  \subcaptionbox{$5D$} {\incfig[height=.33\textwidth]{Figures/Coherence_Z5D_Cd.png}} \qquad
  \subcaptionbox{$10D$}{\incfig[height=.33\textwidth]{Figures/Coherence_Z10D_Cd.png}}
  \caption{Drag coherence for various aspect ratio at $V_{R,n}=5$}
  \label{fig:Coherence_Cd}
\end{figure}

Figure~\ref{fig:force_VIV} shows ... different peak value, wider, lower psd
smaller $L/D$. 

\begin{figure}[htb!]
  \subcaptionbox {Power spectral densities of $C_l$}
    [.48\linewidth]{\incfig[width=.48\textwidth]{Figures/St_f.png}}
  \hspace*{\fill}
  \subcaptionbox{Zoom-in }
    [.48\linewidth]{\incfig[width=.48\textwidth]{Figures/St_f_2.png}}
    \caption{Comparison of predicted power spectral densities
      (PSDs) of $C_l$ for various frequencies at $V_{R,n}=5$.}
  \label{fig:force_VIV}
\end{figure}

\clearpage

\begin{figure}[htb!]
  \subcaptionbox{Spectra}
    {\incfig[width=.44\textwidth]{Figures/RV3/St_1_RV.pdf}}
  \qquad
  \subcaptionbox{Coherence}
    {\incfig[width=.44\textwidth]{Figures/RV3/Coherence_Cl_Z10.pdf}} \\
  \subcaptionbox{Coherence, $L/D=5$}
    {\incfig[width=.32\textwidth]{Figures/RV3/Coherence_Cl_Z5.pdf}}
  \;
  \subcaptionbox{Coherence, $L/D=2$}
    {\incfig[width=.32\textwidth]{Figures/RV3/Coherence_Cl_Z2.pdf}}
  \;
  \subcaptionbox{Coherence, $L/D=1$}
    {\incfig[width=.32\textwidth]{Figures/RV3/Coherence_Cl_Z1.pdf}}
    \caption{$V_{R,n}=3$}
  \label{fig:RV_3}
\end{figure}

\begin{figure}[htb!]
  \subcaptionbox{Spectra}
    [.48\linewidth]{\incfig[width=.44\textwidth]{Figures/RV5_9/St_1_RV.pdf}}
  \hspace*{\fill}
  \subcaptionbox{Coherence, $L/D=10$}
    [.48\linewidth]{\incfig[width=.44\textwidth]{Figures/RV5_9/Coherence_Cl_Z10.pdf}} \\
  \subcaptionbox{Coherence, $L/D=5$}
    [.32\linewidth]{\incfig[width=.32\textwidth]{Figures/RV5_9/Coherence_Cl_Z5.pdf}}
  \hspace*{\fill}
  \subcaptionbox{Coherence, $L/D=2$}
    [.32\linewidth]{\incfig[width=.32\textwidth]{Figures/RV5_9/Coherence_Cl_Z2.pdf}}
  \hspace*{\fill}
  \subcaptionbox{Coherence, $L/D=1$}
    [.32\linewidth]{\incfig[width=.32\textwidth]{Figures/RV5_9/Coherence_Cl_Z1.pdf}}
    \caption{$V_{R,n}=5.9$}
  \label{fig:RV_5p9}
\end{figure}

%%%%%%%%%%%%%%%%%%%%%%%%%%%%%%%%%%
\section{Conclusion}
\label{sec:conclusions}
%%%%%%%%%%%%%%%%%%%%%%%%%%%%%%%%%%%%



%% The Appendices part is started with the command \appendix;
%% appendix sections are then done as normal sections
%% \appendix

%% \section{}
%% \label{}

%% If you have bibdatabase file and want bibtex to generate the
%% bibitems, please use
%%
%\end{linenumbers}



%\bibliographystyle{elsarticle-num} 
\bibliography{references}
\bibliographystyle{model3a-num-names}

%% else use the following coding to input the bibitems directly in the
%% TeX file.
%\begin{thebibliography}{00}
%% \bibitem[Author(year)]{label}
%% Text of bibliographic item
%\bibitem[ ()]{}
%\end{thebibliography}

\end{document}
\endinput
