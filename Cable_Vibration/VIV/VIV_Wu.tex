\documentclass[12pt,authoryear]{elsarticle}
\usepackage{graphicx}
\usepackage{subcaption}
\usepackage{makecell}
\usepackage[table]{xcolor}
\usepackage[margin=1in]{geometry}
\usepackage{amsmath}
\usepackage{amssymb}
\usepackage{bm}
\linespread{1.5} %line spacing

\newcommand{\incfig}{\centering\includegraphics}
\newcommand{\abs}[1]{\left| #1 \right|} % for absolute value
% Font size for subcaption
\DeclareCaptionFont{mysize}{\fontsize{9}{9.6}\selectfont}
\captionsetup[sub]{font=mysize}
% For editing
\newcommand{\hl}[1]{\textcolor{red}{#1}}
\newcommand{\hb}[1]{\textcolor{blue}{#1}}

%\captionsetup{compatibility=false}
%\journal{Journal of Fluids and Structures}


\begin{document}

\begin{frontmatter}

\title{Numerical Study of Static and Vortex-Induced Vibrations of Normal and
Yawed Cylinder with Detached Eddy Simulations}

%% use optional labels to link authors explicitly to addresses:
%% \author[label1,label2]{}
%% \address[label1]{}
%% \address[label2]{}

\author[1]{Xingeng Wu}
\author[1]{Mohammad Jafari} 
\author[2]{Partha Sarkar}
\author[3]{Anupam Sharma\corref{a}}
\ead{sharma@iastate.edu}
\cortext[a]{Corresponding author}
\fntext[1]{Graduate Student}
\fntext[2]{Professor, Iowa State University}
\fntext[3]{Associate Professor, Iowa State University}

\address{Department of Aerospace Engineering, Iowa State University, Ames, Iowa, 50011}

\begin{abstract}
A computational approach based on a $k-\omega$ delayed detached eddy simulation
model for predicting aerodynamic loads on a smooth circular cylinder is
verified against experiments. Comparisons with experiments are performed for
flow over a stationary cylinder and for a cylinder oscillating in the
transverse direction due to flow-induced forces (vortex-induced vibration or
VIV). For the static cases, measurement data from the literature is used to
validate the predictions for normally-incident flow. New experiments are
conducted as a part of this study for yawed flow, where the cylinder axis is
inclined with respect to the inflow velocity at the desired yaw angle,
$\beta=30^\circ$. Good agreement is observed between the predictions and
measurements for mean and rms surface pressure and wake velocity deficit.
However the lift spectrum prediction shows a slight offset in frequency from
the measurements. Three yawed flow cases ($\beta=15^\circ, 30^\circ, \,\&\,
45^\circ$) are simulated and the results are found to be independent of $\beta$
(dynamic scaling) when the flow speed normal to the cylinder axis is selected
as the reference velocity scale.  

Dynamic (VIV) simulations are performed by coupling the flow solver with a
solid dynamics solver where the cylinder motion in the transverse direction is
modeled as a spring-mass-damper system. The simulations accurately predict the
displacement amplitude and unsteady loading over a wide range of reduced
velocity, including the region where lock-in occurs. VIV simulations are
performed at two yaw angles, $\beta=0^\circ$ and $45^\circ$ and the
appropriately-scaled results are found to be similar over the range of reduced
velocities tested with slightly higher discrepancy around the reduced velocity
corresponding to the natural frequency of the system.
\end{abstract}

\begin{keyword}
  Vortex-Induced Vibrations \sep Detached Eddy Simulations \sep Yawed Cylinder \sep Independence Principle
\end{keyword}
\end{frontmatter}

%\modulolinenumbers[2] %\linenumbers
%\begin{linenumbers}

%% main text
%%%%%%%%%%%%%%%%%%%%%%%%
%%%%%%%%%%%%%%%%%%%%%%%%
\section{Introduction}
\label{sec:intro}
%%%%%%%%%%%%%%%%%%%%%%%%
%
Inclined and horizontally- or vertically-spanned cylinder are used in various
engineering applications: cable-stayed, suspension, and tied-arch bridges,
power transmission lines, offshore risers, sub-sea pipelines etc. These
cylinders are prone to large-amplitude flow-induced vibration, which can lead
to catastrophic failure of the cylinders and the structures supported by them.
The vibration mechanisms involve complex aeroelastic (motion-induced)
interactions that depend on the spatial orientation, geometry,
surface-characteristics, and dynamic properties of
cylinders. \cite{davenport1995dynamics} shows that large amplitude vibrations
can lead to either catastrophic- or fatigue failure of the cylinders and/or the
adjoining structures, which poses a significant threat to the safety and
serviceability of these systems.

Wind induced vibration can be classified into the following categories:
K\'arm\'an vortex-induced vibration (VIV), rain-wind induced vibration (RWIV),
wake-induced vibration, and dry- or iced-cable galloping. While VIV and RWIV
occur at relatively low wind speeds ($<20$ m/s), wake-induced vibration and or
dry-cable galloping occur at higher speeds. Simulation of any of these
aeroelastic phenomena requires accurate computation of the aerodynamic loads
acting on the cable.  This paper presents a comprehensive verification of a
high-fidelity flow simulation technique (detached eddy simulations or DES) for
flow over rigidly and elastically-mounted rigid circular cylinders in normal
and yawed flow. Yaw angle ($\beta$) is defined as the angle between the
cylinder axis and a vector orthogonal to the flow velocity vector in the plane
of the cylinder axis.  Yawed flow is considered because RWIV and dry-cable
galloping occur only when $\beta \ne 0$. The diameter-based Reynolds number in
the simulations is $Re_D=2\times 10^4$. At this $Re_D$, the boundary layer is
laminar when separation occurs and transition to turbulence takes place in the
free shear layer. The results of this paper are presented in two parts - the
first part deals with rigidly-mounted (static) cylinders and the second part
addresses elastically-mounted cylinders.

In the first part of the paper, simulations of flow over a static cylinder are
verified with measurements from the literature as well as new experiments that
are performed for $\beta=0^\circ$ and $30^\circ$ as a part of this study;
Simulations are performed for four values of $\beta$
($=0^\circ,~15^\circ,~30^\circ,$ and $45^\circ$) to investigate the effect of
yaw and to examine the validity of the {\em independence principle} (IP) which
states that the aerodynamic loads scaled using the velocity component normal to
the cylinder axis are independent of $\beta$ (\citet{zdravkovich2003flow}).
Prior research (see e.g., \citet{franzini2009experimental} and
\citet{zhao2009direct}) has suggested the validity of IP for $\beta$ up to
$45^\circ$. Comparisons between the simulations and measurements include mean
and rms of surface pressure, mean wake velocity profile, and power spectra of
lift and drag. The agreement with the measured data ranges from very good to
acceptable. The results of yawed flow simulations confirm the validity of IP
for the static configurations analyzed.

Vortex-induced vibrations are self-excited oscillations that occur due to the
unsteady lift generated on a cylinder as a result of the vortex shedding in its
wake. VIV is a very common occurrence in many engineering applications, e.g.,
offshore structures, transmission lines, stacks, and bridges. Of particular
interest is the phenomenon of ``lock-in'' or ``synchronization'' which occurs
when the vortex-shedding frequency gets locked in with the natural frequency of
the system. High but finite-amplitude (limit-cycle) oscillations are observed
in this resonance condition due to nonlinearity in the system. These
high-amplitude oscillations can cause significant damage to the structure. VIV
has been investigated extensively in the past few decades due to its practical
significance. \cite{bearman1984vortex},~\cite{sarpkaya2004critical} and
\cite{williamson2004vortex} present detailed reviews of the research on this
topic.

Investigations of VIV over the past several decades, have resulted in an
improved understanding of the phenomenon. Until recently, most of these
investigations considered relatively high mass and damping systems. In the last
two decades, VIV of low mass and damping systems has garnered considerable
interest due to the applicability of the phenomenon in marine engineering and
offshore wind turbines (mooring lines). Such low mass-damping systems have been
largely investigated using experiments (e.g., \citet{jain2013vortex} and
\citet{franzini2013one}) and semi-analytical methods that model vortex shedding
using the classical Van der Pol oscillator (e.g., ~\cite{xu2008high}). 

Experiments of VIV in low mass-damping systems have been typically conducted in
water channels with partially or fully-submerged cylinders. Such setup can
suffer from finite-span effects due to different end conditions on the two ends
-- wall on one side and free surface on the other. These effects are distinctly
observable in yawed flow measurements where the results are found to be
dependent on the direction with respect to the flow the cylinder is yawed
(i.e., $\beta=45^\circ$ versus $-45^\circ$). Simulations do not suffer from end
effects as periodic boundary conditions enable simulating an infinitely long
cylinder. 

Numerical computations of VIV of low mass-damping systems have been performed
using the Reynolds Averaged Navier-Stokes (RANS) equations (e.g.,
\citet{guilmineau2004numerical}), large eddy simulations or LES (e.g.,
\citet{al2004vortex}), and direct numerical simulations or DNS (e.g.,
\citet{lucor2005vortex}, \citet{evangelinos2000dns}, and \citet{dong2005dns}).
While the DNS approch is attractive from the perspective of resolving the
entire turbulence spectrum, it is only feasible for very low Reynolds number
($Re$) due to the associated computation cost. Large Eddy Simulation (LES) is
more feasible at moderate $Re$ while Detached Eddy Simulations (DES) can be
used over a very large range of $Re$ without making the computations
prohibitively expensive. 

In the second part of this paper, the phenomenon of VIV is investigated with
elastically-mounted cylinder simulations and the results are verified against
measured data available in the literature. The VIV simulations are also
performed at multiple values of $\beta$ ($=0^\circ$ and $45^\circ$) to
investigate the validity of the independence principle for VIV of low
mass-damping systems.

With the detailed comparisons with measurements for rigid and
elastically-mounted oscillating cylinder presented here, this paper can serve
asa benchmark for the accurancy of the DES technique in predicting aerodynamic
loads on cables in static condition and in dynamic motion.
%%%%%%%%%%%%%%%%%%%%%%%%%%%%%%%%%%%%%%%%%%

%%%%%%%%%%%%%%%%%%%%%%%%%%%%%%%%%%%%%%%%%%
\section{Computational Methodology}
\label{sec:methodology}
%%%%%%%%%%%%%%%%%%%%%%%%%%%%%%%%%%%%%%%%%%
%
The flow is approximated to be incompressible since the flow Mach number is
less than $0.2$. Various degrees of approximations can be utilized to model
flow turbulence: from resolving only time-averaged quantities in Reynolds
Averaged Navier-Stokes or RANS, to resolving the tiniest of turbulent eddies in
Direct Numerical Simulations or DNS. Large eddy simulations (LES) resolve
energy containing eddies but model the net effect of smaller
(unresolved/universal) eddies on larger (resolved) eddies. The detached eddy
simulation (DES) technique~\citep{spalart1997comments} is a hybrid approach
that uses RANS equations to simulate attached flow near solid surfaces and LES
for separated (detached) flow away from them. DES allows computation of high
Reynolds number flows relatively inexpensively by removing the constraint in
LES to have very fine grids near solid boundaries.

Flow over stationary slender structures with circular cross-section has been
studied using unsteady RANS~\citep{pontaza2009three},
DES~\citep{travin2000detached,yeo2011computational,yeo2012aerodynamic},
LES~\citep{breuer1998large,kravchenko2000numerical,catalano2003numerical}, and
DNS~\citep{dong2005dns,zhao2009direct} approaches. Detailed flow simulations
have been performed using DES for a single, stationary, yawed cylinder in
uniform inflow~\citep{yeo2008investigation} and oscillating
inflow~\citep{yeo2012aerodynamic}. 
%DES has also been used to investigate the use of strakes in cables for
%aerodynamic mitigation of wind-induced oscillations
%by~\cite{yeo2011computational}. 
High-fidelity simulations using DNS, LES, and DES have been instrumental in
gaining insights into the problem of flow-induced vibration.

In LES and DES, the equations are spatially filtered (low-pass) and the
numerical procedure solves for the filtered quantities that can be resolved by
the grid. The sub-filter (or sub-grid) quantities exert a ``stress'' on the
filtered flow-field, which is modeled using a sub-grid scale (SGS) stress
model. Denoting spatially filtered quantities by overhead tilde ($^\sim$), the
incompressible flow equations with an eddy-viscosity turbulence model are
%
\begin{align}
  \frac{\partial{\tilde{U_i}}}{\partial{x_i}}&=0,~{\rm and} \nonumber \\
   \frac{\partial{\tilde{U_i}}}{\partial{t}}+
   \frac{\partial{(\tilde{U_j}\tilde{U_i})}}{\partial{x_j}}
   &=-\frac{1}{\rho}\frac{\partial{\tilde{p}}}{\partial{x_i}}+\nu\frac{\partial{^2\tilde{U_i}}}{\partial{x_j^2}}-\frac{\partial{\tau_{ij}}}{\partial{x_j}},
 \label{eq:geqs}
\end{align}
%
where $\tau_{ij} = \widetilde{U_i U_j}-\tilde{U_i}\tilde{U_j} = -2 \, \nu_{SGS}
\, \tilde{S}$ and $\tilde{S} = (\partial{\tilde{U_i}}/\partial{x_j} +
\partial{\tilde{U_j}}/\partial{x_i})/2$. In the above, SGS denotes a sub-grid
scale quantity, $\tau_{ij}^{SGS}$ denotes the sub-grid scale stress tensor
which is modeled as the product of the eddy viscosity, $\nu_{SGS}$ and the
strain rate tensor $S_{ij}$; turbulence models of such type are referred to as
eddy-viscosity models. DES is a non-zonal hybrid RANS-LES method, where a RANS
turbulence model is used to compute the eddy viscosity for the SGS stress
tensor in the corresponding LES. In the original DES formulation
(\cite{spalart1997comments}), the Spalart-Allmaras (SA) LES and SA-RANS models
were used. We use the method developed by~\cite{yin2015dynamic}, which
introduces a dynamic procedure to improve the DES capability by correcting for
modeled stress depletion and log-layer mismatch. This improved model is
referred to as delayed detached eddy simulation or DDES. The model has been
implemented in the open source CFD software OpenFOAM. All the simulations
results in this paper are obtained using OpenFOAM. The numerical scheme uses
second order backward difference for time integration and linear interpolation
with central differencing for spatial discretization of the governing equations.

%%%%%%%%%%%%%%%%%%%%%%%%%%%%%%%%%%%%%
\subsection{Delayed Detached Eddy Simulation (DDES) Model}
\label{sec:DDES}
%%%%%%%%%%%%%%%%%%%%%%%%%%%%%%%%%%%%%
%
A summary of the DDES model used in this study is provided here; details are
available in~\cite{yin2015dynamic}. It uses a $k$-$\omega$ turbulence closure
model in the RANS zones, and the same model is used to calculate $\nu_T$ in the
LES zones. The eddy viscosity in the $k$-$\omega$ DDES can be defined as
$\nu_T=l_{DDES}^2\, \omega$, where $l_{DDES}$ is the DDES length scale. The
different length scales in the $k$-$\omega$ DDES model are defined as
%
\begin{align}  
  l_{DDES} &=l_{RANS}-f_d\, \max( 0,~l_{RANS}-l_{LES}), \nonumber \\
  l_{RANS} &=\sqrt{k}/\omega, \\
  l_{LES} &=C_{DES}\bigtriangleup. \nonumber
  \label{eq:ddes_lscale}
\end{align}
%
In the above, $l_{RANS}$ and $l_{LES}$ are the length scales of the RANS and LES
branches respectively and $\bigtriangleup = f_d \,V^{1/3}+(1-f_d ) \,h_{max}$,
where $h_{max} = \max(dx,dy,dz)$ is the maximum grid size, and $f_d$ is a
shielding function of the DDES model, defined as $f_d = 1 -
\tanh\{(8\,r_d)^3\}$, with
\[
  r_d=\frac{k/\omega+\nu}{\kappa^2 \, d_w^2 \, \sqrt{U_{i,j} U_{i,j} }},
\]
$\nu$ is the molecular viscosity, $\kappa$ is the von K\'arm\'an constant, $d_w$ is
the distance between the cell and the nearest wall, and $U_{i,j}=\partial_j
U_i$ is the velocity gradient.  In the RANS branch, the transport equation for
k and $\omega$ are written as
%
\begin{align}  
  \frac{Dk}{Dt} & = 2\nu_T |S|^2-C_\mu k\omega+\partial{_j [(\nu+\sigma_k \nu_T ) \partial{_j k}]}, \nonumber \\
  \frac{D\omega}{Dt} & = 2C_{\omega1} |S|^2-C_{\omega2} \omega^2 
                    +\partial{_j [(\nu+\sigma_\omega \nu_T )\partial{_j \omega}]},\\
    &~~~~~~~\rm{where}~\nu_T=k^2/\omega. \nonumber
    \label{eq:transportEqs}
\end{align}  
%
In the LES region ($f_d=1,l_{DDES}=C_{DES} \, \bigtriangleup$), the eddy
viscosity switches to
$\nu_T=l_{DDES}^2\,\omega=(C_{DES}\bigtriangleup)^2\omega$, which is similar to
the eddy viscosity in the Smagorinsky model, $\nu_s=(C_s \bigtriangleup)^2
|S|$.

The LES branch of this $k$-$\omega$ DDES model uses a dynamic procedure which
defines the value of $C_{DES}$ as
%
\begin{align}
  C_{DES}   & =  \max( C_{lm},~C_{dyn} ), \nonumber \\
  C_{dyn}^2 & = \max \left( 0,\; 0.5 \frac{L_{i,j} M_{i,j}}{M_{i,j} M_{i,j}}\right), \nonumber \\
  C_{lim}   &=C_{DES}^0 \left[1-\tanh \left(\alpha \exp \left(\frac{-\beta \, h_{max}}{L_k}\right)\right)\right],\\
  C_{DES}^0 &=0.12, \quad   L_k=\left(\frac{\nu^3}{\epsilon}\right)^{\frac{1}{4}},  \quad  \alpha=25,  \quad  \beta=0.05, \nonumber \\
  \epsilon  &= 2 \left(C_{DES}^0 h_{max} \right)^2 \omega\,|S|^2+C_\mu k\,\omega. \nonumber
\end{align}  
%
For further details about the DES model, the reader is referred
to~\cite{yin2015dynamic}.

%%%%%%%%%%%%%%%%%%%%%%%%%%%%%%%%%%%%%%%%%%
\subsection{Solid Body Dynamics and Coupling}
\label{sec:coupling}
%%%%%%%%%%%%%%%%%%%%%%%%%%%%%%%%%%%%%%%%%%
%
Simulations of VIV are performed with the pimpleDyMFoam solver which uses the
{\em sixDoFRigidBodyMotion} feature of OpenFOAM. Single degree of freedom (dof)
motion of the cylinder is considered here with displacement permitted only in
the $y-$direction (orthogonal to the freestream flow direction). At each time
step, pimpleDyMFoam calculates the motion and updates the displacement of the
cylinder by integrating the following equation in time using the Crank-Nicolson
method.
%
\begin{equation}
  m\,\ddot{y} + c\,\dot{y} + k_s \,y = F_{fluid},
  \label{eq:solidBodyDynamics}
\end{equation}
%
where $m$ is the mass of the rigid cylinder, $\ddot{y}$, $\dot{y}$, and $y$ are the
instantaneous acceleration, velocity and displacement of the cylinder,
respectively, $c$ is the damping of the system, $k_s$ is the spring stiffness and
$F_{fluid}$ is the transverse (cross-stream) component of the fluid force
acting on the cylinder surface.

In order to maintain the quality of the meshes inside the boundary layer in
dynamic simulations, the mesh around the cylinder (up to 1 diameter from the
cylinder axis) moves with the cylinder without deforming. The mesh away from
the cylinder is deformed based on the position of the cylinder. The solver then
updates the fluxes according to the motion and uses the PIMPLE method to solve
the incompressible Navier-Stokes equations.

%%%%%%%%%%%%%%%%%%%%%%%%%%%%%%%%%%%%%%%%%%
\subsection{Computational Grid and Mesh Sensitivity Study}
\label{sec:grids}
%%%%%%%%%%%%%%%%%%%%%%%%%%%%%%%%%%%%%%%%%%
%
The outer boundary of the computational domain is circular with a radius of
$25\times D$, where $D$ is the diameter of the cylinder. The cylinder is placed
in the center of the domain and the span dimension is $L=10\times D$ for all
the simulations presented here. Periodic boundary conditions are used in the
span direction, while freestream condition is used on the outer radial
boundary. The domain is discretized using a multi-block grid that has three
blocks: (1) an O-grid is used to resolve the flow around the cylinder, (2) an
H-grid to resolve the wake, and (3) a C-grid for the far field. In order to
accurately capture the aerodynamic forces on the cylinder, the flow around the
cylinder and in the near-wake region has to be resolved with high precision. A
fine mesh is therefore applied in these regions. Figure~\ref{fig:Mesh} shows a
cross-sectional view of the full computational domain as well as a zoom-view to
highlight the grid topology.
%
\begin{figure}[htb!]
  \centering
  \subcaptionbox{Overall CFD domain}%
    [.48\linewidth]{\incfig[width=.48\textwidth]{Figures/Mesh1.png}}
  \hspace*{\fill}
  \subcaptionbox{Zoom view of near-cylinder mesh}%
    [.48\linewidth]{\incfig[width=.48\textwidth]{Figures/Mesh2.png}}
  \caption{Cross-sectional views of the computational grid}
  \label{fig:Mesh}
\end{figure}

A mesh sensitivity study is performed for normally-incident flow.
Table~\ref{tab:meshSize} provides a summary of the three meshes used for this
study. These are labeled `Mesh 1', `Mesh 2', and `Mesh 3'.  Cell counts,
Strouhal number ($St$), mean drag coefficient ($\overline{C}_d$), and mean back
pressure coefficient ($\overline{C}_{p,b}$) are also listed in the table.
Figure~\ref{fig:Cp_compare_LS_Mesh} compares the predicted mean aerodynamic
pressure coefficient, $\overline{C}_p=2 (\overline{p}-p_\infty)/(\rho
V_\infty^2)$ and the r.m.s. of pressure coeff., $C_{p'rms}$ for the different
grids. Mesh 2 and Mesh 3 show consistent distributions of $\overline{C}_p$ and
$C_{p'rms}$ and predict the same location for boundary layer separation
($\theta \sim 80^\circ$); separation location is delayed with Mesh 1.  Based on
these results, Mesh 3 is chosen for the subsequent simulations.
%
\begin{table}[htb!]
  \caption{Summary of the test cases simulated to investigate sensitivity of
    results to mesh size. $N_\theta$, $N_r$, and $N_z$ denote number of cells in 
    circumferential, radial, and span directions respectively. Also tabulated 
    are $St$, $\overline{C}_d$, and $\overline{C}_{p,b}$.}
  \label{tab:meshSize} 
  \begin{center}
    \begin{tabular}{c|c|c|c|c|c}
      \textbf{Mesh name} &  \textbf{Cell counts} ${(N_{\theta} \times N_r \times N_z)}$ & \textbf{Total cell count} & 
      \boldsymbol{$St$} & \boldsymbol{$\overline{C_d}$} & \boldsymbol{$\overline{C_{p,b}}$} \\ \hline
      \hline
      Mesh 1 & $157 \times 233 \times 65$  & 2.37 M & 0.201 & 1.065 & -0.983 \\ \hline
      Mesh 2 & $188 \times 275 \times 80$  & 4.14 M & 0.208 & 1.125 & -1.132 \\ \hline
      Mesh 3 & $236 \times 343 \times 100$ & 8.09 M & 0.208 & 1.137 & -1.148 \\
      \hline \hline
    \end{tabular}
  \end{center}
\end{table}

%
\begin{figure}[htb!]
  \centering
  \subcaptionbox{mean pressure coeff., $\overline{C}_p$}
    [.48\linewidth]{\incfig[width=.48\textwidth]{Figures/Cp_Mesh_Re20k_no_exp.pdf}}
  \hspace*{\fill}
  \subcaptionbox{r.m.s. of pressure coeff., $C_{p'rms}$ }
    [.48\linewidth]{\incfig[width=.48\textwidth]{Figures/CpRMS_Mesh_Re20_k_no_exp.pdf}}
  \caption{Results of mesh refinement study}
  \label{fig:Cp_compare_LS_Mesh}
\end{figure}
%%%%%%%%%%%%%%%%%%%%%%%%%%%%%%%%%%%%%%%%%%%%%%%%%%%%%%%

\section{Experimental Setup and Measurements}
\label{sec:experiments}
%%%%%%%%%%%%%%%%%%%%%%%%%%%%%%%%%%%%%%%%%%%%%%%%%%%%%%%
%
Static wind tunnel experiments were conducted on a smooth cylinder of circular
cross section representing a stay cable to measure the aerodynamic forces and
the velocity distribution in its wake. These experiments were performed in the
Aerodynamic/Atmospheric Boundary Layer (AABL) Wind and Gust Tunnel located in
the Department of Aerospace Engineering at Iowa State University. This wind
tunnel has an aerodynamic test section of $2.44$ m ($8.0$ ft) width $\times$
$1.83$ m ($6.0$ ft) height and a design maximum wind speed of $53$ m/s ($173.9$
ft/s). A polished aluminum tube with diameter, $D=0.127$ m and length, $L=1.52$
m was selected as the smooth cylinder model. Although the aspect ratio
($L/D=12$) is sufficiently large to minimize edge effects at the mid-span of
the circular cylinder, two circular plates of diameter $4\,D$ were attached to
the ends of the cylinder. These plates were adjusted for each cylinder yaw
angle to be parallel to the incoming airflow so that nearly 2D flow could be
achieved over the cylinder. The maximum blockage ratio in the tunnel with the
model was approximately 5\% for all measurements. Figure~\ref{fig:ExpSetup}
displays the model setup in the AABL tunnel with the cylinder in normal-flow
and yawed-flow configurations. An innovative multi-functional static setup was
designed to measure the pressure distributions and velocity profiles for
different yaw angles. As shown in Fig.~\ref{fig:ExpSetup}, this setup properly
secures the model for different yaw angles.
%
\begin{figure}[htb!]
  \centering
  \subcaptionbox{setup for normal-flow measurements}
    [.48\linewidth]{\incfig[width=.48\textwidth]{Figures/Experiment_NormalFlow.png}}
  \hspace*{\fill}
  \subcaptionbox{setup for yawed flow measurements}
    [.48\linewidth]{\incfig[width=.48\textwidth]{Figures/Experiment_YawedFlow.png}}
  \caption{Pictures showing the model setup used to allow measurements at
    arbitrary inflow angles. The Cobra probe used to measure the wake is shown in
    (a).}
  \label{fig:ExpSetup}
\end{figure}

The model has 128 pressure taps distributed on its surface to measure local
instantaneous pressure (see Fig.~\ref{fig:experimentTaps}). These pressure
values are used to compute aerodynamic lift and drag (viscous part ignored) on
the cylinder as well as pressure correlations along the span. There are 36
pressure taps at equal angular spacing of 10 degrees along each of the three
rings located on the cylinder. The three rings are labeled Right (R), Middle
(M), and Left (L) as seen in Fig.~\ref{fig:experimentTaps} (a) and are spaced
$4D$ and $5D$ apart from each other along the span. Another set of pressure
taps are located at a fixed angular location at equal spacing of $1D$ along the
span between the rings (see Fig.~\ref{fig:experimentTaps} (a,b)).

%%%%%%%%%%%%%%%%%%%%%%%%%%%%%%%%%%%%%%%%
\subsection{Data Acquisition System}
\label{sec:data_acquisition}
%%%%%%%%%%%%%%%%%%%%%%%%%%%%%%%%%%%%%%%%
%
For wake measurement, one Cobra Probe (4-hole velocity probe,
TFI\textsuperscript{\textregistered}.) mounted on a traverse system was used to
measure the velocity field behind the model (see Fig.~\ref{fig:ExpSetup} (a)).
In order to minimize the blockage effect of the traverse system, its cross
section was streamlined by using an airfoil section.  For velocity
measurements, the sampling rate was $1250$ Hz and the sampling time was $60$ s.
Wake measurements were made $2.5 D$ downstream of the model (measured from the
cylinder axis), where the turbulence intensity was lower than the maximum
allowable value (overall 30\%) for the Cobra Probe.  

Two 64-channel pressure modules (Scanivalve ZOC 33/64 Px) were utilized to
capture the local pressure. In addition, an Ethernet remote A/D system (ERAD)
was used as a collection system to read information from the ZOC. The sampling
rate and sampling time for all pressure measurements were $250$ Hz and $60$ s
respectively. The Scantel program from Scanivalve was used for pressure data
acquisition. In order to minimize the error of measurement due to the tube
length, both ZOCs were placed inside the wind tunnel near the model
(Fig.~\ref{fig:ExpSetup} (b)). The wake measurement traverse system was removed
when surface pressure measurements were made.
%
\begin{figure}[htb!]
  \centering
  \subcaptionbox{Pressure taps on the cylinder model}
    [.48\linewidth]{\incfig[width=.48\textwidth]{Figures/experiment_PressureTap.png}}
  \hspace*{\fill}
  \subcaptionbox{Distribution of pressure taps in a ring}
    [.48\linewidth]{\incfig[width=.48\textwidth]{Figures/experiment_PressureTapCrossSection.png}}
  \caption{Schematics illustrating the locations of surface pressure taps on
    the cylinder model.}
  \label{fig:experimentTaps}
\end{figure}
%%%%%%%%%%%%%%%%%%%%%%%%%%%%%%%%%%%%%%%%%%%%%%%%%%%%%%%%%%%%%

%%%%%%%%%%%%%%%%%%%%%%%%%%%%%%%%%%%%%%%%%%%%%%%%%%%%%%%%%%%%%
\section{Numerical Results and Verification with Measured Data}
\label{sec:comp_results}
%%%%%%%%%%%%%%%%%%%%%%%%%%%%%%%%%%%%%%%%%%%%%%%%%%%%%%%%%%%%%
%
The objective of this paper is to demonstrate the capability of detached eddy
simulations to predict aerodynamic loads on a static cylinder and an
elastically-mounted cylinder.  Verification with existing experimental data in
the literature, and data from new experiments conducted as a part of this
study, are presented for both the static cylinder and the elastically-mounted
cylinder. Smooth inflow is used -- zero turbulence in the numerical simulations
and the minimum possible inflow turbulence intensity ($\sim$0.2\%) in the
tunnel. Cylinder surface is very smooth and hence surface roughness is not
modeled in the simulations.

Static testing is performed for (1) flow normal to the cylinder axis, and (2)
flow at an angle to the cylinder axis (inclined cylinder); three inclination
angles are analyzed in this study. These cases are simulated at Reynolds number
$Re_D=20,000$ which is corresponding to laminar separation (LS), where flows
are laminar before separation and transition to turbulence occurs in the shear
layer.

Dynamic testing is performed for an elastically-mounted cylinder in eight
different inflow reduce velocities with (1) flow normal to the cylinder axis,
and (2) flow at an angle to the cylinder axis (inclined cylinder). However, due
to high computational cost of simulations, only one inclination angle is
analyzed in the dynamic study($45^\circ$). The cylinder is limited to vibrate
along the direction which is perpendicular to both the flow direction and the
cylinder axis. All dynamic cases are simulated at the same Reynolds number,
$Re_{D,n}=20,000$.
%%%%%%%%%%%%%%%%%%%%%%%%%%%%%%%%%%%%%%%%%%
% input section
%%%%%%%%%%%%%%%%%%%%%%%%%%%%%%%%%%%%%%%%%%%%%%%%%%%%%%%%%%%%%
\section{Static-Cylinder Results}
\label{sec:comp_results}
%%%%%%%%%%%%%%%%%%%%%%%%%%%%%%%%%%%%%%%%%%%%%%%%%%%%%%%%%%%%%
%
The objective of this paper is to demonstrate the capability of DES to predict
aerodynamic loads on a static (tethered) cylinder and an elastically-mounted
cylinder in normal and yawed flow.  This section discusses the results of the
simulations of flow over a static cylinder.

Simulations are performed for (1) flow normal to the cylinder axis, and (2)
flow at an angle to the cylinder axis (inclined/yawed cylinder); three yaw
angles ($\beta$) are analyzed in this study. Smooth inflow is used -- zero
turbulence in the numerical simulations and the minimum possible inflow
turbulence intensity ($\sim$0.2\%) in the tunnel. Cylinder surface is very
smooth and hence surface roughness is not modeled in the simulations.
Verification is performed with experimental data available in the literature,
as well as data from new experiments conducted as a part of this study.

%%%%%%%%%%%%%%%%%%%%%%%%%%%%%%%%%%%%%%%%%%
\subsection{Normally-Incident Flow}
\label{sec:normally_incident}
%%%%%%%%%%%%%%%%%%%%%%%%%%%%%%%%%%%%%%%%%%
%
Table~\ref{tab:comparisonRe20k} summarizes the simulation results for the
static cases and compares them with two sets of experimental data. Exp-I refers
to the data from~\cite{norberg2013pressure} and Exp-ISU is from our
measurements. Simulations are performed at $Re_D=20,000$, which is the same as
Exp-I, but the $Re_D$ in Exp-ISU is higher ($=51,500$).  The Strouhal number
($St$), the mean drag coefficient, $\overline{C}_d$ and the mean back pressure
coefficient, $\overline{C}_{pb}$ are compared in the table. Strouhal number is
defined as $St = f_v\,D/V_\infty$, where $f_v$ is the vortex-shedding
frequency, $D$ is the cylinder diameter, and $V_\infty$ is the freestream flow
speed.

Figure~\ref{fig:Cp_compared_Re20k} compares the predicted mean aerodynamic
pressure coefficient, $\overline{C}_p$ and the root mean square of perturbation
pressure coefficient, $C_{p'rms} = \sqrt{(\,\overline{C^2_p} -
\overline{C}^2_p\,)}$ with the data from the two experiments. The predicted
$\overline{C}_p$ agrees very well with the data from Exp-I; Exp-ISU data shows
slightly lower $\overline{C}_p$ than observed in Exp-I and the simulation,
after $100^\circ$, and the mean back pressure, $\overline{C}_{pb}$ is lower as
well. The predicted $C_{p'rms}$ distribution lies in between the two
measurements. Both measurements as well as the simulation show the peak of
$C_{p'rms}$ to be around $80^\circ$, which indicates the location of flow
separation. The predicted distribution over the cylinder surface agrees well
with the measurements.
%
\begin{table}[htb!]
  \caption{Summary of results for normally-incident flow simulations} 
  \label{tab:comparisonRe20k} 
  \begin{center}
  \begin{tabular}{c|c|c|c|c}
      $\boldsymbol{Re_D}$ & \textbf{Method} & $\boldsymbol{St}$ & $\boldsymbol{\overline{C}_d}$ & $\boldsymbol{\overline{C}_{pb}}$ \\ \hline
      \hline
      20,000 & $k$-$\omega$ DDES & 0.21 & 1.13 & -1.16 \\ \hline
      20,000 & Exp-I    & 0.19 & 1.22 & -1.1  \\ \hline
      51,500 & Exp-ISU         & 0.21 & 1.14 & -1.3  \\ \hline
      \hline
  \end{tabular}
  \end{center}
\end {table}

\begin{figure}[htb!]
  \centering
  \subcaptionbox{mean pressure coeff., $\overline{C}_p$}
    [.48\linewidth]{\incfig[width=.48\textwidth]{Figures/Cp_Compared.png}}
  \hspace*{\fill}
  \subcaptionbox{r.m.s. of pressure coeff., $C_{p'rms}$ }
    [.48\linewidth]{\incfig[width=.48\textwidth]{Figures/CpRMS_Compared.png}}
  \caption{Comparisons of mean and rms of aerodynamic pressure coefficient
  between the simulation and experimental measurements.}
  \label{fig:Cp_compared_Re20k}
\end{figure}

Figure~\ref{fig:velocity_Normal_Re20k} plots the predicted and measured wake
velocity profiles at the axial station, $x/D=2$; the cylinder axis is located
at $x/D=0$. The peak wake deficit and the wake profile are predicted
accurately. The measured data shows a slight asymmetry in the data, which is
perhaps due to an asymmetry in the experimental setup (the distance from the
tunnel wall between the top and bottom surfaces of the cylinder is slightly
different). The simulation data is averaged over 120 wake shedding cycles and
the experimental data is averaged over 540 cycles.
%
\begin{figure}[htb!]
  \incfig[width=.5\textwidth]{Figures/velocity_Normal_Re20k.png}
  \caption{Comparison of predicted and measured velocity profiles in the
    cylinder wake $2D$ downstream of the cylinder axis. $u$ is the streamwise
    component of velocity and $y$ is normal to the flow direction and the cylinder
    span.}
  \label{fig:velocity_Normal_Re20k}
\end{figure}

Figure~\ref{fig:force_20k} presents predicted temporal variation of sectional
lift and drag coefficients ($C_l$ and $C_d$). As expected for a circular
cylinder, the mean lift coefficient ($\overline{C}_l$) is zero but the mean
drag coefficient ($\overline{C}_d$) is finite. The high-frequency oscillations
which are apparent in $C_l$ time history, are due to K\'arm\'an vortex
shedding. The Strouhal number ($St$) is $\sim 0.2$ as expected for bluff bodies
for the $Re_D$ considered here (see~\cite{travin2000detached}
and~\cite{norberg2013pressure}). In addition to the oscillations at the
vortex-shedding frequency ($f_v$), the entire signal appears to modulate at a
frequency which is an order of magnitude lower than $f_v$. This modulation has
a certain randomness to it and is not perfectly periodic. This modulation
phenomenon has been reported elsewhere, see e.g.,~\cite{travin2000detached}. 
%
\begin{figure}[htb!]
  \incfig[width=0.9\textwidth]{Figures/force_20k.png}
  \caption{Predicted temporal variations of lift and drag coefficients}
  \label{fig:force_20k}
\end{figure}

Figure~\ref{fig:St_Compared_Re20k} (a) compares the power spectral densities of
$C_l$ between data from Exp-ISU and the simulation. Non-dimensional frequency,
$k=fD/V_\infty$ is used to plot the spectra. The lift in the measurements is
obtained by integrating the surface pressure measured using the pressure taps.
Figure~\ref{fig:St_Compared_Re20k} (b) presents the DES computed spectra of
$C_d$. Because vortex shedding alternates between the top and bottom sides of
the cylinder, one vortex shedding period contains two cycles of drag but only
one cycle of lift. This can be seen in Figure~\ref{fig:St_Compared_Re20k},
where the spectral peak for $C_l$ occurs at $f_v$ while the spectral peak for
drag is at $2\,f_v$. Both measurement and prediction agree very well with
each other and show the peak for lift to be around $f_v$ corresponding to
$k=St\sim0.2$.
%
\begin{figure}[htb!]
  \centering
  \subcaptionbox{PSD of $C_l$ }
    [.49\linewidth]{\incfig[width=.49\textwidth]{Figures/St_Compared_Cl_Re20k.pdf}}
  \hspace*{\fill}
  \subcaptionbox{PSD of $C_d$}
    [.49\linewidth]{\incfig[width=.49\textwidth]{Figures/St_Compared_Cd_Re20k.pdf}}
    \caption{Comparison of predicted and measured power spectral densities
      (PSDs) of $C_l$ and $C_d$. The
      measured data here is from our experiments (Exp-ISU).}
\label{fig:St_Compared_Re20k}
\end{figure}

The peak frequency and its first three harmonics that occur at $k = 0.4, 0.6,
\& \,0.8$, are identified in the figure using vertical grid lines and labeled
as $2f_v,\,3f_v,\,\&\,4f_v$. The prediction and experiment both show a second,
smaller peak in the lift spectrum at the third harmonic ($k=0.6$).  Since the
lift vector alternates with the side the vortex sheds from, only odd harmonics
of $f_v$ (i.e., $3f_v,~5f_v,\ldots$) are expected in the spectra. Therefore, no
peak is observed in the lift spectra at the second harmonic ($k=0.4$) or higher
{\em even} harmonics in either the measured or the simulated data. The spectral
shape of the PSD of $C_l$ is correctly predicted, although the predicted
magnitude is slightly higher than the measured data.

%%%%%%%%%%%%%%%%%%%%%%%%%%%%%%%%%%%%%
\subsection{Yawed Flow (Inclined Cylinder)}
\label{sec:inclined_cylinder}
%%%%%%%%%%%%%%%%%%%%%%%%%%%%%%%%%%%%%
%
The schematic in Fig.~\ref{fig:yawedModel} illustrates the setup for the
inclined-cylinder simulations. The relative inclination of the cylinder axis
with respect to the flow is obtained by yawing the flow rather than inclining
the cylinder; these simulations are therefore also referred to as yawed-flow
simulations. Other than yawing the inflow, the setup is exactly the same as for
normally-incident flow.

Yaw angle, $\beta$ is defined as the angle between the inflow velocity vector
$\boldsymbol{V_\infty}$ and the $x$ axis; the cylinder is aligned with the $z$
axis (see Fig.~\ref{fig:yawedModel}. The normal component of the flow velocity
is $V_n=V_\infty \cos\,\beta$ and the spanwise component is $V_z = V_\infty
\,\sin\,\beta$, where $V_\infty=\abs{\boldsymbol{V_\infty}}$. The computational
domain is $L=10\times D$ long in the spanwise direction to investigate spanwise
variation of aerodynamic forces.
%
\begin{figure}[htb!]
  \incfig[width=0.6\textwidth]{Figures/yawedModel.png}
  \caption{A schematic of the computational setup for static inclined cylinder
    simulations. The right figure is a cross-sectional view.  The inflow is set
    to an angle with respect to the cylinder axis, which stays aligned with the $z$
    axis of the coordinate system.}
  \label{fig:yawedModel}
\end{figure}

Table~\ref{tab:comparisonYawedRe20k} summarizes the Strouhal number ($St$) and
the back pressure coefficient ($C_{pb}$) for four different flow yaw angles,
$\beta=0,15,30,\,\&\,45$ degrees. The velocity component normal to the cylinder
axis ($V_n$) is used as the reference velocity scale to define a new set of
non-dimensional quantities, such as Reynolds number, $Re_{D,n} = \rho V_n D /
\mu$, Strouhal number, $St_{n}=f_v\,D /V_n$, and aerodynamic pressure
coefficient, $\overline{C}_{p,n}=2 (\overline{p}-p_\infty)/(\rho V_n^2)$. The
the mean back pressure coefficient normalized in this manner is labeled as
$\overline{C}_{pb,n}$. The measured value of $\overline{C}_{pb,n}$ is lower
than that predicted by the simulations (see
Table~\ref{tab:comparisonYawedRe20k}).
%
%%%%%%%%%%%%%%%%
\begin{table}[htb!]
  \caption{Summary of simulation results for four different flow yaw angles
  ($\beta=0, 15, 30,~\&~45$ deg). Experimental data is only shown for
  $\beta=30^\circ$.} 
  \label{tab:comparisonYawedRe20k} 
  \begin{center}
  \begin{tabular}{c|c|c|c|c|c}
      \textbf{Method} & \textbf{flow angle,} $\boldsymbol{\beta}$ & $\boldsymbol{Re_D}$ & $\boldsymbol{Re_{D,n}}$  & $\boldsymbol{St_{n}}$ & $\boldsymbol{\overline{C}_{pb,n}}$  \\ \hline
      \hline
      Simulation  & $0^\circ$  & 20,000 & 20,000 & 0.21 & -1.15 \\ \hline
      Simulation  & $15^\circ$ & 20,000 & 19,318 & 0.21 & -1.11 \\ \hline
      Simulation  & $30^\circ$ & 20,000 & 17,320 & 0.20 & -1.11  \\ \hline
      \rowcolor[gray]{.9}
      Exp-ISU     & $30^\circ$ & 51,500 & 44,600 & 0.19 & -1.27  \\ \hline
      Simulation  & $45^\circ$ & 20,000 & 14,142 & 0.21 & -1.16  \\ \hline
      \hline
  \end{tabular}
  \end{center}
\end{table}

Figure~\ref{fig:Cp_Compared_Yawed_Exp-ISU} compares with measured data the
predicted mean aerodynamic pressure coefficient ($\overline{C}_{p,n}$) and root
mean square of perturbation pressure coefficient, $C_{p'rms,n}$ for
$\beta=30^\circ$. The predicted back pressure ($\overline{C}_{pb,n}$) is found
to be slightly higher than Exp-ISU data, which is consistent with the
observation for the normally-incident flow cases. The predicted $C_{p'rms,n}$
distribution agrees very well with measurement, especially for
$\theta<120^\circ$, where $\theta$ is the angular position on the cylinder
surface measured from upstream. The peak of $C_{p'rms,n}$ is observed around
$80^\circ$ in both experiment and simulation, which is indicative of the
location of separation of the shear layer. For $\theta > 120^\circ$, the
measured data shows higher $C_{p'rms,n}$ than predicted by the simulations. A
similar underprediction is observed in the normally-incident flow case.
%
\begin{figure}[htb!]
  \centering
  \subcaptionbox{mean pressure coeff.,$\overline{C}_{p,n}$}
    [.48\linewidth]{\incfig[width=.48\textwidth]{Figures/Cp_Compared_Yawed_Exp-ISU.png}}
  \hspace*{\fill}
  \subcaptionbox{rms of perturbation pressure, $C_{p'rms,n}$ }
    [.48\linewidth]{\incfig[width=.48\textwidth]{Figures/CpRMS_Compared_Yawed_Exp-ISU.png}}
  \caption{Comparisons between simulation and experimental measurements for
    $\beta=30^\circ$ yawed-flow case.}
\label{fig:Cp_Compared_Yawed_Exp-ISU}
\end{figure}

%Figure~\ref{fig:velocity_Yawed30_Re20k.png} presents predicted and measured
%wake velocity profiles for $\beta=30^\circ$ case at $x/D=2$. The predicted peak
%wake deficit matches remarkably well with Exp-ISU data. However, the
%experimental data shows higher overshoots in streamwise velocity in the shear
%layer region than predicted by the simulations. The experimental data is also
%very slightly asymmetric, which is likely due to the fact that the cylinder is
%located closer to one side of the tunnel wall. It should be noted that the
%asymmetry is very small and the wall effects are minimal.
%%
%\begin{figure}[htb!]
%  \incfig[width=.5\textwidth]{Figures/velocity_Yawed30_Re20k.png}
%  \caption{Comparison of predicted and measured velocity profiles for
%    $\beta=30^\circ$ yawed flow in the cylinder wake $2D$ downstream of the cylinder axis}
%  \label{fig:velocity_Yawed30_Re20k.png}
%\end{figure}
%
%Figure~\ref{fig:St_Compared_Yawed30_Exp-ISU} compares the power spectral
%densities of the transverse force coefficient (along the $y$ axis), $C_{y,n}$
%for $\beta=30^\circ$ case between Exp-ISU data and predictions, where $C_{y,n}
%= 2\, F_y/ \left( \rho V^2_n (D \times L) \right)$, $F_y$ is the net force over
%the entire cylinder; longitudinal force coefficient, $C_{x,n}$ is similarly
%defined. In the simulation, the first peak is observed around $St_{p,n}\sim
%0.2$, which is the same as for the normally-incidence flow case (see
%Figure~\ref{fig:St_Compared_Re20k} (a)). The spectral shape is correctly
%predicted by the simulation although the measured curve appears to be shifted
%along the $x$ axis; this is likely due to a scaling factor in frequency (log
%scale is used for frequency in Fig.~\ref{fig:St_Compared_Yawed30_Exp-ISU}),
%arising perhaps from a slight mismatch in the measurement of the inflow
%velocity in the experiment.
%%
%\begin{figure}[htb!]
%  \incfig[width=.6\textwidth]{Figures/St_Compared_Yawed30_Exp-ISU.png}
%  \caption{Comparison of predicted and experimental power spectral densities
%    (PSDs) of force coefficient $C_{y,n}$ for $\beta=30^\circ$ yawed-flow
%    cases.}
%  \label{fig:St_Compared_Yawed30_Exp-ISU}
%\end{figure}

Figure~\ref{fig:Compared_Yawed_Re20k} compares the predicted mean aerodynamic
pressure coefficient ($\overline{C}_{p,n}$), for four different values of
inflow yaw angle, $\beta$. The distribution of $\overline{C}_{p,n}$ is found to
be very similar irrespective of $\beta$;~\cite{zdravkovich2003flow} refers to
this as `independence principle' (IP). IP is also observed in the power
spectral densities of the transverse force coefficient, $C_{y,n}$ for the same
set of values of $\beta$ analyzed. $C_{y,n}=2\, f_y/ (\rho V^2_n)$, where $f_y$
is force per unit projected area ($f_y = F_y/(L\,D)$) in the $y$ direction, and $C_{x,n}$
is correspondingly defined for the $x-$component of force.
Figure~\ref{fig:Compared_Yawed_Re20k} (b) shows that the spectra collapse when
$V_n$ is used to normalize the force coefficients and the frequency; the
abscissa in Fig.~\ref{fig:Compared_Yawed_Re20k} (b) is $k_n = fD/V_n$.  
%
\begin{figure}[htb!]
  \subcaptionbox{Mean pressure coeff., $\overline{C}_{p,n}$}[0.48\linewidth]
    {\incfig[width=.48\textwidth]{Figures/Cp_Compared_Yawed_Re20k.png}}
  \hspace*{\fill}
  \subcaptionbox{PSDs of $C_{y,n}$}[0.53\linewidth]
    {\incfig[width=.53\textwidth]{Figures/St_Compared_Yawed_Cl_Re20k.pdf}}
  \caption{Independence principle: comparisons of (a) $\overline{C}_{p,n}$, and
    (b) power spectral densities (PSDs) of $C_{y,n}$ between predictions for various $\beta$ values.}
\label{fig:Compared_Yawed_Re20k}
\end{figure}

Figure~\ref{fig:Spatial_temporal_CxCyRe20k} shows spatio-temporal plots of the
force coefficients $C_{x,n}$ and $C_{y,n}$. The contours show that $C_{x,n}$
and $C_{y,n}$ vary along the span, indicating that vortex shedding does not
occur simultaneously along the entire span. A spatial drift from left to right
with increasing time can be seen in the contours (more visible in the $C_{x,n}$
spatio-temporal plot) which is indicative of spanwise flow over the cylinder.
%
\begin{figure}[htb!]
  \subcaptionbox{$C_{y,n}$ }
    [.48\linewidth]{\incfig[width=.48\textwidth]{Figures/Spatial_temporal_Cy_Re20k_Yawed30.png}}
  \subcaptionbox {$C_{x,n}$}
    [.48\linewidth]{\incfig[width=.48\textwidth]{Figures/Spatial_temporal_Cx_Re20k_Yawed30.png}}
  \hspace*{\fill}
  \caption{Spatio-temporal distribution of force coefficients at
  $\beta=30^\circ$}
  \label{fig:Spatial_temporal_CxCyRe20k}
\end{figure}

Figure~\ref{fig:Coherence_Yawed30_Re20k} presents coherence of force
coefficients for $\beta=30^\circ$ case. Magnitude squared coherence,
$\gamma^2(\Delta z)$ is defined as
%
\begin{align}
  \gamma^2(\Delta z) &= \frac{\langle \abs{S_{xy}}^2 \rangle}{\langle
    S_{xx}\rangle \langle S_{yy} \rangle},
  \label{eq:coherence}
\end{align}
%
where $S_{xy}$ denotes cross-spectral density of the quantity ($C_{x,n}$ or
$C_{y,n}$) at two points separated by a distance $\Delta z$ along the span, and
$S_{xx},~S_{yy}$ are auto-spectral densities; angular brackets denote ensemble
averaging, however ergodicity assumption is used to relate that to time
averaging. The coherence plot of $C_{y,n}$ indicates that spanwise correlation
is very high (over nearly the entire cylinder span) at the vortex shedding
frequency, but is small at other frequencies, which is expected based on
literature. $C_{x,n}$ however is not that highly correlated along the span even
at the peak vortex shedding frequency.
%
\begin{figure}[htb!]
  \subcaptionbox{$C_{y,n}$ }
    [.48\linewidth]{\incfig[width=.48\textwidth]{Figures/Cl_coherence_Yawed30_Re20k.png}}
  \subcaptionbox {$C_{x,n}$}
    [.48\linewidth]{\incfig[width=.48\textwidth]{Figures/Cd_coherence_Yawed30_Re20k.png}}
  \hspace*{\fill}
    \caption{Magnitude squared coherence, $\gamma^2(\Delta z)$ of transverse
      and longitudinal force coefficients - $C_{y,n}$ and $C_{x,n}$ for
      $\beta=30^\circ$ case.}
  \label{fig:Coherence_Yawed30_Re20k}
\end{figure}
%%%%%%%%%%%%%%%%%%%%%%%%%%%%%%%%%%%%%%%%%%%%%%%%%%


% input section
%%%%%%%%%%%%%%%%%%%%%%%%%%%%%%%%%%%%%%%%%%%%%%%%%%
\subsection{Vortex-Induced Vibration (VIV)}
\label{sec:VIV}
%%%%%%%%%%%%%%%%%%%%%%%%%%%%%%%%%%%%%%%%%%%%%%%%%%
%
A schematic of the computational setup for the vortex-induced vibration (VIV)
simulations is presented in Figure~\ref{fig:VIVmodel}. The setup is the same as
for the static simulations except for an additional forced mass-spring-damper
system. The cylinder is allowed to move only in the $y$ (cross-stream)
direction. The $y$ component of the integrated aerodynamic surface force on the
cylinder (denoted by $F_y$) drives the mass-spring-damper system given by
%
\begin{equation}
  m \frac{{\rm d}^2 y}{{\rm d}t^2} + c \frac{{\rm d} y}{{\rm d}t} + k y = F_y(t).
  \label{eq:mass-spring-damper}
\end{equation}
 
In the simulations the mass ratio $m^*=m/(\rho {\cal V})=2.6$, where $m$ is the
mass of the cylinder, ${\cal V}=\pi (D^2/4) . S$ is the volume of the cylinder,
$S$ is the cylinder span, and $\rho$ is the density of the fluid flowing over
the cylinder. The mechanical damping ratio of the system $\zeta = c/c_c$ is
$0.001$ where, $c_c=2\sqrt{k.m}$ is the critical damping, and the spring
stiffness $k$ is obtained from the natural frequency, $f_N$ as
$k=m(2\pi\,f_N)^2$.  The values of these parameters are selected to match the
measurements presented in~\cite{franzini2013one}.  This measurement dataset is
referred as Exp II in this paper. The predictions are also compared to another
dataset reported in~\cite{khalak1997fluid}, which is referred as Exp III here.
The mass ratio and damping ratio used in Exp III are slightly different
($m^*=2.4$ and $\zeta=0.0045$) from Exp II and the simulations. It should be
noted that the measurement results have end effects due to the finite length of
the cylinder. {\color{red} The experiments were conducted in water channels
\ldots check aspect ratio and differences between Exp II and III} The
simulations use periodic boundary conditions in the span direction, which
theoretically simulates an infinite span. However, span-periodicity can induce
artificial effects if the spanwise coherence is greater than the simulated
span.
%
\begin{figure}[htb!]
  \incfig[width=.6\textwidth]{Figures/VIV_setup.jpg}
  \caption{A schematic of the computational setup for oscillating cylinder
    simulations. The right figure is a cross-sectional view.  The inflow is set
    to an angle with respect to the cylinder axis, which stays aligned with the $z$
    axis of the coordinate system.}
  \label{fig:VIVmodel}
\end{figure}

Figure~\ref{fig:VIV_amp} compares the predicted non-dimensional
mean amplitude $\bar{A}/D$ with the measurements of Exp II and III over a wide
range of reduced velocity $V_{R,n}$, which is defined as
$V_{R,n}=V_n/(f_N\,D)$, where $f_N$ is the natural frequency of the system. The
subscript $n$ refers to the component of the vector normal to the cylinder axis
to accommmodate for yawed flow. Two different yaw angle flows are evaluated --
$\beta=0^\circ$ and $45^\circ$ -- both at $Re_{D,n}=20,000$. Reduced velocity
is the inverse of Strouhal number, $V_{R,n} = 1/St$.

\citet{khalak1997fluid} identified the following four distinct branches in
their VIV measurements for the zero-yaw case: the ``initial excitation''
branch, the ``upper'' branch, the ``lower'' branch, and the
``desynchronization'' branch. These are labeled and identified with solid black
lines as best curve fits of the measured data in Fig.~\ref{fig:VIV_amp} (a). In
the {\em initial excitation} branch, the mean amplitude grows rapidly with $V_{R,n}$.
The scaled displacement oscillation amplitude ($A/D$) reaches a peak in the
{\em upper} branch, drops to 60\% of the peak value in the {\em lower} branch, and then
finally drops to a negligible value at higher $V_{R,n}$ in the
{\em desynchronization} branch. The current DES simulations agree very well with the
data (particularly with Exp III) in the {\em initial excitation} and {\em upper} branches.
The peak amplitude is well captured and occurs around $V_{R,n}=4.76$, which
corresponds to the peak vortex shedding Strouhal number, $St=0.21$ for a
stationary cylinder. The predicted amplitude is slightly lower than the
measurements in the {\em lower} and {\em desynchronization} branches. Considering the
relatively large differences in the two sets of measurements (between Exp II
and Exp III), the prediction accuracy of the simulations is very good.
%
\begin{figure}[htb!]
  \centering
  \subcaptionbox{$\beta=0^\circ$} {\incfig[width=.47\textwidth]{fig/viv_amp_noyaw.pdf}}
  \qquad
  \subcaptionbox{$\beta=45^\circ$}{\incfig[width=.45\textwidth]{fig/viv_amp_yaw45_wInset.pdf}} \\
    \caption{Comparison of predicted and measured non-dimensional mean
      amplitude $A/D$ over a range of reduced velocities $V_{R,n}$ for a)
      $\beta=0^\circ$, and b) $\beta=45^\circ$. The inset in the plot on the right
      shows the two setups (UP and DN) used in Exp II for yawed-flow
      measurements.} 
  \label{fig:VIV_amp}
\end{figure}

Figure~\ref{fig:VIV_amp} (b) compares the predicted VIV amplitude with the
measurements from Exp II. The measurements were taken for two different
configurations of the cylinder, shown in the inset in the figure. Since the top
and bottom surfaces are not the same, the two configurations are not identical
and the measured data for the two configurations shows a large difference.
Since the setup in the experiment is asymmetric (wall at the bottom and free
surface on the top) and the fact that end plates were not used, end effects
(finite-span effect) might be the reason for the observed differences between
the two configurations. The predictions agree better with the ``Up''
configuration in the {\em initial excitation} and {\em upper} branches, and
with the ``Dn'' configuration in the {\em lower} branch. Exp II did not collect
any data at higher $V_{R,n}$ to test the prediction accuracy in the {\em
desynchronization} branch.

Figure~\ref{fig:VIV_yaw} investigates the effect of yaw angle on the amplitude
of VIV. The data from Exp II is plotted in Fig.~\ref{fig:VIV_yaw} (a) and the
DES predictions in Fig.~\ref{fig:VIV_yaw} (b). The DES predictions show little
difference between $\beta=0^\circ$ and $45^\circ$ except around $V_{R,n}=4$ and
$6$, where $A/D$ is highly sensitive to changes in $V_{R,n}$. As far as the
general variation of $A/D$ with $V_{R,n}$ (as shown in the plot with lines) is
considered, the DES predictions for both yaw angles show the same behavior,
suggesting that the independence principle also holds for VIV. From the
measurements however, one can only conclude that yaw-independence holds
primarily in the Initial Excitation branch. The large difference in the data
between the Up and Dn configurations for $\beta=45^\circ$ limits the
verification of yaw-independence in VIV predicted by DES to the {\em initial
excitation} branch.
%
\begin{figure}[htb!]
  \subcaptionbox{Exp II}{\incfig[width=0.45\textwidth]{fig/viv_independence_Exp}}
  \qquad
  \subcaptionbox{DES}   {\incfig[width=0.47\textwidth]{fig/viv_independence}} \\
  \caption{Scaled displacement amplitude ($A/D$) for $\beta=0^\circ$ and
    $45^\circ$ degrees for a range of $V_{R,n}$ obtained from a) measurements
    from Exp II, and b) DES predictions with the data from Exp III for
    $\beta=0^\circ$ as a guide.}
  \label{fig:VIV_yaw}
\end{figure}

Figure~\ref{fig:VIV_freq} compares the normalized frequency ($f/f_N$) between
DES predictions and measurements from Exp III over a range of $V_{R,n}$. The
natural frequency, $f_N$ is determined from the mechanical properties (spring
and mass) of the cylinder system. The frequency $f$ is determined using the
peak transverse unsteady loading on the cylinder ($F_y$) in the simulations,
but in the experiments, it is determined using the transverse displacement of
the cylinder. The dynamical system of the cylinder is driven by $F_y$ and the
peak frequency corresponding to $F_y$ is selected to be $f$ for DES. This
information is not available for the measurements, and hence the frequency
corresponding to peak displacement is selected as $f_N$. The purple dashed line
corresponds to the vortex shedding frequency for a static cylinder
($St_p=0.21$), and the horizontal black dashed line presents the natural
frequency ($f=f_N$). The simulations capture the peak frequency well in all the
branches.

In the {\em initial excitation} branch, the cylinder oscillates at the driving
frequency ($f$) and $f/f_N$ increases linearly with the reduced velocity
$V_{R,n}$. Beyond resonance, which occurs at $f=f_N$, the oscillating frequency
is ``locked-in'' with the natural frequency of the system. This occurs in the
{\em upper} and {\em lower} branches identified in Fig.~\ref{fig:VIV_amp}. A
strict lock-in would imply $f/f_N=1$ for the range of $V_{R,n}$ during which
the system is locked in. However, for a low mass-damping system such as the one
considered here, the lock-in frequency is higher than the natural frequency,
i.e., $f/f_N>1$. As $V_{R,n}$ increases beyond lock-in, the system again starts
oscillating at the driving frequency and $f/f_N$ follows the purple line in the
figure. Figure~\ref{fig:VIV_freq} also plots the simulation results for
$\beta=45^\circ$, which are consistent with the results for $\beta=0^\circ$,
further confirming yaw-independence in the predictions.
%
\begin{figure}[htb!]
  \incfig[width=.5\textwidth]{fig/viv_freq.pdf}
  \caption{Non-dimensional frequency $f/f_N$ for various reduced velocities
    $V_{R,n}$}
  \label{fig:VIV_freq}
\end{figure}

As seen in Figs.~\ref{fig:VIV_amp} and~\ref{fig:VIV_yaw}, the amplitude jumps
up from the {\em initial excitation} branch to the {\em upper} branch, and
jumps down from the {\em upper} branch to the {\em lower} branch as $V_{R,n}$
is increased. These jumps occur as the mode in which the vortex shedding occurs
switches between two possible configurations. These are shown in
Fig.~\ref{fig:VIV_Q} using iso-surface of the Q-criterion as well using
schematics. In the `2S mode', two single vortices shed alternately from either
side of the cylinder in one cycle of cylinder motion, while in the `2P mode',
two pairs of vortices are shed from each side of the cylinder in one cycle.
Such 2P mode was observed in the smoke visualizations
of~\cite{brika1993vortex}. In the {\em initial excitation} branch, vortex
shedding occurs in the 2S mode, while in the {\em lower} branch, it switches to
the 2P mode.  In the {\em upper} branch of the lock-in regime, the vortex
shedding switches between the 2S and 2P modes. Other modes can also be observed
(e.g., 2P+S and P+S modes) in forced vibration motion.
%
\begin{figure}[htb!]
  \subcaptionbox{Q-criterion iso-surfaces; $V_{R,n}=4$}{\incfig[width=.48\textwidth]{Figures/Q_45_RV4_1.png}} \qquad
  \subcaptionbox{Q-criterion iso-surfaces; $V_{R,n}=8$}{\incfig[width=.48\textwidth]{Figures/Q_45_RV8_1.png}} \\
%
  \subcaptionbox{2S mode; $V_{R,n}=4$}{\incfig[width=.48\textwidth]{Figures/2S_RV4.png}} \qquad
  \subcaptionbox{2P mode; $V_{R,n}=8$}{\incfig[width=.48\textwidth]{Figures/2P_RV8.png}} \\
    \caption{Iso-surfaces of the Q-criterion and schematics of the two modes of
      vortex shedding observed in the simulations. The plots are shown for the
      $\beta=45^\circ$ case.}
  \label{fig:VIV_Q}
\end{figure}
%
%The force coefficients are shown in Fig.~\ref{fig:force_VIV}. It is well known
%that the vibration motion can significantly increase the fluctuation of forces.
%Mean transverse force coefficient $\bar{C}_{x,n}$ for reduce velocities
%$V_{R,N} < 5.9$ have a very similar curve as mean amplitude $\bar{A}/D$ in
%Fig.~\ref{fig:VIV_amp}. However, mean amplitude $\bar{A}/D$ is almost constant
%in the {\em lower} branch while $\bar{C}_{x,n}$ continuously declines. The predicted
%rms of longitudinal force coefficients $C_{y,n,rms}$ has very sharp peak at
%$V_{R,n}=4$, which is slightly different from Exp. II. Similar to $\bar{A}/D$
%at $V_{R,n}=4$, $C_{y,n,rms}$ for $\beta=45^\circ$ is much larger than
%$\beta=0^\circ$. Overall, two yawed angle flow simulations have very similar
%results for small amplitude vibrations. Independent principle can be applied to
%forces coefficients except for the {\em upper} branch ($4 \leqslant V_{R,n}\leqslant
%5.9$).
%%
%\begin{figure}[htb!]
%  \subcaptionbox {$C_{x,n}$}
%    [.48\linewidth]{\incfig[width=.48\textwidth]{Figures/Cd_VIV.png}}
%  \hspace*{\fill}
%  \subcaptionbox{$C_{y,n,rms}$ }
%    [.48\linewidth]{\incfig[width=.48\textwidth]{Figures/Cl_VIV.png}}
%    \caption{Mean transverse and rms of longitudinal force coefficients, $\bar{C}_{x,n}$ and $C_{y,n,rms}$ for
%      $\beta=0^\circ$ and $45^\circ$ cases.}
%  \label{fig:force_VIV}
%\end{figure}



%%%%%%%%%%%%%%%%%%%%%%%%%%%%%%%%%%%%%%%%%%
\section{Conclusion}
\label{sec:conclusions}
%%%%%%%%%%%%%%%%%%%%%%%%%%%%%%%%%%%%
%
A computational methodology based on a $k-\omega$ delayed detached eddy
simulation (DDES) model and in-house experiments are used to investigate
aerodynamic loading on a smooth circular cylinder. Simulations are performed
for the cylinder in normally-incident flow (static and dynamic) and yawed flow
(3 cases). The computational methodology for predicting aerodynamic loading on
the cylinder is verified against experimental data in normally-incident flow
($\beta=0^\circ$) and yawed flow ($\beta=30^\circ$). The agreement between the
simulations and the experiments for normally-incident flow is very good, and
the results of the yawed flow simulation with $\beta=30^\circ$ in reasonable
agreement with the experiment. Overall, these comparisons show that the
computational methodology is able to accurately predict aerodynamic loading on
a static, smooth circular cylinder in smooth inflow.

Comparisons of simulation results for different flow angles ($\beta$) show that
the aerodynamic loads do not vary with yaw angle when the loads and frequency
are non-dimensionalized using the component of the flow velocity normal to the
cylinder axis. This indifference to yaw angle, referred to as the independence
principle, is observed for yawed flow up to $45^\circ$.

The numerical study of VIV of an elastically-mounted cylinder in
normally-incident flow agrees well with Exp III, but show some differences with
Exp II. The difference might be caused by the end effects of the experiment.
Other than the upper branch, the numerical results for two different yawed flow
($\beta=0^\circ$ and $45^\circ$) show reasonable agreement, which indicates
independence principle is applicable for the most regime of VIV, except for the
upper branch.

%%%%%%%%%%%%%%%%%%%%%%%%%%%%%%%%%%%%
\section{Acknowledgments}
\label{sec:acknowledgement}
%%%%%%%%%%%%%%%%%%%%%%%%%%%%%%%%%%%%
Funding for this research is provided by the National Science Foundation (Grant
\#NSF/ CMMI-1537917). Computational resources are provided by NSF XSEDE (Grant
\#TG-CTS130004) and the Argonne Leadership Computing Facility, which is a DOE
Office of Science User Facility supported under Contract DE-AC02-06CH11357.

%%%%%%%%%%%%%%%%%%%%%%%%%%%%%%%%%%%%
%\section{References}
%\label{sec:references}
\bibliographystyle{elsarticle-harv} 
\bibliography{references}
%%%%%%%%%%%%%%%%%%%%%%%%%%%%%%%%%%%%

\end{document}
